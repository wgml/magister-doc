\chapter{Wstęp}
\label{cha:intruduction}

%TODO Przed tym dodać abstrakt PL i EN (na jednej stronie)

Przetwarzanie obrazów i ich sekwencji stanowi pole rozległych badań. %TODO naukowych i przemysłowych 
W ich ramach, projektowane są algorytmy umożliwiające akwizycję, modyfikowanie, analizę i prezentację obrazów. %TODO jeszcze rozpoznawanie treści
Często motywacją badań jest próba naśladowania zjawisk związanych z narządem wzroku człowieka i dążenie do uzyskania takiego opisu zjawisk, aby umożliwić wykonanie zbliżonego do nich algorytmu przy użyciu układów elektronicznych. %TODO zjawisk -> sposobu działania, drugie zjawisk jest jakieś zbyt ogólne
Odmiennym zagadnieniem jest poszukiwanie możliwości realizacji przetwarzania obrazów w taki sposób, aby uzyskać informacje niewidoczne dla ludzkiego oka. %TODO tu mógłby Pan rozwinąć, co ma Pan na myśli...

Techniki przetwarzania obrazów opierają się zwykle na analizie i redukcji informacji związanej z sekwencją pikseli w taki sposób, aby uzyskać obraz wynikowy, na którym uwypuklone będą kluczowe z punktu widzenia algorytmu własności. %TODO informacji zawartej w sekwencji.. 
Wynikiem działania procedury może być również zbiór cech opisujących badane zjawiska.
%TODO W tym miejscu bym przytoczył klasyczny podział na przetwarzanie, analizę i rozpoznawanie i krótko omówił...

Techniki przetwarzania obrazów, a zwłaszcza ich sekwencji znajdują zastosowanie w coraz większej liczbie przypadków. %TODO przypadków -> złe słowo.
Jedną z dziedzin wykorzystujących techniki wizyjne, która ulega prężnemu rozwojowi w ostatnich latach jest budowa systemów ADAS (\emph{ang.} Advanced driver-assistance systems). %TODO ulega rozwojowi -> średnie...
Celem projektowania zaawansowanych systemów wsparcia kierowcy jest stopniowe zwiększanie autonomii pojazdów i ograniczenie zaangażowania kierowcy. 
Wśród wielu czujników wymaganych do poprawnego działania takich systemów znaleźć można czujniki wizyjne, których sygnały są następnie przetwarzane w celu uzyskania informacji na temat jezdni, innych uczestników ruchu, oznakowania czy potencjalnych zagrożeń. 
Opracowanie współcześnie stosowanych technik znaleźć można w pracach \cite{Bengler2014,Velez2017}.
%TODO OK, choć kilka zdań więcej też by można. 1. Wprost nazwać co wchodzi w skład ADAS (jakie podsystemy) 2. Coś o pojazdach autonomicznych.,

Inny zbiór technik wykorzystywany jest w celu rozpoznawania twarzy oraz emocji. %TODO może jeszcze detekcji twarzy.
Zagadnienie to znajduje zastosowanie w ramach projektowania nie tylko systemów przemysłowych, ale jest również powszechnie stosowane w oprogramowaniu współcześnie dostępnych aparatów cyfrowych czy w ramach serwisów społecznościowych. 
Analizę wykorzystywanych w tym celu algorytmów znaleźć można w pracy \cite{Anil2016}.
%TODO też nieco szerzej o tych aplikacjach

Współcześnie, coraz większe znaczenie mają również systemy śledzenia przechodniów i analizy ich działań w celu wykrycia zachowań niepożądanych. %TODO przechodniów -> osób. działań -> zachowań
Motywując to zwiększeniem bezpieczeństwa, badane są takie zagadnienia jak detekcja porzuconych bagaży, obecność osób nieuprawnionych w ustalonych strefach czy śledzenie ruchu przy użyciu wielu kamer. %TODO jeszcze re-identyfikacja
Systemy te mogą działać niezależnie lub stanowić jeden z elementów zintegrowanego oprogramowania, wykorzystującego dane z wielu źródeł \cite{Sriram2016,Hussain2016,Gouo2015}.
%TODO O tym monitoringu też 2 zdania więcej. Przede wszystkim z czego wynika potrzeba automatyzacji.

Równolegle do rozwoju algorytmów wizyjnych, badane są możliwości wykorzystania ich w rzeczywistych systemach, uruchamianych na układach elektronicznych różnego typu. %TODO rzeczywistych -> a co te algorytmy na wirutalnych są rozwijane. Tzw. wiem o co Panu chodzi, ale zły dobór słów.
Algorytmy wizyjne projektowane są z myślą o uruchamianiu na powszechnie stosowanych układach procesorowych w architekturach x86 lub ARM, mikrokontrolerach, układach ASIC i FPGA. %TODO rozwinąć skróty

Pośród wymienionych platform wyróżnić można rodzinę Zynq \cite{zybo-reference-manual}, integrującą możliwości układów FPGA oraz procesorów ARM. 
Dzięki zastosowaniu układu programowalnego, możliwe jest projektowanie algorytmów wizyjnych wykonywanych w sposób strumieniowy, zapewniając wysoką wydajność przy stosunkowo niskim zapotrzebowaniu na energię. %TODO może logiki programowalnej...
Uzupełnieniem takiego układu jest procesor ARM, umożliwiający wykorzystanie algorytmów, które wymagają swobodnego dostępu do kontekstu obliczeniowego. %TODO to jest jeden z aspektów, drugi to dominajca instrukcji, 3 trudność implementacji sprzętowe, itp

Układy Zynq pozwalają wykorzystać zalety algorytmów projektowanych z myślą o implementacji przy użyciu języków HDL (\emph{ang.} Hardware Description Language) oraz powszechnie stosowanych języków proceduralnych. 
Układ ten pozwala na uruchomienie systemu operacyjnego, ze szczególnym uwzględnieniem systemu PetaLinux \cite{petalinux-tools}, dzięki czemu możliwy jest dostęp do szerokiego zbioru narzędzi związanych z powszechnie stosowanymi systemami operacyjnymi.

\section{Cel pracy}

Celem niniejszej pracy jest uruchomienie oraz konfiguracja systemu PetaLinux na platformie Zynq, ze szczególnym uwzględnieniem funkcjonalności, które mogą zostać wykorzystane we wbudowanych systemach wizyjnych. %TODO jest -> było

Przeprowadzić należy analizę architektury układu oraz dostępnych systemów operacyjnych i systemów czasu rzeczywistego. %TODO W pierwszym etapie przeprowadzono....
Następnie, opracować należy zagadnienia teoretyczne i praktyczne związane z funkcjonalnościami systemu, które mogą znaleźć zastosowanie we wbudowanych systemach wizyjnych. %TODO oprawcowano

Działanie komponentów powinno zostać zaprezentowane na przykładzie wybranego systemu wizyjnego. %TODO Ostateczenie działanie ... 

\section{Zawartość pracy}

Materiał zebrano w czterech rozdziałach. %TODO praca została podzielona na 4 rozdziały (5), bo zwykle wstęp też się ujmuje (choć post factum)

Rozdział \ref{cha:platform} zawiera opis i analizę platformy Zynq-7000. Omówiono krótko specyfikację układu. Poruszono zagadnienia związane z dostępnymi systemami operacyjnymi, z uwzględnieniem zalet i wad każdego z proponowanych rozwiązań. Zbadano również możliwość wykorzystania systemów czasu rzeczywistego.

Rozdział \ref{cha:functionalities} zawiera analizę funkcjonalności układu, które mogą zostać wykorzystane w systemach wizyjnych. Zbadano możliwości wykorzystania systemu operacyjnego PetaLinux i jego integracji z układem reprogramowalnym. Opisano również protokół AXI, ze szczególnym uwzględnieniem modułów AXI DMA (\emph{ang.} Direct Memory Access) oraz VDMA (\emph{ang.} Video DMA).

W rozdziale \ref{cha:project} zaprezentowano wnioski związane z wykorzystaniem systemu PetaLinux w systemie wizyjnym, którego zadaniem jest generacja tła. Zaproponowano metody integracji rozwiązań budowanych na dwóch niezależnych platformach, wskazano ograniczenia i potencjalne kierunki rozwoju. %TODO co to są te dwie niezależne platformy ? Tez oprocz wnioskow sam system. Trochę to inaczej ująć...

Rozdział \ref{cha:vivado-conf} zawiera zbiór instrukcji związanych z konfiguracją funkcjonalności omawianych w poprzednich rozdziałach, na przykładzie układu ZYBO. Zaprezentowano w nim kroki wymagane do poprawnej konfiguracji wykorzystywanych elementów systemu oraz wskazano metody umożliwiające weryfikację poprawności działania.