\chapter{Podsumowanie}
\label{chap:summary}

%TODO TK - Do powtórnej lektury



Celem niniejszej pracy była analiza możliwości systemu operacyjnego PetaLinux uruchomionego na układzie z~rodziny Zynq i próba wykorzystania go w zagadnieniach związanych z wbudowanymi systemami wizyjnymi. 
Użycie systemu operacyjnego oferuje większy, względem aplikacji typu \textit{bare-metal}, zasób możliwości -- w tym komunikację sieciową, metody przechowywania danych, a także realizację zaawansowanych zadań algorytmicznych dzięki zastosowaniu zewnętrznych bibliotek. 

W ramach pracy zbadano możliwości wykorzystania istniejących systemów operacyjnych do działania na badanej platformie. 
Omówiono wady i zalety każdego ze znanych autorowi rozwiązań. 
Poruszono również zagadnienie realizacji obliczeń w czasie rzeczywistym w układzie Zynq. 
Zagadnienia te zebrano w rozdziale \ref{cha:platform}. %TODO powt. zagadnienia.
%TODO Konkrety: zbudowano i uruchomiono: Peta, Ubuntu i ze zródeł.

Kolejnym krokiem było opracowanie funkcjonalności, które mogą znaleźć zastosowanie w projektach związanych z~przetwarzaniem obrazów i sekwencji wizyjnych. 
Zebrano informacje teoretyczne oraz wnioski z zastosowań praktycznych protokołu AXI oraz wyjaśniono jego rolę w projektowaniu współczesnych układów obliczeniowych realizowanych na platformie Zynq.
Wyniki analizy teoretycznej oraz wnioski uzyskane na etapie projektowania implementacji zebrano w rozdziale \ref{cha:functionalities}. 

Na podstawie informacji zebranych w dwóch wymienionych powyżej rozdziałach, zaproponowano metodę realizacji algorytmu generacji tła w logice reprogramowalnej. %TODO no nie...  zrealizowano przykładowy system wizyjny -- segmentację obikeów pierwsszoplanowych. Został on podziaelony pomiędzy logikę programowlaną () i system procesorowy. 
Wynik działania modułu był następnie transmitowany do pamięci operacyjnej procesora ARM przy użyciu mechanizmu AXI VDMA. 
Aplikacja działająca pod kontrolą systemu PetaLinux była odpowiedzialna za końcową analizę obrazu i archiwizację wyników.
%TODO i to dopasować.

W ramach realizacji projektu zaproponowano metodę buforowania pełnej ramki obrazu przy użyciu modułu AXI VDMA. 
Zaprojektowano również moduł wyznaczający różnicę dwóch kolejnych klatek obrazu, który może stanowić element składowy rozbudowanych systemów wizyjnych.
W trakcie realizacji projektu napotkano na ograniczenia, w tym niewystarczającą liczbę zasobów logicznych oraz niewielką wydajność procedur algorytmicznych uruchamianych na procesorze ARM, w wyniku których nie zbudowano kompletnego systemu wizyjnego wykorzystującego proponowane techniki.
%TODO Raczej działające w czasie rzeczywistym.
Opis projektu i wnioski z jego realizacji zebrano w rozdziale \ref{cha:project}.

Zebrano również informacje związane z praktycznym wykorzystaniem omawianych funkcjonalności w projektach wizyjnych. 
Przedstawiono metody konfiguracji kolejnych modułów w ujęciu ogólnym, możliwe do wykorzystania w trakcie projektowania rozwiązań algorytmicznych na dowolny układ rodziny Zynq. 
Informacje te znaleźć można w rozdziale \ref{cha:vivado-conf}.

Materiał zebrany i przedstawiony w ramach niniejszej pracy może posłużyć za podstawę do realizacji zaawansowanych algorytmów wizyjnych z wykorzystaniem platformy Zynq i systemu operacyjnego PetaLinux. 
Przedstawione techniki mogą zostać wykorzystane do rozwoju istniejących, jak i projektowania nowych aplikacji.

Zbiór przedstawionych w ramach pracy technik nie wyczerpuje możliwości badanego systemu operacyjnego. 
W ujęciu ogólnym, PetaLinux, lub inny system operacyjny działający na układzie z rodziny Zynq, pozwala na projektowanie aplikacji, które różnią się w znacznym stopniu od programów wykorzystywanych w systemach wbudowanych o ograniczonych możliwościach. 
Zastosowanie zaawansowanych technik projektowania aplikacji pozwali na uzyskanie efektu zbliżonego do programów stosowanych w życiu codziennym, na przykład dzięki wykorzystaniu protokołów komunikacji sieciowej.
Innym kierunkiem rozwoju może być zastosowanie systemów czasu rzeczywistego we współpracy z klasycznym systemem operacyjnym do realizacji obliczeń algorytmicznych z uwzględnieniem rygoru czasowego.

Zaproponowany moduł generacji tła może być dalej rozbudowywany i wykorzystywany w złożonych aplikacjach realizowanych na platformie Zynq.
Zbiór funkcjonalności badanych w ramach pracy nie wyczerpuje możliwości dostępnych w systemie PetaLinux i pokrewnych. Dalszy rozwój badań może pozwolić na zwiększenie praktyczności użycia systemu do realizacji zadań przetwarzania obrazów dzięki zdefiniowaniu obszernego zbioru zasobów możliwych do wykorzystania na etapie projektowania aplikacji.
%TODO Jakaś powtórka.
%TODO Jeszcze dopracować te dalsze kierunki

%TODO Przed tymi istniejącymy dodatkami jeszcze dodatek A - spis zawartości CD.
%TODO Spis treści przed dodatki.
% OK, dodałem dodatek, treść wypełnię później
% pracę sformatowałem zgodnie z tym, co znalazłem w zasadach dyplomowania.
% przenieść spis przed dotatki?

%TODO - tzn. spis literatury :) nie treści :)