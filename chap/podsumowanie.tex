\chapter{Podsumowanie}
\label{chap:summary}

Celem niniejszej pracy była analiza możliwości systemu operacyjnego PetaLinux uruchomionego na układzie rodziny Zynq i próba wykorzystania go w zagadnieniach związanych z wbudowanymi systemami wizyjnymi. Użycie systemu operacyjnego oferuje, względem aplikacji typu bare-metal, większy zasób możliwości -- w tym komunikację sieciową, metody przechowywania danych, a także zaawansowane zadania algorytmiczne, dzięki zastosowaniu zewnętrznych bibliotek.

W rozdziale \ref{cha:platform} autor przedstawił analizę układów rodziny Zynq. Zbadano możliwości wykorzystania istniejących systemów operacyjnych do działania na badanej platformie. Omówiono wady i zalety każdego ze znanych autorowi rozwiązań. Poruszono również zagadnienie realizacji obliczeń w czasie rzeczywistym na układzie Zynq.

W rozdziale \ref{cha:functionalities} zebrano opis szeregu funkcjonalności, które, z perspektywy autora, znaleźć mogą zastosowanie w projektach realizujących zadania przetwarzania obrazów i sekwencji wizyjnych. Omówiono protokół AXI i wyjaśniono jego rolę w projektowaniu współczesnych układów obliczeniowych, realizowanych na platformie FPGA. W ramach rozdziału zebrano wiadomości teoretyczne, a także informacje praktyczne związane z implementacją badanych funkcjonalności.

Rozdział \ref{cha:project} zawiera wnioski z realizacji proponowanych projektów wizyjnych przy użyciu omawianych funkcjonalności. Autor zaproponował sposób realizacji podstawowego modułu wyznaczającego różnicę dwóch kolejnych obrazów należących do sekwencji wizyjnej. W dalszej części rozdziału, zaproponowano  metodę wykorzystania modułu w systemach wizyjnych wymagających buforowania pełnych klatek obrazu na przykładzie modułu generacji tła. Ze względu na ograniczenia przedstawione w omawianym rozdziale, nie zrealizowano kompletnej aplikacji wizyjnej wykorzystującej proponowane techniki.

W rozdziale \ref{cha:vivado-conf} zebrano informacje związane z praktycznym wykorzystaniem omawianych funkcjonalności w projektach wizyjnych. Przedstawiono metody konfiguracji kolejnych modułów w ujęciu ogólnym, nie związanym z konkretnym układem obliczeniowym.

Zdaniem autora, materiał zebrany i przedstawiony w ramach niniejszej pracy może posłużyć za podstawę do realizacji zaawansowanych algorytmów wizyjnych z wykorzystaniem platformy Zynq i systemu operacyjnego PetaLinux. Przedstawione techniki mogą zostać wykorzystane w sposób uniwersalny do rozwoju istniejących, jak i projektowania nowych aplikacji.

Zbiór przedstawionych przez autora w ramach pracy technik nie wyczerpuje możliwości badanego systemu operacyjnego. W ujęciu ogólnym, PetaLinux, lub inny system operacyjny działający na układzie rodziny Zynq, pozwala na projektowanie aplikacji, którye różnią się w znacznym stopniu od programów wykorzystywanych w systemach wbudowanych o ograniczonych możliwościach. Zastosowanie zaawansowanych technik projektowania aplikacji pozwali na uzyskanie efektu zbliżonego do programów stosowanych w życiu codziennym, na przykład dzięki wykorzystaniu protokołów komunikacji sieciowej.

Innym kierunkiem rozwoju może być zastosowanie systemów czasu rzeczywistego we współpracy z klasycznym systemem operacyjnym do realizacji obliczeń algorytmicznych z uwzględnieniem rygoru czasowego.