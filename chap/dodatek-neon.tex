\section{Aplikacja w architekturze NEON}
\label{cha:neon-source}

%TODO by się przydał "lekki" komentarz
% OK

Na listingach poniżej zaprezentowano implementację procedury wyznaczającej wartość iloczynu skalarnego dwóch wektorów o zadanej długości. 
Listing \ref{lis:dot-product-base}. zawiera implementację bazową, wykorzystującą podstawowe operacje dostępne w języku C.
Kod listingu \ref{lis:dot-product-neon}. wykorzystuje funkcjonalności modułu NEON w celu potencjalnego zmaksymalizowania wydajności operacji. 
Obie implementacje porównano z procedurą napisaną w asemblerze, wykorzystującą instrukcje koprocesora VFP, przedstawioną na listingu \ref{lis:dot-product-asm}.

Sposób wykorzystania procedur w programie przedstawiono na listingu \ref{lis:dot-product-main}. 
Kompilacja programu wymaga aktywacji przełączników odpowiedzialnych za obsługę instrukcji NEON. 
Wykorzystano również techniki optymalizacji udostępniane w kompilatorze \textit{gcc}. 
Polecenie kompilacji przedstawiono poniżej:

\begin{lstlisting}[breaklines=true]
arm-linux-gnueabihf-gcc -Wall -O3 -mcpu=cortex-a9 -mfpu=neon -ftree-vectorize -mvectorize-with-neon-quad -mfloat-abi=hard -ffast-math -funsafe-math-optimizations -g -c -o "main.o" "main.c"
\end{lstlisting}

\begin{lstlisting}[breaklines=true, language=C, caption=Implementacja bazowa., label=lis:dot-product-base]
float dot_product(float *first, float *second, unsigned int len) {
	float sum = 0.0;
	for (unsigned int i = 0; i < len; i++)
		sum += first[i] * second[i];
	return sum;
}
\end{lstlisting}

\begin{lstlisting}[breaklines=true, language=C, caption=Implementacja w architekturze NEON. (Źródło: \cite{xilinx-neon}), label=lis:dot-product-neon]
float dot_product_neon(float * restrict first, float * restrict second, unsigned int len) {
	float32x4_t vec1_q, vec2_q;
	float32x4_t sum_q = {0.0, 0.0, 0.0, 0.0};
	float32x2_t tmp[2];
	float result;
	for( int i=0; i<( len & ~3); i+=4 )
	{
		vec1_q=vld1q_f32(&first[i]);
		vec2_q=vld1q_f32(&second[i]);
		sum_q = vmlaq_f32(sum_q, vec1_q, vec2_q );
	}
	tmp[0] = vget_high_f32(sum_q);
	tmp[1] = vget_low_f32 (sum_q);
	tmp[0] = vpadd_f32(tmp[0], tmp[1]);
	tmp[0] = vpadd_f32(tmp[0], tmp[0]);
	result = vget_lane_f32(tmp[0], 0);
return result;
}
\end{lstlisting}

\begin{lstlisting}[breaklines=true, language=C, caption=Implementacja w asemblerze. (Źródło: \cite{dot-product-asm}), label=lis:dot-product-asm]
float dot_product_asm(float * restrict first, float * restrict second, unsigned int len) {
	float net1D=0.0f;
	asm volatile (
		"vmov.f32 q8, #0.0"
		"1:"
		"subs %3, %3, #4"
		"vld1.f32 {d0,d1}, [%1]!"
		"vld1.f32 {d4,d5}, [%2]!"
		"vmla.f32 q8, q0, q2"
		"bgt 1b"
		"vpadd.f32 d0, d16, d17"
		"vadd.f32 %0, s0, s1"
		: "=w"(net1D)
		: "r"(first), "r"(second), "r"(len)
		: "q0", "q2", "q8");
	return net1D;
}
\end{lstlisting}

\begin{lstlisting}[breaklines=true, language=C, caption=Główna procedura programu., label=lis:dot-product-main]
int main() {
	float first[] = {1.1, 2.2, 3.3};
	float second[] = {-1.1, -2.2, -3.3};
	
	float result_base = dot_product(first, second, 3);
	float result_neon = dot_product_neon(first, second, 3);
	float result_asm = dot_product_asm(first, second, 3);
	
	return 0;
}
\end{lstlisting}