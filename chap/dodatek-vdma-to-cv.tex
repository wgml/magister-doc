\section{Konwersja danych pomiędzy VDMA i OpenCV}
\label{sec:vdma-to-opencv}

Na listingu \ref{lis:vdma-to-opencv-code} zaprezentowano metodę konwersji sygnału wizyjnego pomiędzy elementami obliczeniowymi wykonanymi w architekturach FPGA i ARM.

\begin{lstlisting}[language=C++, label=lis:vdma-to-opencv-code, caption=Konwersja sygnału wizyjnego pomiędzy AXI VDMA i \texttt{cv::Mat}.]
#include "opencv2/core/core.hpp"

cv::Mat const from_vdma(unsigned char *ptr, std::size_t width, std::size_t height,
	std::size_t bytes_per_pixel)
{
	return cv::Mat(height, width, CV_8UC(bytes_per_pixel), ptr);
}

void to_vdma(cv::Mat const &image, std::size_t bytes_per_pixel, unsigned char *ptr)
{
	assert(image.isContinuous());
	if (ptr != image.ptr())
		std::memcpy(ptr, image.ptr(),
			image.rows * image.cols * bytes_per_pixel);
}

int main(int, char**)
{
	const std::size_t width = 1280, height = 720, bytes_per_pixel = 4;
	unsigned char framebuffer_ptr[width * height * bytes_per_pixel];
	
	for(int i =0; i < 10000;i++)
	{
		cv::Mat image = from_vdma(framebuffer_ptr,
			width, height, bytes_per_pixel);
		
		// opcjonalna kopia
		image = image.clone();
		
		algorithm(image);
		
		to_vdma(image, bytes_per_pixel, framebuffer_ptr);
		
		await_next_frame();
	}
	return 0;
}

\end{lstlisting}