\renewcommand{\abstractname}{Streszczenie}
\begin{abstract}
	
Celem pracy było uruchomienie i konfiguracja systemu operacyjnego PetaLinux na platformie Zynq.
Zasadniczą uwagę poświęcono funkcjonalnościom, które mogą zostać wykorzystane do realizacji wbudowanych systemów wizyjnych.
W ramach pracy przeprowadzono analizę wybranych funkcji systemu operacyjnego i zrealizowano przykładowy algorytm generacji tła.
Opracowano także sposób komunikacji pomiędzy modułami logiki programowalnej, a procesorem obliczeniowym z wykorzystaniem interfejsu AXI.
Użyto funkcjonalności systemu operacyjnego do analizy i prezentacji wyników działania aplikacji.
Metody omówione w pracy pozwalają na budowę zaawansowanych aplikacji wykorzystujących możliwości logiki programowalnej i procesora ARM oraz integrację obu metod w jednym projekcie.

\bigskip

\begin{center}\textbf{Abstract}\end{center}.



The thesis discusses issues related to running and configuring a~PetaLinux operating system on the Zynq platform.
The research was focused of functionalities which can be incorporated into embedded vision systems.
Analysis of selected PetaLinux features was conducted and a~sample background generation algorithm was implemented as a result.
Possibilities of communication between programmable logic modules and CPU through AXI interface were analysed.
The operating system features were used to perform video analysis and results presentation.
Methods discussed in this thesis enable users to create complex applications integrating programmable logic and ARM processor features in one project.

\end{abstract}
