\renewcommand{\abstractname}{Streszczenie}
\begin{abstract}
Celem pracy było uruchomienie i konfiguracja systemu operacyjnego PetaLinux na platformie Zynq.
Uwagę poświęcono funkcjonalnościom, które mogą zostać wykorzystane do realizacji wbudowanych systemów wizyjnych.
W ramach pracy przeprowadzono analizę wybranych funkcji systemu operacyjnego i zrealizowano algorytm generacji tła na bazie kontekstu dwóch ramek sygnału wizyjnego.
Przeanalizowano możliwość komunikacji pomiędzy modułami logiki programowalnego a procesorem obliczeniowym z wykorzystaniem interfejsu AXI.
Wykorzystano funkcjonalności systemu operacyjnego do analizy i prezentacji wyników działania aplikacji.
Metody omówione w pracy pozwalają na budowę zaawansowanych aplikacji wykorzystujących możliwości logiki programowalnej i procesora ARM oraz integrację obu metod w jednym projekcie.

\bigskip
The thesis discuses attempts of running and configuring PetaLinux operating system on the Zynq platform.
Research was focused of functionalities which can be incorporated into embedded vision systems.
Analysis of selected PetaLinux features was conducted and algorithm of background generation was implemented as a result.
Possibilities of communication between programmable logic modules and CPU through AXI interface were analyzed.
Operating system features were used to perform video analysis and results presentation.
Methods discussed in thesis enable user to create complex applications integrating programmable logic and ARM processor features in one project.

\end{abstract}
