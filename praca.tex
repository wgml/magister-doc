\documentclass[11pt]{aghdpl}
% \documentclass[en,11pt]{aghdpl}  % praca w języku angielskim

% Lista wszystkich języków stanowiących języki pozycji bibliograficznych użytych w pracy.
% (Zgodnie z zasadami tworzenia bibliografii każda pozycja powinna zostać utworzona zgodnie z zasadami języka, w którym dana publikacja została napisana.)
\usepackage[english,polish]{babel}

% Użyj polskiego łamania wyrazów (zamiast domyślnego angielskiego).
\usepackage{polski}

\usepackage[utf8]{inputenc}

% dodatkowe pakiety

\usepackage{mathtools}
\usepackage{amsfonts}
\usepackage{amsmath}
\usepackage{amsthm}

% --- < bibliografia > ---

\usepackage[
style=numeric,
sorting=none,
%
% Zastosuj styl wpisu bibliograficznego właściwy językowi publikacji.
language=autobib,
autolang=other,
% Zapisuj datę dostępu do strony WWW w formacie RRRR-MM-DD.
urldate=iso8601,
% Nie dodawaj numerów stron, na których występuje cytowanie.
backref=false,
% Podawaj ISBN.
isbn=true,
% Nie podawaj URL-i, o ile nie jest to konieczne.
url=false,
%
% Ustawienia związane z polskimi normami dla bibliografii.
maxbibnames=3,
% Jeżeli używamy BibTeXa:
backend=bibtex
]{biblatex}

\usepackage{csquotes}
% Ponieważ `csquotes` nie posiada polskiego stylu, można skorzystać z mocno zbliżonego stylu chorwackiego.
\DeclareQuoteAlias{croatian}{polish}

\addbibresource{bibliografia.bib}

% Nie wyświetlaj wybranych pól.
%\AtEveryBibitem{\clearfield{note}}


% ------------------------
% --- < listingi > ---

% Użyj czcionki kroju Courier.
\usepackage{courier}

\usepackage{listings}
\lstloadlanguages{TeX}

\lstset{
	literate={ą}{{\k{a}}}1
           {ć}{{\'c}}1
           {ę}{{\k{e}}}1
           {ó}{{\'o}}1
           {ń}{{\'n}}1
           {ł}{{\l{}}}1
           {ś}{{\'s}}1
           {ź}{{\'z}}1
           {ż}{{\.z}}1
           {Ą}{{\k{A}}}1
           {Ć}{{\'C}}1
           {Ę}{{\k{E}}}1
           {Ó}{{\'O}}1
           {Ń}{{\'N}}1
           {Ł}{{\L{}}}1
           {Ś}{{\'S}}1
           {Ź}{{\'Z}}1
           {Ż}{{\.Z}}1,
	basicstyle=\footnotesize\ttfamily,
}

% ------------------------

\AtBeginDocument{
	\renewcommand{\tablename}{Tabela}
	\renewcommand{\figurename}{Rys.}
}

% ------------------------
% --- < tabele > ---

\usepackage{array}
\usepackage{tabularx}
\usepackage{multirow}
\usepackage{booktabs}
\usepackage{makecell}
\usepackage[flushleft]{threeparttable}

% defines the X column to use m (\parbox[c]) instead of p (`parbox[t]`)
\newcolumntype{C}[1]{>{\hsize=#1\hsize\centering\arraybackslash}X}


%---------------------------------------------------------------------------

\author{Wojciech Gumuła}
\shortauthor{W. Gumuła}

%\titlePL{Przygotowanie bardzo długiej i pasjonującej pracy dyplomowej w~systemie~\LaTeX}
%\titleEN{Preparation of a very long and fascinating bachelor or master thesis in \LaTeX}

\titlePL{Wykorzystanie systemu operacyjnego Linux we wbudowanych systemach wizyjnych zrealizowanych na platformie Zynq.}
\titleEN{The use of the Linux operating system in embedded vision systems implemented on the Zynq platform.}


\shorttitlePL{System operacyjny Linux na platformie Zynq.} % skrócona wersja tytułu jeśli jest bardzo długi
\shorttitleEN{Linux OS on the Zynq platform.}

\thesistype{Praca dyplomowa magisterska}
%\thesistype{Master of Science Thesis}

\supervisor{dr Tomasz Kryjak}
%\supervisor{Marcin Szpyrka PhD, DSc}

\degreeprogramme{Automatyka i Robotyka}
%\degreeprogramme{Computer Science}

\date{2017}

\department{Katedra Automatyki i Inżynierii Biomedycznej }
%\department{Department of Applied Computer Science}

\faculty{Wydział Elektrotechniki, Automatyki,\protect\\[-1mm] Informatyki i Inżynierii Biomedycznej}
%\faculty{Faculty of Electrical Engineering, Automatics, Computer Science and Biomedical Engineering}

\acknowledgements{Serdecznie dziękuję Adamie Giży za bycie Giżą.}


\setlength{\cftsecnumwidth}{10mm}

%---------------------------------------------------------------------------
\setcounter{secnumdepth}{4}
\brokenpenalty=10000\relax

\begin{document}

\titlepages

% Ponowne zdefiniowanie stylu `plain`, aby usunąć numer strony z pierwszej strony spisu treści i poszczególnych rozdziałów.
\fancypagestyle{plain}
{
	% Usuń nagłówek i stopkę
	\fancyhf{}
	% Usuń linie.
	\renewcommand{\headrulewidth}{0pt}
	\renewcommand{\footrulewidth}{0pt}
}

\setcounter{tocdepth}{2}
\tableofcontents
\clearpage

\chapter{Wstęp}
\label{cha:intruduction}

%TODO Przed tym dodać abstrakt PL i EN (na jednej stronie)

Przetwarzanie obrazów i ich sekwencji stanowi pole rozległych badań naukowych i przemysłowych. %TODO naukowych i przemysłowych 
% OK

W ich ramach, projektowane są algorytmy umożliwiające akwizycję, modyfikowanie, analizę, rozpoznawanie treści i prezentację obrazów. %TODO jeszcze rozpoznawanie treści
% OK
Często motywacją badań jest próba naśladowania zjawisk związanych z narządem wzroku człowieka i dążenie do uzyskania takiego opisu sposobu działania, aby umożliwić wykonanie zbliżonego do nich algorytmu przy użyciu układów elektronicznych. %TODO zjawisk -> sposobu działania, drugie zjawisk jest jakieś zbyt ogólne
% OK
Odmiennym zagadnieniem jest poszukiwanie możliwości realizacji przetwarzania obrazów w taki sposób, aby uzyskać informacje niewidoczne dla ludzkiego oka, w oparciu o parametry obrazu o niewielkiej zmienności. Temat ten obejmuje analizę obrazów w celu wykrycia możliwych modyfikacji obrazu oryginalnego czy algorytmy częstotliwościowe. %TODO tu mógłby Pan rozwinąć, co ma Pan na myśli...
% OK

Techniki przetwarzania obrazów opierają się zwykle na analizie i redukcji informacji zawartej w sekwencji pikseli w taki sposób, aby uzyskać obraz wynikowy, na którym uwypuklone będą kluczowe z punktu widzenia algorytmu własności. %TODO informacji zawartej w sekwencji.. 
% OK
Wynikiem działania procedury może być również zbiór cech opisujących badane zjawiska.
%TODO W tym miejscu bym przytoczył klasyczny podział na przetwarzanie, analizę i rozpoznawanie i krótko omówił...
% OK

Zdefiniować można szereg operacji składających się na proces przetwarzania obrazu \cite{Tadeusiewicz1997}.
\begin{itemize}
	\item Akwizycja -- przygotowanie cyfrowej reprezentacji obrazu, czytelnej dla układu obliczeniowego.
	
	\item Przetwarzanie -- proces modyfikacji danych wejściowych w celu przystosowania do obróbki algorytmicznej. Wykorzystujący, między innymi, operacje skalowania, zmiany przestrzeni barw czy usuwania zakłóceń.
	
	\item Analiza -- redukcja informacji wizyjnej w celu uzyskania opisu jakościowego lub ilościowego badanych cech i eliminacja zbędnych z perspektywy rozpatrywanego zadania informacji.
	
	\item Rozpoznawanie -- proces uzyskiwania informacji wynikowych na podstawie wektora cech.
\end{itemize}

Techniki przetwarzania obrazów, a zwłaszcza ich sekwencji znajdują zastosowanie w coraz większej liczbie dziedzin. %TODO przypadków -> złe słowo.
% OK?
Jedną z dziedzin wykorzystujących techniki wizyjne, która jest prężnie rozwijana w ostatnich latach jest budowa systemów ADAS (\emph{ang.} Advanced driver-assistance systems). %TODO ulega rozwojowi -> średnie...
% OK?
Ich działanie, poza sygnałami wizyjnymi, wymaga użycia sygnałów o innych charakterystykach, między innymi czujników optycznych oraz systemów \emph{LIDAR}.
Celem projektowania zaawansowanych systemów wsparcia kierowcy jest stopniowe zwiększanie autonomii pojazdów i ograniczenie zaangażowania kierowcy. W szerszej perspektywie, rozwój systemów ADAS może pozwolić na zaprojektowanie pojazdów w pełni autonomicznych, pozwalających na transport osób i towarów bez udziału kierowców.
Dane z czujników wizyjne mogą być przetwarzane w celu uzyskania informacji na temat jezdni, innych uczestników ruchu, oznakowania czy potencjalnych zagrożeń. 
Opracowanie współcześnie stosowanych technik znaleźć można w pracach \cite{Bengler2014,Velez2017}.
%TODO OK, choć kilka zdań więcej też by można. 1. Wprost nazwać co wchodzi w skład ADAS (jakie podsystemy) 2. Coś o pojazdach autonomicznych.,
% OK

Inny zbiór technik wykorzystywany jest w celu detekcji i rozpoznawania twarzy oraz badania emocji. %TODO może jeszcze detekcji twarzy.
% OK
Zagadnienie to znajduje zastosowanie w ramach projektowania nie tylko systemów przemysłowych, ale jest również powszechnie stosowane w oprogramowaniu współcześnie dostępnych aparatów cyfrowych czy w ramach serwisów społecznościowych. Metody te mogą również pozwolić na budowę systemów weryfikacji użytkownika bez konieczności zdefiniowania hasła dostępu. Znajdują także zastosowanie w interfejsach przystosowanych do pracy z osobami niepełnosprawnymi.
Analizę wykorzystywanych w tym celu algorytmów znaleźć można w pracy \cite{Anil2016}.
%TODO też nieco szerzej o tych aplikacjach
% OK

Współcześnie, coraz większe znaczenie mają również systemy śledzenia osób i analizy ich zachowań w celu wykrycia zachowań niepożądanych. %TODO przechodniów -> osób. działań -> zachowań
% OK
Motywując to zwiększeniem bezpieczeństwa, badane są takie zagadnienia jak detekcja porzuconych bagaży, obecność osób nieuprawnionych w ustalonych strefach czy śledzenie ruchu i reidentyfikacja przy użyciu wielu kamer. %TODO jeszcze re-identyfikacja
% OK
Potrzeba automatyzacji wynika ze złożoności projektowanych systemów, które zasięgiem obejmować mogą całe aglomeracje i pozwalać na obserwację zachować tysięcy osób. Z tego powodu, praktycznie niemożliwe jest zapewnienie liczby operatorów pozwalającej na wykorzystanie informacji wejściowych w pełni.
Systemy te mogą działać niezależnie lub stanowić jeden z elementów zintegrowanego oprogramowania, wykorzystującego dane z wielu źródeł \cite{Sriram2016,Hussain2016,Gouo2015}.
%TODO O tym monitoringu też 2 zdania więcej. Przede wszystkim z czego wynika potrzeba automatyzacji.
% OK

Równolegle do rozwoju algorytmów wizyjnych, badane są techniki implementacji pozwalających na wykorzystanie ich w systemach uruchamianych na układach elektronicznych różnego typu. %TODO rzeczywistych -> a co te algorytmy na wirutalnych są rozwijane. Tzw. wiem o co Panu chodzi, ale zły dobór słów.
% OK?
Algorytmy wizyjne projektowane są z myślą o uruchamianiu na powszechnie stosowanych układach procesorowych w architekturach x86 lub ARM, mikrokontrolerach, układach ASIC (\emph{ang.} Application Specific Integrated Circuit) i FPGA (\emph{ang.} Field-Programmable Gate Array). %TODO rozwinąć skróty
% OK

Pośród wymienionych platform wyróżnić można rodzinę Zynq \cite{zybo-reference-manual}, integrującą możliwości układów FPGA oraz procesorów ARM. 
Dzięki zastosowaniu logiki programowalnej, możliwe jest projektowanie algorytmów wizyjnych wykonywanych w sposób strumieniowy, zapewniając wysoką wydajność przy stosunkowo niskim zapotrzebowaniu na energię. %TODO może logiki programowalnej...
% OK
Uzupełnieniem takiego układu jest procesor ARM, umożliwiający wykorzystanie algorytmów, które wymagają swobodnego dostępu do kontekstu obliczeniowego. Procesor sekwencyjny jest również, w porównaniu do układów logicznych, lepiej przystosowany do wykonywania algorytmów zdominowanych przez instrukcje lub takich, których sprzętowa implementacja jest trudna lub niemożliwa.%TODO to jest jeden z aspektów, drugi to dominajca instrukcji, 3 trudność implementacji sprzętowe, itp
% OK

Układy Zynq pozwalają wykorzystać zalety algorytmów projektowanych z myślą o implementacji przy użyciu języków HDL (\emph{ang.} Hardware Description Language) oraz powszechnie stosowanych języków proceduralnych. 
Układ ten pozwala na uruchomienie systemu operacyjnego, ze szczególnym uwzględnieniem systemu PetaLinux \cite{petalinux-tools}, dzięki czemu możliwy jest dostęp do szerokiego zbioru narzędzi związanych z powszechnie stosowanymi systemami operacyjnymi.

\section{Cel pracy}

Celem niniejszej pracy było uruchomienie oraz konfiguracja systemu PetaLinux na platformie Zynq, ze szczególnym uwzględnieniem funkcjonalności, które mogą zostać wykorzystane we wbudowanych systemach wizyjnych. %TODO jest -> było
% OK

Na pierwszym etapie analizę architektury układu oraz dostępnych systemów operacyjnych i systemów czasu rzeczywistego. %TODO W pierwszym etapie przeprowadzono....
% OK
Następnie, opracowano należy zagadnienia teoretyczne i praktyczne związane z funkcjonalnościami systemu, które mogą znaleźć zastosowanie we wbudowanych systemach wizyjnych. %TODO oprawcowano
% OK
Ostatecznie, działanie komponentów zaprezentowano na przykładzie wybranego systemu wizyjnego. %TODO Ostateczenie działanie ... 
% OK

\section{Zawartość pracy}

Praca podzielona została na pięć rozdziałów. %TODO praca została podzielona na 4 rozdziały (5), bo zwykle wstęp też się ujmuje (choć post factum)
% OK

Rozdział \ref{cha:platform} zawiera opis i analizę platformy Zynq-7000. Omówiono krótko specyfikację układu. Poruszono zagadnienia związane z dostępnymi systemami operacyjnymi, z uwzględnieniem zalet i wad każdego z proponowanych rozwiązań. Zbadano również możliwość wykorzystania systemów czasu rzeczywistego.

Rozdział \ref{cha:functionalities} zawiera analizę funkcjonalności układu, które mogą zostać wykorzystane w systemach wizyjnych. Zbadano możliwości wykorzystania systemu operacyjnego PetaLinux i jego integracji z układem reprogramowalnym. Opisano również protokół AXI, ze szczególnym uwzględnieniem modułów AXI DMA (\emph{ang.} Direct Memory Access) oraz VDMA (\emph{ang.} Video DMA).

W rozdziale \ref{cha:project} zaprezentowano system wizyjny wykorzystujący omawiane funkcjonalności, którego zadaniem jest generacja tła. Zaproponowano metody integracji rozwiązań budowanych z wykorzystaniem obu części układu, wskazano ograniczenia i potencjalne kierunki rozwoju. %TODO co to są te dwie niezależne platformy ? Tez oprocz wnioskow sam system. Trochę to inaczej ująć...
% OK?

Rozdział \ref{cha:vivado-conf} zawiera zbiór instrukcji związanych z konfiguracją funkcjonalności omawianych w poprzednich rozdziałach, na przykładzie układu ZYBO. Zaprezentowano w nim kroki wymagane do poprawnej konfiguracji wykorzystywanych elementów systemu oraz wskazano metody umożliwiające weryfikację poprawności działania.


\printbibliography

\end{document}
