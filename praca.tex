\documentclass[11pt]{aghdpl}
% \documentclass[en,11pt]{aghdpl}  % praca w języku angielskim
\usepackage{lipsum}
% Lista wszystkich języków stanowiących języki pozycji bibliograficznych użytych w pracy.
% (Zgodnie z zasadami tworzenia bibliografii każda pozycja powinna zostać utworzona zgodnie z zasadami języka, w którym dana publikacja została napisana.)
\usepackage[english,polish]{babel}

% Użyj polskiego łamania wyrazów (zamiast domyślnego angielskiego).
\usepackage{polski}

\usepackage[utf8]{inputenc}

% dodatkowe pakiety
\usepackage{url}
\usepackage{mathtools}
\usepackage{amsfonts}
\usepackage{amsmath}
\usepackage{amsthm}
\usepackage{abstract}
\usepackage[section]{placeins}
\usepackage{subcaption}

% --- < bibliografia > ---

\usepackage[
style=numeric,
sorting=none,
%
% Zastosuj styl wpisu bibliograficznego właściwy językowi publikacji.
language=autobib,
autolang=other,
% Zapisuj datę dostępu do strony WWW w formacie RRRR-MM-DD.
urldate=iso8601,
% Nie dodawaj numerów stron, na których występuje cytowanie.
backref=false,
% Podawaj ISBN.
isbn=true,
% Nie podawaj URL-i, o ile nie jest to konieczne.
url=false,
%
% Ustawienia związane z polskimi normami dla bibliografii.
maxbibnames=3,
% Jeżeli używamy BibTeXa:
backend=bibtex
]{biblatex}

\usepackage{csquotes}
% Ponieważ `csquotes` nie posiada polskiego stylu, można skorzystać z mocno zbliżonego stylu chorwackiego.
\DeclareQuoteAlias{croatian}{polish}

\addbibresource{bibliografia.bib}

% Nie wyświetlaj wybranych pól.
%\AtEveryBibitem{\clearfield{note}}


% ------------------------
% --- < listingi > ---

% Użyj czcionki kroju Courier.
\usepackage{courier}

\usepackage{listings}
\lstset{escapeinside={(*@}{@*)}}

\usepackage[toc,page]{appendix}

\usepackage[hidelinks]{hyperref}
\usepackage{float}

\newenvironment{conditions}
{\par\vspace{\abovedisplayskip}\noindent\begin{tabular}{>{$}l<{$} @{${}-{}$} l}}
	{\end{tabular}\par\vspace{\belowdisplayskip}}

\lstloadlanguages{TeX}
\usepackage{xcolor}

\belowcaptionskip=-10pt
\lstset{
	literate={ą}{{\k{a}}}1
           {ć}{{\'c}}1
           {ę}{{\k{e}}}1
           {ó}{{\'o}}1
           {ń}{{\'n}}1
           {ł}{{\l{}}}1
           {ś}{{\'s}}1
           {ź}{{\'z}}1
           {ż}{{\.z}}1
           {Ą}{{\k{A}}}1
           {Ć}{{\'C}}1
           {Ę}{{\k{E}}}1
           {Ó}{{\'O}}1
           {Ń}{{\'N}}1
           {Ł}{{\L{}}}1
           {Ś}{{\'S}}1
           {Ź}{{\'Z}}1
           {Ż}{{\.Z}}1,
	basicstyle=\footnotesize\ttfamily,
	showstringspaces=false,
	tabsize=2,
	rulecolor=\color{gray},
	frame=single,
	xleftmargin=\fboxsep,
	xrightmargin=-\fboxsep,
	numbers=none
}

% usuwa sierotki, wymaga lualatex
\usepackage{fontspec}
\usepackage[nosingleletter]{impnattypo}
% ------------------------

\AtBeginDocument{
	\renewcommand{\tablename}{Tabela}
	\renewcommand{\figurename}{Rys.}
}

% ------------------------
% --- < tabele > ---

\usepackage{array}
\usepackage{tabularx}
\usepackage{multirow}
\usepackage{booktabs}
\usepackage{makecell}
\usepackage[flushleft]{threeparttable}

% defines the X column to use m (\parbox[c]) instead of p (`parbox[t]`)
\newcolumntype{C}[1]{>{\hsize=#1\hsize\centering\arraybackslash}X}

%---------------------------------------------------------------------------

\author{Wojciech Gumuła}
\shortauthor{W. Gumuła}

\titlePL{Wykorzystanie systemu operacyjnego Linux we wbudowanych systemach wizyjnych zrealizowanych na platformie Zynq.}
\titleEN{The use of the Linux operating system in embedded vision systems implemented on the Zynq platform.}

\shorttitlePL{System operacyjny Linux na platformie Zynq.} % skrócona wersja tytułu jeśli jest bardzo długi
\shorttitleEN{Linux operating system on the Zynq platform.}

\thesistype{Praca dyplomowa magisterska}

\supervisor{dr inż. Tomasz Kryjak}

\degreeprogramme{Automatyka i Robotyka}

\date{2017}

\department{Katedra Automatyki i Inżynierii Biomedycznej }

\faculty{Wydział Elektrotechniki, Automatyki,\protect\\[-1mm] Informatyki i Inżynierii Biomedycznej}

\acknowledgements{}


\setlength{\cftsecnumwidth}{10mm}

%---------------------------------------------------------------------------
\setcounter{secnumdepth}{4}
\brokenpenalty=10000\relax

\begin{document}

\titlepages

\fancypagestyle{plain}
{
	% Usuń nagłówek i stopkę
	\fancyhf{}
	% Usuń linie.
	\renewcommand{\headrulewidth}{0pt}
	\renewcommand{\footrulewidth}{0pt}
}

\renewcommand{\abstractname}{}
\begin{abstract}
\lipsum[1]

\bigskip

\lipsum[1]
\end{abstract}


\setcounter{tocdepth}{1}
\tableofcontents
\clearpage

\chapter{Wstęp}
\label{cha:intruduction}


Przetwarzanie obrazów i ich sekwencji stanowi pole rozległych badań naukowych i przemysłowych.
W ich ramach, projektowane są algorytmy umożliwiające akwizycję, modyfikowanie, analizę, rozpoznawanie treści i prezentację obrazów.
Często motywacją badań jest próba naśladowania zjawisk związanych z narządem wzroku człowieka i dążenie do uzyskania takiego opisu sposobu jego działania, aby umożliwić wykonanie zbliżonego do niego algorytmu przy użyciu układów elektronicznych. 
Odmiennym zagadnieniem jest poszukiwanie możliwości realizacji przetwarzania obrazów w taki sposób, aby uzyskać informacje niewidoczne dla ludzkiego oka, w oparciu o parametry obrazu o niewielkiej zmienności. Temat ten obejmuje analizę obrazów w celu wykrycia możliwych modyfikacji obrazu oryginalnego czy algorytmy wykorzystujące analizę widmową sygnału. 

Techniki przetwarzania obrazów opierają się zwykle na analizie i redukcji informacji zawartej w sekwencji pikseli w taki sposób, aby uzyskać obraz wynikowy, na którym uwypuklone będą kluczowe z punktu widzenia algorytmu własności.
Wynikiem działania procedury może być również zbiór cech opisujących badane zjawiska.

Zdefiniować można szereg operacji składających się na proces przetwarzania obrazu \cite{Tadeusiewicz1997}.
\begin{itemize}
	\item Akwizycja -- przygotowanie cyfrowej reprezentacji obrazu, ,,czytelnej'' dla układu obliczeniowego.
	
	\item Przetwarzanie -- proces modyfikacji danych wejściowych w celu przystosowania do obróbki algorytmicznej, wykorzystujący, między innymi, operacje skalowania, zmiany przestrzeni barw czy usuwania zakłóceń (filtracji).
	
	\item Analiza -- redukcja informacji wizyjnej w celu uzyskania opisu jakościowego lub ilościowego badanych cech i eliminacja zbędnych z perspektywy rozpatrywanego zadania informacji.
	
	\item Rozpoznawanie -- proces uzyskiwania informacji wynikowych na podstawie wektora cech.
\end{itemize}

Techniki przetwarzania obrazów, a zwłaszcza ich sekwencji, znajdują zastosowanie w coraz większej liczbie dziedzin nauki i~techniki.
Jedną z nich jest prężnie rozwijana w ostatnich latach budowa systemów ADAS (\emph{ang.} Advanced driver-assistance systems).
Ich działanie, poza sygnałami wizyjnymi, wymaga użycia sygnałów o innych charakterystykach, między innymi czujników optycznych oraz systemów \emph{LIDAR} (\emph{ang.} Light Detection and Ranging). 

Celem projektowania zaawansowanych systemów wsparcia kierowcy jest stopniowe zwiększanie autonomii pojazdów i ograniczenie zaangażowania kierowcy. W szerszej perspektywie, rozwój systemów ADAS może pozwolić na zaprojektowanie pojazdów w pełni autonomicznych, pozwalających na transport osób i towarów bez udziału kierowców.
Dane z czujników wizyjnych mogą być przetwarzane w celu uzyskania informacji na temat lokalizacji i przebiegu jezdni, innych uczestników ruchu, oznakowania czy potencjalnych zagrożeń. 
Opracowanie współcześnie stosowanych technik znaleźć można w pracach \cite{Bengler2014,Velez2017}.

Inny zbiór technik wykorzystywany jest w celu detekcji i rozpoznawania twarzy oraz badania emocji.
Zagadnienie to znajduje zastosowanie w ramach projektowania nie tylko systemów przemysłowych, ale jest również powszechnie stosowane w oprogramowaniu współcześnie dostępnych aparatów cyfrowych czy w ramach serwisów społecznościowych. 
Metody te mogą również pozwolić na budowę systemów weryfikacji użytkownika bez konieczności zdefiniowania hasła dostępu. 
Znajdują także zastosowanie w interfejsach przystosowanych do pracy z osobami niepełnosprawnymi.
Analizę wykorzystywanych w tym celu algorytmów znaleźć można w pracy \cite{Anil2016}.

Współcześnie, coraz większe znaczenie mają również systemy śledzenia osób i analizy ich zachowań w celu wykrycia działań niepożądanych.
Motywując to zwiększeniem bezpieczeństwa, badane są takie zagadnienia jak detekcja porzuconych bagaży, obecność osób nieuprawnionych w ustalonych strefach czy śledzenie ruchu i re-identyfikacja przy użyciu wielu kamer.
Potrzeba automatyzacji wynika ze złożoności projektowanych systemów, które zasięgiem obejmować mogą całe aglomeracje i pozwalać na obserwację zachowań tysięcy osób. 
Z tego powodu, praktycznie niemożliwe jest zapewnienie odpowiedniej liczby operatorów -- tj. takiej, która w~pełni pozwoliłaby na wykorzystanie i analizę pozyskanych informacji w czasie rzeczywistym. 
Omawiane systemy mogą działać niezależnie lub stanowić jeden z elementów zintegrowanego oprogramowania, wykorzystującego dane z wielu źródeł \cite{Sriram2016,Hussain2016,Gouo2015}.


Równolegle do rozwoju algorytmów wizyjnych, badane są techniki implementacji pozwalające na wykorzystanie ich w systemach uruchamianych na układach elektronicznych różnego typu. 
Algorytmy wizyjne projektowane są z myślą o uruchamianiu na powszechnie stosowanych układach procesorowych w architekturach rodziny x86 lub ARM, mikrokontrolerach, układach ASIC (\emph{ang.} Application Specific Integrated Circuit) i FPGA (\emph{ang.} Field-Programmable Gate Array).

Pośród wymienionych platform wyróżnić można rodzinę Zynq \cite{zybo-reference-manual}, integrującą możliwości układów FPGA oraz procesorów ARM. 
Dzięki zastosowaniu logiki programowalnej, możliwe jest projektowanie algorytmów wizyjnych wykonywanych w sposób strumieniowy, zapewniając wysoką wydajność przy stosunkowo niskim zapotrzebowaniu na energię.
Uzupełnieniem takiego układu jest procesor ARM, umożliwiający wykorzystanie algorytmów, które wymagają swobodnego dostępu do kontekstu obliczeniowego. 
Procesor sekwencyjny jest również, w porównaniu do układów logicznych, lepiej przystosowany do wykonywania algorytmów zdominowanych przez instrukcje lub takich, których sprzętowa implementacja jest trudna lub niemożliwa.

Układy Zynq pozwalają wykorzystać zalety algorytmów projektowanych z myślą o implementacji przy użyciu języków HDL (\emph{ang.} Hardware Description Language) oraz powszechnie stosowanych języków proceduralnych. 
W szczególności umożliwiają także na uruchomienie systemu operacyjnego, ze szczególnym uwzględnieniem systemu PetaLinux \cite{petalinux-tools}, dzięki czemu możliwy jest dostęp do szerokiego zbioru narzędzi związanych z powszechnie stosowanymi systemami operacyjnymi.

\section{Cel pracy}

Celem niniejszej pracy było uruchomienie oraz skonfigurowanie systemu PetaLinux na platformie Zynq, ze szczególnym uwzględnieniem funkcjonalności, które mogą zostać wykorzystane we wbudowanych systemach wizyjnych.
W pierwszym etapie przeprowadzono analizę architektury układu oraz dostępnych systemów operacyjnych i systemów czasu rzeczywistego. 
Następnie, opracowano zagadnienia teoretyczne i praktyczne związane z funkcjonalnościami systemu, które mogą znaleźć zastosowanie we wbudowanych systemach wizyjnych. 
Ostatecznie, działanie komponentów zaprezentowano na przykładzie wybranego systemu wizyjnego.

\section{Zawartość pracy}

Praca podzielona została na pięć rozdziałów.

Rozdział \ref{cha:platform}. zawiera opis i analizę platformy Zynq-7000. Omówiono krótko specyfikację układu. Poruszono zagadnienia związane z dostępnymi systemami operacyjnymi, z uwzględnieniem zalet i wad każdego z proponowanych rozwiązań. Opisano również możliwość wykorzystania systemów czasu rzeczywistego.

Rozdział \ref{cha:functionalities}. zawiera analizę funkcjonalności układu, które mogą zostać wykorzystane w systemach wizyjnych. Zbadano możliwości wykorzystania systemu operacyjnego PetaLinux i jego integracji z układem reprogramowalnym. Opisano również protokół AXI, ze szczególnym uwzględnieniem modułów AXI DMA (\emph{ang.} Direct Memory Access) oraz VDMA (\emph{ang.} Video DMA).

W rozdziale \ref{cha:project}. zaprezentowano system wizyjny wykorzystujący omawiane funkcjonalności, którego zadaniem jest generacja tła i~segmentacja obiektów pierwszoplanowych. Zaproponowano metody integracji rozwiązań implementowanych w obu częściach układu, wskazano ograniczenia i potencjalne kierunki rozwoju.

Rozdział \ref{cha:vivado-conf}. zawiera zbiór instrukcji związanych z konfiguracją funkcjonalności omawianych w poprzednich rozdziałach, na przykładzie platformy uruchomieniowej ZYBO. Zaprezentowano w nim kroki wymagane do poprawnej konfiguracji wykorzystywanych elementów systemu oraz wskazano metody umożliwiające weryfikację poprawności działania.

Pracę kończy rozdział zawierający krótkie podsumowanie wykonywanych zadań oraz wskazanie dalszych kierunków rozwoju.


\chapter{Platforma Zynq-7000}
\label{cha:platform}

Karta uruchomieniowa ZYBO jest przedstawicielem rodziny układów SoC (\emph{ang}. System-on-a-chip) Zynq-7000. %TODO ZYBO to jest karta uruchomieniowa/ewaluacyjna, układ Zynq. Trzeba to rozróżnić.
% OK
SoC to układy scalone integrujące zbiór układów elektronicznych, takich jak mikroprocesor, układy koprocesujące, interfejsy wejścia i wyjścia, czy pamięci. Są one powszechnie wykorzystywane do projektowania systemów wbudowanych. %TODO wszystkie elementy budowy ??? inaczej to trzeba napisać...
% OK?
Centralną część układu rodziny Zynq-7000 stanowi dwurdzeniowy procesor o architekturze ARM w wersji Cortex-A9, współpracujący z działającym równolegle układem FPGA, opartym na architekturze Artix-7 lub Kintex-7 \cite{zynq-homepage}. %TODO proszę zwrócić uwagę, że \cite powinno być przed kropką. Nie wspomniał Pan o logice programowalnej. 
% OK, poprawię masowo po 1. korekcie
Są to układy heterogeniczne, łączące w sobie elementy klasycznego układu FPGA (\emph{PL}, \emph{ang.} Programmable Logic) oraz procesora ARM (\emph{PS}, \emph{ang.} Processing System). %TODO to zdanie jest powt. wcześniejszego - jakoś trzeba to zintegrować.
% OK

Karta ZYBO wyposażona jest ponadto w 512 MB pamięci RAM, złącza HDMI i VGA do transmisji obrazu, gniazda Jack do przesyłu sygnału dźwiękowego, gniazdo USB oraz slot pamięci MicroSD. %TODO Karta ZYBO, dzwięku -> sygnału dźwiękowego
% OK
Komunikacja sieciowa jest możliwa dzięki implementacji stosu TCP/IP i obecności gniazda RJ-45 \cite{zynq-datasheet}.

Układy rodziny Zynq-7000 są stosowane w aplikacjach systemów wsparcia kierowców, systemach wizyjnych wysokich rozdzielczości, cyfrowego przetwarzania sygnałów czy kryptograficznych. Zaproponowano wykorzystanie zalet elementów FPGA i CPU do projektowania systemów wsparcia kierowców ADAS, co pozwoliło na redukcję czasu odpowiedzi systemu i zapotrzebowania na energię \cite{GuanwenZhong}.
W innej pracy zbadano możliwość wykorzystania układu do transmisji sygnału o rozdzielczości 4K ($3840 \times 2160$ pikseli) przy możliwie niewielkim zużyciu zasobów i energii \cite{MaleenAbeydeera}.
Wśród zagadnień kryptograficznych, realizowanych przy użyciu omawianej rodziny układu wymienić można algorytmy generowania liczb pseudolosowych o wysokiej rozdzielczości	\cite{PawelDabal2014}. %TODO (po kropce) poza tym moze o jedno zdadnie wiecej na temat kazdego artykuły i inny układ (nie na końcu, a po zdaniu cite)
% OK

Na rysunku \ref{fig:zynq-overview} przedstawiono schemat omawianej architektury.

\begin{figure}[!htb]
	\centering
	\includegraphics[width=12cm]{img/zyng-platform.png}
	\caption{Schemat architektury Zynq-7000. (Źródło: \cite{zybo-reference-manual})}
	\label{fig:zynq-overview}
\end{figure}
%TODO może nieco większy
% OK

Schemat przedstawia podział układu na części \emph{PL} -- oznaczonej kolorem żółtym, oraz \emph{PS} -- na zielono.
Architektura części programowalnej zbliżona jest do powszechnie stosowanych układów FPGA. %TODO konkretnie jaka seria Xilinx
% dodałem tę informację nieco wyżej
Wyposażono ją jednak w zbiór portów umożliwiających wydajną komunikację z procesorem. 
Ponadto, konfiguracja tej części wykonywana jest na starcie przez procesor lub przy użyciu interfejsu JTAG i układ nie zawiera elementów pozwalających na wykorzystanie logiki programowalnej niezależnie.
Część procesorowa wyposażona jest w szereg interfejsów, w tym kontroler pamięci DDR3, interfejs komunikacji AMBA oraz zbiór interfejsów peryferyjnych.

Procesor wyposażony jest w koprocesor arytmetyczny (\emph{FPU}), wspierający w obliczeniach na liczbach zmiennoprzecinkowych oraz wspiera obsługę architektury SIMD (\emph{ang.} Single Instruction, Multiple Data) -- pozwalającej na przetwarzanie wielu strumieni danych przy użyciu jednego strumienia instrukcji. 
Zagadnienia te szerzej opisane zostały w rozdziale \ref{sec:arm-neon}.
%TODO Tu dobrze by było się odwoływać do nazw bloków z rysunku (np. FPU).
% OK

Układ wyposażony jest w kontroler pamięci DDR, obsługujący żądania dostępu ze strony zarówno procesora, jak i logiki programowalnej. %TODO logiki programowalnej (bez powt. układ)
% OK
Pamięć jest współdzielona między obiema częściami. %TODO obiema częściami
% OK
Pozwala to na wymianę danych, do czego wykorzystywany jest standard AXI. Zastosowanie znajduje również mechanizm DMA, pozwalający na przeprowadzanie operacji z użyciem pamięci bez udziału procesora. 
Interfejs pozwala na transmisję pojedynczych słów danych, umożliwiając konfigurację parametrów pracy modułów algorytmicznych, jak i na transmisję o wysokiej przepustowości. 
Pozwala to, dzięki zastosowaniu modułu VDMA, na przesyłanie obrazu o rozdzielczości HD z częstotliwością osiągającą wartości 680 klatek na sekundę \cite{axi-vdma-guide}. 
Interfejs AXI opisano szerzej w rozdziale \ref{sec:axi-std}.

Dzięki zastosowaniu kontrolera przerwań (\emph{General Interrupt Controller}) możliwe jest wykorzystanie modułów logiki programowalnej komunikujących się z procesorem z wykorzystaniem techniki zgłaszania żądań. Zagadnienie to opisano w rozdziale \ref{sec:axi-interrupts}.
Ponadto, dostępne są powszechnie spotykane układy zegarowe (\emph{Timers}) i \emph{Watchdog}, odpowiedzialny za przerwanie pracy procesora w przypadku wykrycia błędnego wykonania programu.
Kontekst pamięciowy synchronizowany jest pomiędzy oboma rdzeniami procesora dzięki modułowi \emph{Snoop Control Unit}.

Układ logiki programowalnej należeć może do rodzin Artix-7 lub Kintex-7 i zbudowany jest z typowych elementów wykorzystywanych w FPGA o różnej liczebności, związanej z klasą układu:
\begin{itemize}
	\item CLB (\emph{ang.} Configurable Logic Blocks) -- w tym tablice Look-up (\emph{LUT}) -- $14400 - 277400$ elementów, przerzutniki (\emph{FF}) -- $28800 - 554800$ elementów.
	
	\item Pamięci \emph{Block RAM} -- od $1,8Mb$ do $26,5Mb$ ($50 - 755$ elementów).
	
	\item Elementy \emph{DSP} -- wykorzystywane zwykle do implementacji operacji dodawania i mnożenia -- $66 - 2020$ elementów.
	
	\item Bloki \emph{IOB} -- umożliwiające budowę interfejsów wejściowych i wyjściowych.
	
	\item Inne -- w tym interfejs JTAG, PCI Express czy konwertery analogowo-cyfrowe.
\end{itemize}
%TODO Brakuje choćby podstawowego opisu zasobów PL - dodać.
% OK dodałem opis
%TODO Proszę dodać opisy pozostałych elementów PS (choć 1-2 zdania na każdy bloczek)
% OK, dodałem te, które mogą być jakkolwiek istotne z mojej perspektywy.
%TODO Proszę napisać, dlaczego AXI i FPU opisane są później... 
% nie chciałem rozdzielać tych zagadnień na część teoretyczną i praktyczną. Wg mnie nie miałoby to sensu zwłaszcza dla FPU.

%TODO TO bym jakoś bardziej w kierunku rozdziału o systemach operacyjnych przekształcił (bo de facto o tym jest). Tylko wtedy Peta i inne to subsection (przeorganizować hierarchię)
% OK, poprawiłem
\section{Zastosowanie systemu operacyjnego}
\label{sec:arm-programming}

Centralny element architektury stanowi dwurdzeniowy procesor ARM. 
Jest on odpowiedzialny za przeprowadzenie konfiguracji logiki reprogramowalnej. 
Ponadto pozwala na wykonanie dowolnego programu użytkownika. 
Powszechnie stosowana jest konfiguracja \textit{bare-metal}, w której procesor wykonuje program zaprojektowany w pełni przez użytkownika. 
Pozwala to na uzyskanie możliwie największej kontroli nad pracą układu, ogranicza jednak możliwości wykorzystania pełni zasobów procesora oraz utrudnia projektowanie rozbudowanych aplikacji. Brak systemu operacyjnego ogranicza możliwość wykorzystania komunikacji sieciowej na etapie wykonania programu. Możliwości przechowywania wyników i logów aplikacji są niewielkie. Ponadto, użycie zewnętrznych bibliotek, w tym związanych z przetwarzaniem obrazów, takich jak OpenCV, jest niemożliwa. W efekcie, wykorzystanie konfiguracji pozbawionej systemu operacyjnego nie jest możliwe w przypadku aplikacji wymagających nadzoru bez fizycznego dostępu do układu czy przechowywania wyników.
%TODO może nieco więcej szczegółów i przede wszstykim, że brak OS.
% OK

W niniejszej pracy badano możliwość wykorzystania systemu operacyjnego Linux na przykładzie aplikacji systemów wizyjnych. 
Dzięki zastosowaniu Linuxa, możliwe staje się budowanie programów składających się z wielu modułów działających niezależnie. 
System ten wspiera obsługę sieci, co pozwala na wykorzystanie narzędzi komunikacji sieciowej, jak SSH \cite{ssh-protocol}, do konfiguracji i nadzorowania działania aplikacji, co opisano w rozdziale \ref{sec:ssh}. 
Ponadto, możliwe jest wykorzystanie powszechnie dostępnych bibliotek, ułatwiających rozwój aplikacji w krótkim czasie. 
Zagadnienie to badano na przykładzie biblioteki OpenCV \cite{opencv-library}, udostępniającej narzędzia przetwarzania obrazów, co opisano w rozdziale \ref{sec:opencv-lib}.

Zbadano możliwość wykorzystania systemu PetaLinux, rozwijanego przez organizację Xilinx, jak i podstawowej wersji systemu, opartej wyłącznie na źródle jądra, oraz dystrybucji bazującej na Ubuntu Core. %TODO co jest rozumiane przed podstawową wersję systemu ?
% OK?
Ponadto, rozpatrzono możliwość użycia systemu czasu rzeczywistego do wykonania zadań obliczeniowych z zachowaniem reżimu czasowego.
%TODO Wspomnieć o RTOS
% OK

\subsection{PetaLinux} %TODO Sub
% OK

Firma Xilinx zapewnia dostęp do zbioru narzędzi \emph{PetaLinux Tools} \cite{petalinux-tools} umożliwiających przeprowadzenie procesu konfiguracji, budowania i uruchamiania systemu Linux na platformie Zynq. 
Dzięki zintegrowaniu koniecznych narzędzi w jednym pakiecie, proces ten jest w dużej części zautomatyzowany i ogranicza interakcję z programistą, zapewniając przy tym możliwość dowolnej konfiguracji systemu.
Celem wykorzystania omawianego pakietu narzędzi jest zbudowanie systemu operacyjnego gotowego do uruchomienia i umożliwiającego szybką rekonfirugację zarówno elementów logiki reprogramowalnej jak i samego systemu operacyjnego.

Pakiet wymaga dostarczenia zewnętrznych zależności, w tym narzędzi umożliwiających budowanie systemu -- kompilatora, generatora parserów, systemu budowania -- oraz zbioru narzędzi programistycznych i konfiguracyjnych.
W przypadku dystrybucji Debian, zależności mogą być zainstalowane poleceniem:

\begin{lstlisting}[breaklines=true]
apt-get install tofrodos iproute2 gawk gcc git make net-tools libncurses5-dev tftpd zlib1g-dev libssl-dev flex bison libselinux1
\end{lstlisting}

W przypadku pracy na systemie wspierającym architekturę 64-bitową, konieczne jest również zainstalowanie bibliotek programistycznych dla architektury 32-bitowej.

\begin{lstlisting}[breaklines=true]
dpkg --add-architecture i386
apt-get update
apt-get install libc6:i386 libncurses5:i386 libstdc++6:i386
apt-get install libgtk2.0-0:i386 libxtst6:i386 gtk2-engines-murrine:i386 lib32stdc++6 libxt6:i386 libdbus-glib-1-2:i386 libasound2:i386
\end{lstlisting}

Praca z pakietem wymaga ponadto wykorzystania oprogramowania \emph{Vivado Design Suite} \cite{vivado-home} do zaprojektowania układu połączeń logiki reprogramowalnej oraz \emph{Xilinx SDK} \cite{xsdk-home} do kompilacji programów uruchamianych w środowisku systemu operacyjnego układu.
Pierwszym krokiem jest wykonanie projektu w pakiecie \emph{Vivado}. 
Szczegóły procesu opisano w rozdziale \ref{sec:vivado-conf}. %TODO sekcji -> rozdziale (wydaje mi się, ze w PL słowo sekcja nie powinno być uzywane w tym kontekście - kalka z ENG)
% OK, poprawię masowo w całej pracy na końcu 1. korekty
Wyeksportowany projekt Vivado jest konieczny do przeprowadzenia procesu budowania systemu, proces konfiguracji projektu PetaLinux opisano w rozdziale \ref{sec:petalinux-config}. %TODO akapity jednozdaniowe są zakazane...
% OK, spróbuję je wyłapać

Pakiet udostępnia możliwość dodania do budowanego systemu zbioru programów i bibliotek. 
Dostępne jest kilkaset pakietów oprogramowania oferowanych na zasadach wolnych licencji, w tym biblioteki do przetwarzania obrazów. %TODO Dostępne, dostępnego
% OK
Ponadto, pakiet umożliwia dodanie własnoręcznie zbudowanych aplikacji. 
Pozwala to na integrację etapu projektowania aplikacji oraz budowy i uruchamiania systemu operacyjnego w jednym procesie.

Projekt PetaLinux jest niezależny od projektu Vivado i może powstawać równolegle. 
Zmiany w strukturze modułów logiki reprogramowalnej wymagają ponownego zbudowania plików wynikowych systemu operacyjnego PetaLinux, jednak proces ten został wydzielony z oprogramowania Vivado. %TODO co tu jest rozumiane przez system
% OK
Pozwala to na wykorzystanie jednego projektu opisującego logikę współpracującego z aplikacjami \textit{bare-metal} i systemowymi. 

Proces budowania systemu jest czasochłonny, na etapie prototypowania aplikacji praktyczne jest zastosowanie oprogramowania pracującego w trybie \textit{bare-metal}. %TODO bare-metal - pisałbym w \textit{} 
% OK, poprawię masowo na koniec
Pozwala to na przeprowadzanie procesu debugowania aplikacji bezpośrednio z poziomu oprogramowania Vivado/SDK. 
Po upewnieniu się, że sprzętowa część algorytmu działa poprawnie, zaprojektować można aplikację systemową, odpowiedzialną za komunikację, monitorowanie i wykorzystanie wyników działania algorytmu w kompletnym projekcie.

\subsection{Inne dystrybucje systemu}
Niezależnie do analizy zastosowania PetaLinux, zbadano również inne możliwości konfiguracji systemu operacyjnego do zastosowania na platformie ZYBO. 
Wśród dostępnych opcji, rozpatrzono dystrybucję Ubuntu Core oraz budowę systemu Linux ze źródeł.
%TODO ponieważ to jest część teoretyczna, to proszę nie wprowadzać takie narracji po zapozaniu...
% OK?

\subsubsection{Budowa ze źródeł}
Wykorzystanie pakietu PetaLinux związane jest z ograniczeniem dostępności projektu do środowisk, dla których ten pakiet narzędzi jest dostępny. 
W przypadku konieczności uruchomienia projektu na systemie nie wspieranym przez twórców oprogramowania lub potrzeby wprowadzenia dużych zmian w kodzie źródłowym systemu i konfiguracji, konieczne może być przeprowadzenie pełnego procesu budowania samodzielnie. 
Takie podejście pozwala również na pełne zrozumienie znaczenia kolejnych kroków procesu konfiguracji.

Proces budowy systemu składa się z kilku kroków. 
\begin{enumerate}
\item Plik binarny zawierający konfigurację części oprogramowania wykorzystującej logikę programowalną dołączany jest w trakcie procesu budowania systemu operacyjnego.%TODO co tu jest rozumiane przez zaprojektowanie połączeń logiki reprogramowalnej
% OK, zmieniłem to zdanie

\item Konieczne jest zbudowanie dwóch programów rozruchowych (\emph{ang.} bootloader). %TODO program rozruchowy to poprawne określenie PL ?
% wikipedia mówi, że tak, dodałem angielskie znaczenie
Pierwszy z nich -- FSBL (\emph{ang.} First Stage Boot Loader) -- odpowiada za przeprowadzenie procesu wstępnej konfiguracji procesora, kontrolera pamięci i uruchomienie drugiego programu rozruchowego. 
Na drugim etapie rozruchu wykorzystywany jest program U-Boot. 
Jego zadaniem jest przygotowanie środowiska do uruchomienia właściwego systemu operacyjnego.

\item Kolejny krok wymaga zbudowania struktury drzewa urządzeń (\emph{ang.} device tree). 
Pozwala ona na zdefiniowanie i konfigurację urządzeń połączonych z procesorem, dzięki czemu mogą być one obsłużone przez system operacyjny. 
W przypadku układu Zynq, wykorzystanie tej struktury pozwala na konfigurację i komunikację z elementami układu FPGA. %TODO ZYBO czy Zynq...
% Poprawiłem

\item Po przeprowadzeniu wstępnej konfiguracji elementów systemu, możliwe jest wykonanie procesu konfiguracji, budowania i przygotowania wynikowych plików binarnych.
\end{enumerate}
Opisany proces jest skomplikowany i wymaga specjalistycznej wiedzy. 
Dostępne są obszerne opracowania dotyczące tematu, zawierające precyzyjny opis kolejnych wymaganych kroków \cite{zybo-zynq-getting-started,zybo-stock-linux,xilinx-build-kernel}.

%TODO A Pan to robił ? Jeśli tak to link to szczegółowego opisu.
% robiłem, ale na bazie źródeł, które cytuję powyżej. Proces jest dość skomplikowany i podatny na zmiany, ze względu na liczbę narzędzi, które się wykorzystuje, więc nie widziałem sensu budowania kolejnego tutoriala.

\subsubsection{Ubuntu Core}

Ubuntu Core to dystrybucja systemu Linux dedykowana do zastosowań w urządzeniach tzw. internetu rzeczy (IoT -- \emph{ang.} Internet of Things). %TODO (IoT ang.....) nie wiem czy nie lepiej koncepcji -> tzw.
% OK?
Dystrybucja ta oparta jest na podstawowej wersji systemu Ubuntu, przystosowana do uruchomienia na urządzeniach o ograniczonej mocy obliczeniowej.

Dzięki wykorzystaniu Ubuntu Core, możliwy jest dostęp do repozytorium oprogramowania udostępnianego przez dystrybucję. 
W przeciwieństwie do dystrybucji PetaLinux, instalowane oprogramowanie może być aktualizowane w trakcie pracy systemu. 
Cecha ta może być istotna w przypadku aplikacji działających przez długi czas, gdy aktualizacja oprogramowania jest korzystna ze względu na znalezione błędy lub poprawę bezpieczeństwa w kolejnej wersji. %TODO bezpieczeństwo.
% ok?
System Ubuntu Core może być zbudowany i uruchomiony na karcie ZYBO przy użyciu dedykowanego narzędzia \cite{ubuntu-core-zybo}.
%TODO Uwaga tak jak wcześniej...
% OK

\subsection{RTOS}

System operacyjny czasu rzeczywistego (\emph{RTOS}, \emph{ang.} Real Time Operating System) to system operacyjny, którego zadaniem jest obsługa aplikacji przy zachowaniu założeń o nieprzekroczeniu maksymalnego dopuszczalnego czasu odpowiedzi programu. 
Pozwala to na projektowanie aplikacji, w których czas odpowiedzi ma kluczowe znaczenie, w tym systemów sterowania lub krytycznych aplikacji wizyjnych.
Dzięki zastosowaniu dwurdzeniowego procesora w układzie Zynq, rozważyć można zaprojektowanie rozwiązania, w których jeden z rdzeni odpowiada za wykonanie programu Linuxa, a drugi -- aplikacji lub systemu czasu rzeczywistego. %TODO ZYBO/Zynq - aplikacji \textit{bare metal}
% OK

Rozpatrzono możliwość uruchomienia systemu operacyjnego PetaLinux i jego współpracę z aplikacjami czasu rzeczywistego na przykładzie OpenAMP \cite{openamp-home}. 
OpenAMP zapewnia interfejs umożliwiający komunikację pomiędzy programami działającymi w systemie Linux oraz aplikacjami czasu rzeczywistego, wykorzystując do tego narzędzia dostępne już w systemie.
Z punktu widzenia klasycznego systemu operacyjnego, program działający na systemie czasu rzeczywistego jest zewnętrznym zasobem, który oczekuje na zlecenie wykonania konkretnego zadania i wysyła odpowiedź.
Dzięki wykorzystaniu systemu czasu rzeczywistego FreeRTOS \cite{freeRTOS-home}, aplikacje mogą mieć dostęp do zasobów systemowych, w tym pamięci i interfejsów komunikacji.

%TODO Za krótkie akapity - proszę zobaczyć jak to wygląda w txt
% Poprawiłem

System czasu rzeczywistego może być wykorzystany do obliczeń o krytycznym znaczeniu. 
W przypadku wykorzystania klasycznego systemu operacyjnego, nie jest możliwe zagwarantowanie wykonania dowolnego zadania w określonym czasie. 
W trakcie działania aplikacji, system może zadecydować o jej czasowym zatrzymaniu i udostępnieniu zasobów innemu z oczekujących zadań. 
Aplikacja działająca w czasie rzeczywistym pozwala uniknąć tego zjawiska.

System PetaLinux oferuje dostęp do modułów \texttt{RPMsg}, \texttt{remoteproc}, \texttt{virtIO}, które są wymagane do zapewnienia komunikacji z systemem czasu rzeczywistego. 
Udostępnione zostały również aplikacje testowe, które pozwalają na sprawdzenie poprawności działania konfiguracji.
Użycie systemu czasu rzeczywistego wymaga zmian projektowych, w tym konfiguracji dwóch instancji konsoli do komunikacji szeregowej i zdefiniowania struktury drzewa urządzeń określającej obszar pamięci dla obu systemów operacyjnych. 
Po zbudowaniu poprawnie skonfigurowanego systemu i jego uruchomieniu, przetestowanie działania aplikacji wymaga użycia poniższych poleceń.

\begin{lstlisting}[breaklines=true]
modprobe zynq_remoteproc firmware=image_echo_test
modprobe rpmsg_user_dev_driver
echo_test
\end{lstlisting}

W rezultacie uruchomiono moduły odpowiedzialne za obsługę systemu czasu rzeczywistego i przeprowadzono test komunikacji. %TODO W rezultacie uruchomione...
% OK
Konfiguracja i wykorzystanie systemów czasu rzeczywistego wykracza poza zakres niniejszej pracy, a zagadnienie jest obiektem obszernych opracowań \cite{adam-taylor-openamp,zynq-openamp-gsg}.

%TODO to źle kojarzę, że coś Pan z tym eksperymentował ?
% sprawdzałem to, ale moje wnioski właściwie przeczyły temu, co znalazłem w innych źródłach. Tzn. ja to w jakimś stopniu "zhackowałem", Xilinx upiera się, że wymagane jest dostarczenie BSP dedykowanej dwóm systemom, moje próby zbudowania BSP spełzły na niczym, ale gdy uruchomiłem `echo_test` na domyślnej konfiguracji, to jeden rdzeń procesora "zniknął" z Peta i zaczął działać real-time. Ten temat był dopiero rozwijany przez Xilinx gdy nad tym pracowałem, spodziewałbym się, że uporządkują to w przyszłości, bo mocno forsują swoje rozwiązanie - więc nie chciałem opisywać tutaj żadnych niepełnych rozwiązań...

\subsection*{Podsumowanie} %TODO dlaczego to nie ma tytułu. Może to przenieć na koniec jako podsumowanie OS, tylko dodać coś o RTOS.
% OK

Zarówno wykorzystanie pakietu PetaLinux Tools, jak i obu pozostałych metod pozwala na zbudowanie w pełni funkcjonalnej dystrybucji systemu Linux i uruchomienie jej na układzie Zynq. %TODO karcie/Zynq
%OK
Każda z metod wiąże się z innymi ograniczeniami i udostępnia inne możliwości. 
W przypadku użycia narzędzi PetaLinux, użytkownik uzyskuje dostęp do ograniczonego zbioru dodatkowych aplikacji, niewielkiej w porównaniu do repozytoriów udostępnianych w dystrybucji Ubuntu Core. 
Ponadto aktualizacja oprogramowania może wymagać ponownego zbudowania systemu lub nie być możliwa bez aktualizacji pakietu narzędzi. 

Dystrybucja Ubuntu Core zapewnia dostęp do aktualizacji samego systemu, pozwalając na zachowanie bezpieczeństwa działania i dostęp do poprawek kodu oprogramowania. 
Może to być kluczowe w przypadku wykorzystania układu Zynq do działania przez długi czas z oczekiwaną niezawodnością. %TODO ZYBO/Zynq
% OK

W przypadku konieczności dostosowania kodu systemu operacyjnego do własnych potrzeb, praktyczne staje się natomiast wykorzystanie technik budowy systemu bezpośrednio ze źródeł. 
Ogranicza to jednak możliwości instalacji dodatkowego oprogramowania i wymaga dobrej znajomości zagadnień związanych z działaniem systemu Linux.

Pakiet PetaLinux pozwala jednak na największą integrację z oprogramowaniem Vivado, co ułatwia prototypowanie aplikacji. 
Dzięki udostępnieniu repozytorium oprogramowania oraz braku konieczności ingerencji użytkownika w proces budowania systemu, wykorzystanie go jest najlepszym rozwiązaniem w większości projektów. 
Z tego powodu, w niniejszej pracy zdecydowano się na użycie tego rozwiązania na dalszym etapie projektu.

Zastosowanie systemu czasu rzeczywistego współpracującego z innym systemem operacyjnym pozwala na wykonanie krytycznych sekcji kodu z zachowaniem ograniczeń czasowych. Pamiętać należy jednak, że wiąże się to z ograniczeniem maksymalnej wydajności operacji wykonywanych przez klasyczny system operacyjny.

Firma Xilinx zrezygnowała ze wsparcia dla systemu FreeRTOS i podobnych i zdecydowano się na oparcie na bibliotece OpenAMP do realizacji zadań wykonywanych w czasie rzeczywistym. W okresie powstawania pracy, literatura omawiająca integrację biblioteki z systemem operacyjnym dla kart innych producentów nie była powszechnie dostępna. Z tego powodu, realizacja omawianych zadań w przypadku karty ZYBO była poważnie utrudniona.

\chapter{Badane funkcje układu}

\section{Obsługa SSH}
\label{sec:ssh}
Po połączeniu układu z komputerem przez interfejs USB, możliwe jest otworzenie konsoli komunikacji przy użyciu protokołu transmisji szeregowej. Komunikacja odbywa się z prędkością 115200 bodów, ośmioma bitami danych, jednym bitem stopu i bez bitu parzystości.

Komunikacja przy użyciu transmisji szeregowej jest jednak ograniczona do sytuacji, w których możliwy jest bezpośredni dostęp do układu. Ponadto, nie zapewnia wysokiej przepustowości transmisji czy możliwości przesyłu plików. Z tych powodów, korzystne staje się wykorzystanie protokołu \emph{SSH} do nawiązania komunikacji sieciowej.
Omawiany protokół wspierany jest przez zdecydowaną większość dystrybucji systemu Linux i nie wymaga dodatkowej konfiguracji na etapie budowania systemu. Połączenie odbywa się przy użyciu poniższego polecenia.

\begin{lstlisting}[breaklines=true]
ssh root@(*@\textit{}adres-ip-urządzenia}@*)
\end{lstlisting}

Domyślne hasło administratora w przypadku dystrybucji \emph{PetaLinux} to \texttt{root}. Może być ono zmienione na etapie konfiguracji systemu.


\begin{lstlisting}[breaklines=true]
petalinux-config -c rootfs
Petalinux RootFS Settings ---> Root password
\end{lstlisting}

Aby uprościć proces logowania, wykorzystać można mechanizm wymiany kluczy, zapewniany przez protokół.

\begin{lstlisting}[breaklines=true]
ssh-copy-id -i ~/.ssh/id_rsa.pub root@(*@\textit{adres-ip-urządzenia}@*)
\end{lstlisting}

Umożliwia to logowanie bez konieczności podania hasła użytkownika. Skonfigurowany w opisany sposób, protokół daje dostęp do pełnego zbioru narzędzi, w tym zdalnej obsługi konsoli użytkownika, przesyłania plików, tunelowania portów czy zdalnego montowania systemów plików.
\section{FPU i technologia NEON}
\label{sec:arm-neon}
Układ \emph{ZYBO} wyposażony jest w koprocesor arytmetyczny oraz wspiera polecenia wykorzystujące technologię NEON. \cite{neon-home} Elementy te pozwalają na zwiększenie wydajności projektowanych aplikacji w przypadku, gdy wykonywane operacje wymagają przeprowadzania obliczeń na liczbach zmiennoprzecinkowych lub działań wektorowych.

Koprocesor arytmetyczny, FPU (ang. \emph{floating-point unit}), to układ działający we współpracy z jednostką procesora, dedykowany do wykonywania obliczeń na liczbach zmiennoprzecinkowych. Wykorzystanie dedykowanego układu pozwala na zwiększenie szybkości wykonywania operacji arytmetycznych, pierwiastkowania i przesunięć bitowych. W przypadku braku dedykowanego układu FPU, konieczne jest symulowanie jego działania przez wykonywanie większej liczby operacji na liczbach całkowitych, co wiąże się ze spadkiem wydajności.

Technologia NEON pozwala na rozszerzenie puli rozkazów procesora ARM o polecenia wykorzystujące architekturę SIMD zdefiniowaną przez taksonomię Flynna.\cite{Flynn1972}

SIMD (ang. \emph{Single Instruction streams, Multiple Data streams}) to klasa systemów, które pozwalają na przetwarzanie wielu strumieni danych na podstawie jednego strumienia instrukcji. Zastosowania tej architektury obejmują zagadnienia, w których dla wielu wartości wejściowych konieczne jest wykonanie tej samej operacji. Cechę tę posiada wiele operacji związanych z przetwarzaniem sygnałów i obrazów, w tym  wyznaczanie wartości szybkiej transformaty Fouriera, implementacje filtrów FIR i IIR czy operacje skalowania, rotacji i filtracji uśredniającej obrazu.

Rozpatrzono możliwość wykorzystania architektury NEON w zagadnieniach przetwarzania sygnałów. Działanie testowano na podstawie programu wyznaczającego wartość iloczynu skalarnego dwóch wektorów zadanej długości. Porównano trzy implementacje algorytmu, którego kod źródłowy zawarto w dodatku \ref{cha:neon-source}.
Wykorzystano implementację bazową oraz z wykorzystaniem poleceń dostępnych w architekturze NEON i porównano wyniki z implementacją zaprojektowaną w asemblerze.

Implementacja w architekturze NEON wykorzystuje dedykowane funkcje, udostępnione w bibliotece \texttt{arm\_neon.h}, które mają na celu maksymalne zwiększenie wydajności aplikacji. W przypadku pozostałych implementacji, wykorzystywane są polecenia wykonywane na koprocesorze VFP (ang. \emph{Vector Floating-Point}). VFP pozwala na wykonanie tej jednej instrukcji dla wektora danych wejściowych. Układ ten nie należy do rodziny SIMD i wykonuje instrukcje sekwencyjnie, w przeciwieństwie do architektury NEON.


Wyniki testów wydajności zebrano w tabeli \ref{tab:neon-time-results}.

\begin{table}[h]
	\caption{Wyniki testu wydajnościowego.}
	\centering
	\label{tab:neon-time-results}
	\begin{tabular}{|l|l|l|l|}
		\hline
		\multicolumn{4}{|c|}{Bez optymalizacji} \\ \hline
		Implementacja & min {[}s{]} & max {[}s{]} & średnio {[}s{]} \\ \hline
		Bazowa & 0,4266 & 0,4339 & 0,4296 \\ \hline
		NEON & 0,1103 & 0,1108 & 0,1105 \\ \hline
		ASM & 0,4082 & 0,4086 & 0,4083 \\ \hline
		\multicolumn{4}{|c|}{Z optymalizacjami} \\ \hline
		Bazowa & 0,1080 & 0,1152 & 0,1092 \\ \hline
		NEON & 0,1088 & 0,1147 & 0,1090 \\ \hline
		ASM & 0,1087 & 0,1144 & 0,1089 \\ \hline
	\end{tabular}
\end{table}

Rozpatrzono przeprowadzenie procesu komplikacji z wyłączonymi optymalizacjami kompilatora (flaga \texttt{-O0}) oraz z włączonymi wszystkimi optymalizacjami (\texttt{-O3}).

Wykorzystane poleceń NEON wymaga użycia odpowiadających im parametrów kompilacji. Poniżej przedstawiono polecenie kompilacji testowej implementacji wykorzystującej NEON.

\begin{lstlisting}[breaklines]
arm-linux-gnueabihf-gcc -Wall -O3 -mcpu=cortex-a9 -mfpu=neon -ftree-vectorize -mvectorize-with-neon-quad -mfloat-abi=hard -ffast-math -g -c -o "src/main.o" "../src/main.c"
\end{lstlisting}

W sytuacji, gdy wyłączono optymalizacje na etapie kompilacji, zauważalny jest znaczny wzrok wydajności w przypadku wykorzystania instrukcji udostępnianych przez architekturę NEON. Pozwala ona na niemal czterokrotne zwiększenie szybkości działania programu względem pozostałych implementacji. Różnica ta zanika w przypadku wykorzystania możliwości optymalizacji kodu programu na etapie kompilacji. Różnica w szybkości wykonania programu NEON jest niewielka, jednak pozostałe implementacje zostały zoptymalizowane do stanu, w którym koszt ich wykonania porównywalny jest z implementacją w NEON.

Wyniki pozwalają wnioskować o słuszności wykorzystania instrukcji udostępnianych przez architekturę NEON ze względu na możliwy wzrost wydajności. Istotna jest jednak weryfikacja wyników i potwierdzenie poprawy działania aplikacji. W przypadku, gdy różnice między programami są niewielkie, użycie instrukcji NEON może być niekorzystne ze względu na zwiększoną latencję wykonania rozkazów.

\section{Protokół AXI}
\label{sec:axi-std}
Protokół AXI (ang. \emph{Advanced eXtensible Interface}) zdefiniowany został w specyfikacji AMBA (ang. \emph{Advanced Microcontroller Bus Architecture}) 3. W kolejnej wersji dokumentu sprecyzowano standard w najnowszej wersji - AXI4. \cite{axi-spec} Protokół wykorzystywany jest do komunikacji pomiędzy elementami układu lub modułami zbudowanymi wewnątrz logiki reprogramowalnej i jest dedykowany systemom o dużej wydajności i pracującym z wysoką częstotliwością.

Specyfikacja definiuje trzy typy interfejsu:
\begin{itemize}
	\item AXI4 - wykorzystujące technikę MMIO (ang. \emph{Memory-Mapped Input/Output}) do odwzorowania rejestrów w przestrzeni adresowej pamięci RAM i dedykowanej aplikacjom wymagającym dużej wydajności komunikacji.
	\item AXI4-Lite - uproszczona wersja protokołu, wykorzystująca MMIO i dedykowana aplikacjom o mniej rozbudowanych wymaganiach komunikacyjnych.
	\item AXI4-Stream - wersja przepływowa protokołu, nie wykorzystująca technik MMIO.
\end{itemize}

Interfejsy wykorzystujące technikę MMIO stosowane są powszechnie w zadaniach konfiguracji modułów aplikacji czy przesyłania informacji, takich jak ramka sygnału wizyjnego do pamięci procesora. Dzięki reprezentacji elementów logiki reprogramowalnej w pamięci, możliwa jest jednolita analiza działania całego systemu.

Interfejs w wersji \emph{Stream} wykorzystywany jest natomiast do przesyłania sygnału pomiędzy kolejnymi elementami układu, na przykład transmisji kolejnych pikseli obrazu pomiędzy kolejnymi składowymi algorytmu przetwarzania obrazu. Proces przesyłania danych w takiej formie charakteryzuje się większą wydajnością, analiza działania aplikacji jest jednak utrudniona ze względu na brak reprezentacji przesyłanych danych w pamięci.

Możliwe jest również połączenie obu typów interfejsu wewnątrz jednego elementu. Technika ta wykorzystana została w przypadku elementu AXI VDMA, umożliwiając manipulowanie ramkami obrazu wizyjnego przesyłanymi przy użyciu interfejsu \emph{Stream} dzięki buforowaniu w pamięci RAM. Zagadnienie to szerzej opisano w sekcji \ref{sec:axi-vdma}. Podobne techniki wykorzystano również w przypadku interfejsu Ethernet DMA, umożliwiającego komunikację przy użyciu protokołu Ethernet.

\subsection{Przebieg transakcji}

Transakcja komunikacyjna odbywa się pomiędzy dwoma urządzeniami - \emph{master} i \emph{slave}, jednak dzięki zastosowaniu elementów \emph{AXI-Interconnect} możliwe jest połączenie wielu urządzeń, co przedstawiono na schemacie \ref{fig:axi-interconnect}.

\begin{figure}[h]
	\centering
	\def\svgwidth{8cm}
	\input{img/axi-interconnect.pdf_tex}
	\caption{Schemat połączenia Interconnect w protokole AXI.}
	\label{fig:axi-interconnect}
\end{figure}


Komunikacja odbywa się przy użyciu pięciu niezależnych kanałów:
\begin{itemize}
	\item Read Address
	\item Write Address
	\item Read Data
	\item Write Data
	\item Write Response
\end{itemize}

Każdy kanał zawiera zbiór sygnałów wykorzystywanych w trakcie wymiany danych.

Transmisja rozpoczyna się od wykorzystania sygnałów \emph{valid} i \emph{ready}. Urządzenie źródłowe wymusza stan wysoki sygnału \emph{valid} i oczekuje na zmianę wartości sygnału \emph{ready} urządzenia docelowego na stan wysoki. W chwili, gdy oba sygnały znajdują się w tym stanie, właściwe dane mogą zostać przesłane z urządzenia źródłowego do docelowego. Pozwala to na przekazanie takich danych jak adres odczytu/zapisu do pamięci, odczytywanych lub zapisywanych danych i potwierdzenia zapisu. Proces nawiązania transakcji odbywa się niezależnie dla każdego wykorzystywanego kanału.

Procedura odczytu danych składa się z dwóch etapów:
\begin{enumerate}
	\item Zdefiniowanie  przez urządzenie \emph{master} adresu i parametrów transmisji danych na kanale \emph{Read Address}.
	\item Przesłanie przez urządzenie \emph{slave} jednej lub więcej wartości danych na kanale \emph{Read Data}.
\end{enumerate}

Natomiast procedura zapisu wymaga trzech etapów:
\begin{enumerate}
	\item Zdefiniowanie  przez urządzenie \emph{master} adresu i parametrów transmisji danych na kanale \emph{Write Address}.
	\item Przesłanie przez urządzenie \emph{master} jednej lub więcej wartości danych na kanale \emph{Write Data}.
	\item Przesłanie przez urządzenie \emph{slave} odpowiedzi na kanale \emph{Write Response}.
\end{enumerate}

Protokół pozwala ponadto na przesłanie do 256 wartości danych w trakcie jednej transmisji dzięki technice \emph{burst}, a transakcje odczytu i zapisu danych mogą odbywać się równolegle.

Przepływ danych w interfejsie AXI4-Stream odbywa się wyłącznie w jednym kierunku i nie jest możliwy odczyt danych przesłanych wcześniej przez urządzenie \emph{master} do \emph{slave}. Procedura ta jest podobna do transakcji zapisu, jest jednak rozszerzona o możliwość dzielenia operacji na kilka mniejszych i łączenia wielu transakcji w jedną.

\subsection{AXI DMA}
DMA (ang. \emph{Direct Memory Access}) to technika często stosowana w przypadku konieczności wykonywania operacji na pamięci RAM urządzenia z dużą częstotliwością. Wykorzystanie kontrolera DMA pozwala przeprowadzać operacje odczytu i zapisu do pamięci operacyjnej bez konieczności użycia głównej jednostki procesora. Dzięki temu, procesor odpowiada wyłącznie za skonfigurowanie kontrolera DMA i może wykonywać inne operacje w trakcie transmisji danych. Ponadto, stosowanie kontrolera DMA pozwala zwykle na uzyskanie wyższej przepustowości komunikacji z pamięcią i zmniejszenie zużycia energii. Kontroler DMA może również przeprowadzać podstawowe operacje konwersji sygnałów, na przykład, w przypadku sygnału wizyjnego, konwersję sygnałów synchronizacji obrazu. 

DMA pozwala na przesłanie wielu wartości danych w ramach jednej transakcji w trybie \emph{burst}. \emph{Master} przesyła wyłącznie adres pierwszego bajta danych, a kolejne adresy wyznaczane są w trakcie operacji przez urządzenie \emph{slave}. Wyznaczany adres może być inkrementowany, w przypadku, gdy operacja wykonywana jest w pamięci, bądź mieć stałą wartość, co ma miejsce w przypadku zapisu lub odczytu z kolejki FIFO (\emph{First In, First Out}). Interfejs pozwala również na ograniczenie dostępnej przestrzeni adresowej, w efekcie czego wartość adresu po przekroczeniu górnej granicy zakresu przyjmuje ponownie najniższą dopuszczalną wartość. Własność ta może być wykorzystana do projektowania linii buforujących.

Protokół AXI DMA wykorzystuje kolejność bitów, w której najmniej znaczący bajt umieszczony jest jako pierwszy.

Dzięki zastosowaniu techniki DMA możliwa jest konfiguracja parametrów pracy algorytmu zaprojektowanego w układzie logiki reprogramowalnej oraz obserwacja jego działania na etapie wykonania z poziomu procesora ARM. W szerszej perspektywie, pozwala to na udostępnienie interfejsu użytkownika, umożliwiającego nadzór nad pracą algorytmu, na przykład z poziomu konsoli dostępnej przez \emph{ssh} lub w formie interfejsu strony internetowej. Możliwe jest również wykorzystanie modułu umożliwiającego przesyłanie z poziomu logiki FPGA notyfikacji do procesora ARM w celu wymuszenia jego reakcji lub powiadomienia o osiągnięciu zadanego stanu, na przykład przesłanie informacji o ukończeniu iteracji algorytmu dla aktualnej ramki obrazu. Można w tym celu wykorzystać mechanizm przerwań systemowych, co szerzej opisano w sekcji \ref{sec:axi-interrupts}.

Mechanizm DMA zbadano na przykładzie projektu modułu umożliwiającego modyfikację parametrów oraz odczyt aktualnego stanu parametrów. Schemat strukturalny modułu przedstawiono na rysunku \ref{fig:axi-dma-diagram}.

\begin{figure}[h]
	\centering
	\includegraphics[]{img/algorithm-parameters.png}
	\caption{Graficzna reprezentacja modułu AXI DMA w programie Vivado.}
	\label{fig:axi-dma-diagram}
\end{figure}

Proces projektowania oraz komunikacji z modułem przedstawiono w sekcji \ref{sec:vivado-axi-dma}.

Moduł wyposażony jest w interfejs AXI, podpisany \texttt{ctl} oraz związane z nim sygnały, zegarowy - \texttt{ctl\_aclk} oraz reset - \texttt{ctl\_aresetn}. Sygnały wyjściowe pozwalają na odczyt zdefiniowanych parametrów z poziomu innych modułów logiki reprogramowalnej. Dzięki wydzieleniu modułu odpowiedzialnego za konfigurację algorytmu z części wykonującej obliczenia algorytmiczne, możliwe jest uproszczenie kodu języka opisu sprzętu związanego z każdym z modułów oraz zwiększenie czytelności schematu. Jeden moduł konfiguracyjny może być związany z kilkoma, działającymi niezależnie, modułami algorytmicznymi. Ponadto, zmiany w strukturze algorytmu są uproszczone.


\subsection{AXI Video DMA}
\label{sec:axi-vdma}
Interfejs AXI VDMA pozwala na wykorzystanie techniki DMA w przypadku aplikacji przetwarzających sygnał wizyjny.

Mechanizm \emph{Video DMA} oparty został na wykorzystaniu protokołu AXI w wersji Stream oraz Memory Mapped w połączeniu z techniką DMA do buforowania sygnału wizyjnego. 
Sygnał wizyjny przesyłany jest do modułu przy użyciu protokołu strumieniowego, gdzie następnie jest buforowany i zapisywany do komórek pamięci RAM. Zapisany obraz może być odczytany z poziomu procesora ARM. Moduł wspiera również komunikację w drugą stronę, pozwalając na odczyt obrazu z pamięci i przesłanie go dalej w postaci strumienia. Połączenie tych technik pozwala na wykorzystanie modułu do buforowania obrazu lub w celu rozdzielenia zadań algorytmicznych pomiędzy FPGA i CPU.

Moduł VDMA pozwala na zdefiniowanie do trzydziestu dwóch buforów ramek obrazu. Operacje mogą być wykonywane cyklicznie na każdym buforze lub stale na jednym z nich. Pozwala to na wielokrotną transmisję jednej klatki obrazu.

Powszechnie wykorzystywanym zastosowaniem modułu jest mechanizm potrójnego buforowania, umożliwiający zmianę częstotliwości taktowania zegara sygnału wizyjnego. Zapis i odczyt danych może odbywać się niezależnie z tego samego lub różnych buforów. Dzięki zastosowaniu trzech buforów, zagwarantować można, że zapis i odczyt danych zawsze odbywa się z niezależnych obszarów pamięci, co pozwala uniknąć zjawiska nadpisania przechowywanych danych przed ich wyświetleniem.

W niniejszej pracy rozpatrzono możliwość wykorzystania modułu VDMA w celu obsługi algorytmów wymagających kontekstu w postaci dwóch kolejnych ramek obrazu.

\section{OpenMP}

\section{Przerwania systemowe}
\label{sec:axi-interrupts}

\chapter{Analiza sygnału wizyjnego przy użyciu platformy Zynq i systemu PetaLinux}
\label{cha:project}

%kamera usb?

% wejscie -> frame diff ->(bg model?) -> median -> | cpu | -> scale? -> index? (every n frames?) -> save to file -> webserver

Analiza sygnału wizyjnego to proces ekstrakcji informacji opisujących strumień danych przy użyciu algorytmu wizyjnego. Celem analizy jest zwykle redukcja informacji do stanu, który pozwala opisać istotne z punktu widzenia przyjętego zadania przy użyciu możliwie najmniejszej liczby parametrów. Sygnał wejściowy ma zwykle postać obrazu kolorowego, a wynik opisany jest jako obraz binarny lub zbiór cech.

Analiza sygnału wizyjnego stanowi rozwinięcie zagadnienia analizy pojedynczego obrazu. W przypadku algorytmów przetwarzania sekwencji obrazów, na etapie analizy jednej klatki wykorzystać można informacje uzyskane w trakcie obliczeń dla poprzednich ramek sygnału. Rozszerzenie kontekstu o parametry historyczne pozwala na projektowanie bardziej zaawansowanych algorytmów analizy obrazów, zwykle wymaga jednak wykorzystania modułów zewnętrznej pamięci w celu przechowywania danych historycznych.

Wśród algorytmów wymagających kontekstu związanego z więcej niż jedną ramką obrazu wyróżnić można między innymi:
\begin{itemize}
	\item detekcję obiektów pierwszoplanowych -- umożliwia podział obrazu na elementy tła oraz znajdujące się na pierwszym planie, pozwalając zwykle ograniczyć obszar analizy sygnału do fragmentów, z którymi związane są obiekty pierwszoplanowe,

	\item indeksację i śledzenie obiektów -- indeksacja pozwala na przypisanie etykiet do obiektów i wyznaczenie zbioru niezależnych elementów obrazu. Pozwala to śledzić ruch każdego z obiektów oraz analizę zachowań,
	
	\item wyliczanie przepływu optycznego -- pozwala na analizę ruchu obiektów znajdujących się w kadrze, umożliwiając estymację kształtu, odległości czy parametrów ruchu.
\end{itemize}

Implementacja wymienionych typów algorytmów w architekturze potokowej może być utrudniona lub niemożliwa bez użycia zewnętrznego elementu pamięciowego. Ponadto, końcowa analiza wyników algorytmu w architekturze FPGA jest odgórnie ograniczona do przewidywanych parametrów działania systemu -- na przykład, zagadnienie śledzenia obiektów może być ograniczone do maksymalnej zadanej liczby niezależnych elementów.

Ze względu na te ograniczenia, korzystny może okazać się podział algorytmu na niezależne etapy, wykonywane przez elementy logiki reprogramowalnej lub procesor ARM. 

Klasyczny sekwencyjny element obliczeniowy pozwala na adaptację algorytmu do zmieniających się w czasie parametrów obrazu -- na przykład na analizę ruchu zmieniającej się liczby obiektów pierwszoplanowych.

Ponadto, użycie systemu operacyjnego pozwala wykorzystać zaawansowane możliwości prezentacji i przechowywania wyników działania algorytmu wizyjnego, na przykład prezentację wyników przy użyciu interfejsu sieciowego lub zapis wyników do bazy danych.

W poniższym rozdziale zaproponowano metody wykorzystania platformy Zynq na przykładzie wybranych elementów systemów wizyjnych.

\section{Moduł wyznaczania różnicy sekwencji obrazów}

Wyznaczenie różnicy pomiędzy dwoma kolejnymi ramkami strumienia wizyjnego wymaga wyznaczenia dla każdego piksela wartości różnicy, opisanej formułą \ref{eq:frame-difference}.

\begin{equation}
\label{eq:frame-difference}
d^i(x,y) = | p^i(x,y) - p^{i-1}(x,y) |
\end{equation}
gdzie:
\begin{conditions}
	x,y & współrzędne piksela, \\
	i & indeks ramki w sekwencji obrazów, \\
	p^i(x,y) & wartość w $i$-tej ramce dla piksela o współrzędnych $(x,y)$, \\
	d^i(x,y) & wyznaczana wartość różnicy. \\
\end{conditions}

Sygnał źródłowy i wynikowy  mają zwykle charakter obrazu przedstawionego w odcieniach szarości. 
Zagadnienie wyznaczania różnicy dwóch kolejnych obrazów w sekwencji może stanowić przykład algorytmu, którego realizacja w systemach potokowych, pomimo niskiej złożoności obliczeniowej, może być utrudniona. W praktycznych realizacjach, konieczne jest wykorzystanie modułów pamięci operacyjnej w celu zapamiętania ramki obrazu.

Architekturę strumieniową realizującą omawiane zadanie przedstawiono na schemacie \ref{fig:frame-difference}.

\begin{figure}[h]
	\centering
	\def\svgwidth{\textwidth}
	\input{img/frame-difference.pdf_tex}
	\caption{Schemat architektury obliczającej różnicę sekwencji obrazów.}
	\label{fig:frame-difference}
\end{figure}

Wykorzystano moduł AXI VDMA w roli bufora sygnału, opóźniającego dane o pełen cykl strumieniowania ramki obrazu.

Realizacja techniczna bufora wymagała zaprojektowania mechanizmu synchronizacji dwóch niezależnych klatek sygnału wizyjnego. W tym celu wykorzystano moduł kolejki FIFO dla protokołu AXI4-Stream oraz dedykowany element synchronizujący kanał odczytu z bufora VDMA z sygnałem rozpoczęcia nowej ramki obrazu strumienia wejściowego.

Założono, że algorytm będzie wykorzystywany w systemach wizyjnych czasu rzeczywistego, działających w architekturze potokowej. 

Aplikację przystosowano do działania z sygnałem wizyjnym o dowolnej rozdzielczości o częstotliwości transmisji, składającym się z jednego lub wielu kanałów obrazu.

Zastosowano kolejkę FIFO o długości 128 elementów oraz linie buforujące związane z modułem VDMA o tej samej długości.

W celu weryfikacji działania elementu wyznaczającego różnicę sekwencji obrazów zaprojektowano strukturę rozszerzoną o elementy umożliwiające komunikację przy użyciu protokołu AXI oraz przepływ sygnału wizyjnego. Zaprojektowano aplikację umożliwiającą konfigurację modułu w trybie bare-metal oraz przy współpracy systemu PetaLinux.

Sprawdzono działanie aplikacji dla sygnału wizyjnego o rozdzielczości $1280 \times 720$ pikseli i częstotliwości sześćdziesięciu ramek na sekundę.

Szacowane zapotrzebowanie wynikowego systemu na energię elektryczną nie powinno przekroczyć $1,86W$. Właściwa energia wymagana do przeprowadzania operacji obliczeniowych nie przekracza wartości $1,723W$, w tym $1,559W$ ( $90\%$) to energia wymagana do obsługi układu ARM.

W tabeli \ref{tab;frame-difference-utilization} przedstawiono zapotrzebowanie na zasoby FPGA układu ZYBO.

\begin{table}[h]
	\caption{Wykorzystanie zasobów przez aplikację.}
	\centering
	\label{tab;frame-difference-utilization}
	\begin{tabular}{|l|c|c|c|}
		\hline
		\textbf{Rodzaj zasobu} & \textbf{Użycie} & \textbf{Dostępne} & \textbf{Procent użycia}      \\ \hline
		FF                     & $3059$            & $17600$             & $17,38\%$                 \\ \hline
		LUT 6                  & $5721$            & $17600$             & $32,51\%$                 \\ \hline
		SLICE                  & $2550$            & $4400$             & $57,95\%$                 \\ \hline
		DSP 48                 & $0$               & $80$                & $0\%$                    \\ \hline
		BRAM                   & $6$               & $60$                & $10\%$                   \\ \hline
	\end{tabular}
\end{table}

Moduł przeznaczony jest do pracy z częstotliwością $200MHz$, co pozwala na analizę sygnału wideo o rozdzielczości $1920 \times 1080$ pikseli i częstotliwości obrazu $60Hz$.

\section{Moduł generacji tła}
Poza najprostszymi przypadkami analizy ruchu, wyznaczenie różnicy obrazów wewnątrz sekwencji wizyjnej nie stanowi informacji wystarczającej do analizy strumienia obrazów. Zagadnienie to może być jednak elementem składowym bardziej rozbudowanych algorytmów, na przykład modułów realizujących algorytm generacji tła.

Generacja tła to zadanie ekstrakcji elementów \textit{tła} badanego obrazu, a więc takich, które stanowią stały, niezmienny element sceny. Dzięki wydzieleniu obiektów tła, pozostałe elementy obrazu klasyfikowane są jako obiekty pierwszoplanowe. Zwykle, uważa się za nie elementy będące w ruchu. Bardziej zaawansowane metody generacji tła uwzględniają ponadto dodatkowe warunki klasyfikacji obiektów do dwóch z omawianych grup:
\begin{itemize}
	\item cienie -- choć mogą być związane zarówno z elementami tła jak i pierwszoplanowymi, oczekiwane jest zwykle, by nie były uwzględniane w grupie obiektów wymagających analizy,
	\item ruchome elementy tła -- występujące na przykład pod wpływem wiatru ruchy roślin czy deszcz nie powinny być traktowane jako obiekty pierwszoplanowe,
	\item obiekty o niejednorodnym ruchu -- algorytm powinien klasyfikować poprawie obiekty pierwszoplanowe, które pojawiają się na scenie a następnie zatrzymują, nie traktując ich jako elementy tła,
	\item obiekty wzajemnie przesłaniające się -- elementy pierwszego planu mogą, w wyniku ruchu, zostać zasłonięte z perspektywy kamery przez elementy tła. Nie powinno to wpłynąć na zmianę klasyfikacji obiektów z obu grup.
	\item warunki oświetlenia -- możliwość zmiany warunków oświetlenia może wymagać ciągłej korekty parametrów generowanego tła. Uwzględnić należy zarówno zmiany długookresowe, wynikające na przykład z cyklu dobowego, jak i krótkookresowe, wynikające z nagłych zmian, takich jak włączenie lub wyłączenie sztucznego oświetlenia sceny.
\end{itemize}

Zagadnienie realizacji tła nie jest trywialne i wymaga metod uwzględniających część lub wszystkie z wymienionych powyżej ograniczeń. Opracowanie dostępnej literatury poruszającej ten temat znaleźć można w pracy \cite{Kryjak2012}.

W ramach niniejszej pracy zdecydowano się na realizację modułu generacji tła przy pomocy metody modelu tła z bezwładnością, opisanej zależnością \ref{eq:background-model}.

\begin{equation}
\label{eq:background-model}
b^i(x,y) = \alpha p^i(x,y) + (1-\alpha)b^{i-1}(x,y)
\end{equation}
gdzie:
\begin{conditions}
	x,y & współrzędne piksela, \\
	i & indeks ramki w sekwencji obrazów, \\
	p^i(x,y) & wartość w $i$-tej ramce dla piksela o współrzędnych $(x,y)$, \\
	b^i(x,y) & wartość w $i$-tej ramce dla piksela modelu tła o współrzędnych $(x,y)$, \\
	\alpha & współczynnik bezwładności tła z przedziału $(0, 1]$. \\
\end{conditions}

Wadą przedstawionej metody jest jej wrażliwość na krótkookresowe zmiany oświetlenia. Jedną z metod eliminacji zakłóceń występujących cyklicznie jest wprowadzenie możliwości zmiany modelu tła. Stosując kilka niezależnych modeli, budować można warianty obejmujące zbiór najczęściej występujących historycznie danych, uporządkowanych według prawdopodobieństwa wystąpienia. W takim przypadku obliczenia prowadzone są dla każdego modelu tła niezależnie, wartość nie jest jednak aktualizowana w przypadku, gdy stan piksela nie jest zbliżony do oczekiwanego.

Nie zdecydowano się na realizację opisanej metody eliminacji zakłóceń, uzasadniając to zachowaniem czytelności implementacji.

Algorytm dostosowano do pracy z sygnałem opisanym w przestrzeni barw \textit{YCbCr}. Procedura generacji tła odbywa się niezależnie dla każdej składowej sygnału.

Przyjęto, że aktualizacja wartości modelu tła powinna mieć miejsce wyłącznie w przypadku, jeśli aktualnie badany piksel może być uznany za element tła. W tym celu wprowadzono dwa warunki wykonania obliczeń:
\begin{enumerate}
	\item Warunek ruchu.
	
	Aktualizacja powinna mieć miejsce wyłącznie w przypadku, jeśli spełniony jest wartość piksela nie uległa zmianie większej niż dopuszczalna względem poprzedniej ramki obrazu. W przeciwnym razie przyjąć można, że nastąpił ruch elementu i nie należy on do tła. Zależność opisano wzorem \ref{eq:background-model-movement-mask}.
	
	\begin{equation}
	\label{eq:background-model-movement-mask}
	d^i_Y(x,y) > T_{fd}
	\end{equation}
	gdzie:
	\begin{conditions}
		d^i_Y(x,y) & różnica sekwencji dla kanału $Y$ obrazu, opisana wzorem \ref{eq:frame-difference}, \\
		T_{fd} & współczynnik bezwładności ruchu z zakresu $[0,255]$, zwykle nie przekraczający $30$. 
	\end{conditions}

	Większe wartości współczynnika $T_{fd}$ pozwalają dokonać aktualizacji modelu tła dla elementów o coraz większej różnicy względem poprzedniej ramki obrazu.
	
	\item Warunek tła.
	
	Aktualizacja powinna mieć miejsce wyłącznie w przypadku, jeśli piksel może zostać sklasyfikowany jako element tła na bazie aktualnego modelu. Zależność opisano równaniem \ref{eq:background-model-background-mask-1}.
	\begin{equation}
	\label{eq:background-model-background-mask-1}
	m^i_Y(x,y) + 2m^i_{Cb}(x,y) + 2m^i_{Cr}(x,y) > T_{bg}
	\end{equation}
	gdzie:
	\begin{conditions}
		m^i(x,y) & zmiana wartości piksela względem tła, opisana zależnością \ref{eq:background-model-background-mask-2}, \\
		T_{bg} & współczynnik bezwładności przynależności do tła z zakresu $[0,255]$. \\
	\end{conditions}
	
	\begin{equation}
	\label{eq:background-model-background-mask-2}
	m^i_k(x,y) = | p^i_k(x,y) - b^{i-1}_k(x,y)|
	\end{equation}
	gdzie:
	\begin{conditions}
		k & identyfikator kanału sygnału wizyjnego. \\
	\end{conditions}
	
	Większe wartości parametru $T_{bg}$ pozwalają na aktualizację modelu tła w sytuacji, gdy różnica piksela względem aktualnego modelu tła jest znaczna. Jego wartość nie przekracza jednak zwykle $30$.
\end{enumerate}

Aktualizacja modelu tła powinna mieć miejsce wyłącznie w sytuacji, gdy spełnione są oba warunki przedstawione powyżej.

Schemat blokowy algorytmu przestawiono na rysunku \ref{fig:background-model}.

\begin{figure}[h]
	\centering
	\def\svgwidth{\textwidth}
	\input{img/background-model.pdf_tex}
	\caption{Schemat architektury wyliczającej model tła.}
	\label{fig:background-model}
\end{figure}

Algorytm wymaga wykorzystania dwóch buforów AXI VDMA. Jeden z nich przeznaczony jest do buforowania ramki obrazu wejściowego, natomiast drugi przechowuje aktualny model tła. Alternatywą jest zastosowanie wspólnego bufora i przechowywanie w nim dwóch scalonych sygnałów.

Algorytm zintegrowano z układem umożliwiającym komunikację z procesorem ARM, aby umożliwić transmisję uzyskanego modelu tła i jego dalszą analizę. Wykorzystano w tym celu trzeci moduł AXI VDMA. W praktycznych zastosowaniach moduł ten może okazać się zbędny, ze względu na to, że w jednym z pozostałych modułów VDMA przechowywany jest model tła dla poprzedniej klatki obrazu. Opóźnienie jednego cyklu nie powinno wpłynąć negatywnie na jakość działania aplikacji. Niezależny moduł VDMA pozwala jednak na przesyłanie wyników również w przypadku, gdy algorytm generacji tła nie stanowi ostatniego etapu obliczeń.


Ze względu na duże zapotrzebowanie algorytmu na elementy obliczeniowe logiki reprogramowalnej, zdecydowano się ograniczyć rozmiar kolejek FIFO do $64$ elementów.

Sprawdzono działanie aplikacji dla sygnału wizyjnego o rozdzielczości $1280 \times 720$ pikseli i częstotliwości sześćdziesięciu ramek na sekundę.

Szacowane zapotrzebowanie wynikowego systemu na energię elektryczną nie powinno przekroczyć $1,936W$. Właściwa energia wymagana do przeprowadzania operacji obliczeniowych nie przekracza wartości $1,797W$, w tym $1,565W$ ( $87\%$) to energia wymagana do obsługi układu ARM.


W tabeli \ref{tab;background-model-utilization} przedstawiono zapotrzebowanie na zasoby FPGA układu ZYBO.

\begin{table}[h]
	\caption{Wykorzystanie zasobów przez aplikację.}
	\centering
	\label{tab;background-model-utilization}
	\begin{tabular}{|l|c|c|c|}
		\hline
		\textbf{Rodzaj zasobu} & \textbf{Użycie} & \textbf{Dostępne} & \textbf{Procent użycia}      \\ \hline
		FF                     & $6938$            & $17600$             & $39,42\%$                 \\ \hline
		LUT 6                  & $13670$            & $17600$             & $77,67\%$                 \\ \hline
		SLICE                  & $4400$            & $4400$             & $100\%$                 \\ \hline
		DSP 48                 & $15$               & $80$                & $18,75\%$                    \\ \hline
		BRAM                   & $12$               & $60$                & $20\%$                   \\ \hline
	\end{tabular}
\end{table}

Proporcjonalnie duże zużycie zasobów wynika z konieczności wykorzystania wielu modułów AXI VDMA oraz innych elementów wykorzystujących interfejs AXI. Może ono być ograniczone przez wykorzystanie jednego modułu do obsługi buforowania zarówno ramki obrazu, jak i modelu tła. W przypadku złożonych aplikacji, konieczne może okazać się użycie układu o większym zbiorze dostępnych zasobów obliczeniowych.

Moduł przeznaczony jest do pracy z częstotliwością $200MHz$, co pozwala na analizę sygnału wideo o rozdzielczości $1920 \times 1080$ pikseli i częstotliwości obrazu $60Hz$.
\section{Integracja z systemem PetaLinux}
Na etapie prototypowania, elementy logiki reprogramowalnej kontrolowane były przez aplikację działającą w trybie bare-metal, bez wsparcia dla systemu operacyjnego.

Po zakończeniu tego etapu, możliwe stało się zaprojektowanie aplikacji działającej pod kontrolą systemu PetaLinux, umożliwiającej wykorzystanie zaawansowanych funkcji systemu.

Założono, że projektowana aplikacja powinna spełniać szereg wymagań:

\begin{itemize}

	\item Konfiguracja modułów AXI i algorytmu.
	
	Podstawowym zadaniem aplikacji powinno być przeprowadzenie wstępnej konfiguracji modułów, wykorzystując w tym celu interfejs AXI. Proces ten powinien mieć miejsce na etapie uruchamiania aplikacji. Aplikacja powinna być też odpowiedzialna za przeprowadzenie procesu konfiguracji parametrów wykonywanego algorytmu wizyjnego.
	
	Ponadto, działanie algorytmu nie powinno zostać przerwane w razie wyłączenia programu.
	
	\item Konfiguracja aplikacji przy użyciu argumentów wiersza poleceń.
	
	Konfiguracja parametrów działania aplikacji, w tym rozmiar przetwarzanych obrazów i parametry algorytmu powinny być konfigurowane przy użyciu argumentów wiersza poleceń.
	
	\item Monitorowanie działania algorytmu.
	
	Aplikacja powinna udostępniać opcję monitorowania stanu elementów algorytmu, ze szczególnym uwzględnieniem modułów AXI VDMA, odpowiedzialnych za buforowanie danych oraz komunikację z procesorem.
	
	\item Zapis wyników pracy algorytmu.
	
	Program powinien być odpowiedzialny za zapis wyników działania algorytmu, na przykład w formie obrazów przechowywanych w pamięci.
	
	\item Wykorzystanie komunikacji sieciowej.
	
	Uruchomienie aplikacji nie powinno wymagać fizycznego dostępu do układu Zynq. Docelowym narzędziem komunikacji jest protokół SSH. Ponadto, aplikacja powinna udostępniać interfejs wykorzystujący protokół HTTP, umożliwiający weryfikację stanu aplikacji przy użyciu przeglądarki internetowej.
	
	\item Kompatybilność z procedurami biblioteki OpenCV
	
	Aplikacja powinna zapewniać zgodność z technikami programowania wykorzystywanymi przeze bibliotekę OpenCV w stopniu umożliwiającym użycie algorytmów biblioteki ze strukturami danych wykorzystywanymi przez program.
\end{itemize}

Zaprojektowano aplikację w języku C, spełniającą przedstawione wymagania.
Program odpowiedzialny jest za konfigurację elementów logiki reprogramowalnej na podstawie wartości przekazanych przy użyciu argumentów wiersza poleceń. Aplikacja jest odpowiedzialna za monitorowanie działania algorytmu i zapis informacji logu do pliku. Ponadto, umożliwia cykliczny zapis obrazów będących wynikiem działania algorytmu do plików graficznych. 

Proces obsługi aplikacji opiera się na wykorzystaniu protokołu SSH, program udostępnia również interfejs HTTP, umożliwiający uzyskanie aktualnych wyników działania algorytmu.

Zbadano możliwość wykorzystania aplikacji w roli elementu obliczeniowego, odpowiedzialnego za przeprowadzenie części obliczeń algorytmicznych. Zaproponowano moduł indeksacji obiektów na bazie generowanego modelu tła. W tym celu wykorzystano procedurę \texttt{cv::connectedComponents} dostępną w bibliotece OpenCV. 

Ze względu na ograniczenia sprzętowe, moduł indeksacji nie był w stanie spełnić wymagań pracy w czasie rzeczywistym dla sygnału wizyjnego o częstotliwości $60Hz$ i rozdzielczości $1280 \times 720 $ pikseli, a jego wydajność nie przekraczała piętnastu ramek na sekundę.

\section*{Podsumowanie}
Zaproponowane rozwiązania projektowe pozwalają na wykorzystanie części funkcji systemu operacyjnego, które badane były w ramach pracy. Szczególnie istotnym zagadnieniem jest komunikacja pomiędzy elementami zaprojektowanymi w dwóch architekturach. Dzięki wykorzystaniu transmisji danych, możliwe jest zaprojektowanie algorytmów podzielonych na moduły wykonywane naprzemiennie przez obie części układu, wykorzystując atuty obu architektur do możliwie maksymalnego zwiększenia wydajności pełnego algorytmu.

Ponadto, wykorzystanie systemu operacyjnego pozwala na realizację zadań zwykle niemożliwych w przypadku projektu aplikacji realizowanych wyłącznie przy użyciu elementów logiki reprogramowalnej lub sterowanych przez aplikację bare-metal. Program działający pod kontrolą systemu operacyjnego umożliwia prowadzenie zadań konfiguracji, kontroli i monitorowania działania aplikacji z wykorzystaniem komunikacji sieciowej.

Wykorzystanie systemu operacyjnego pozwala również na implementację aplikacji w dowolnym języku programowania. Dzięki temu, stosując dedykowane rozwiązania programistyczne, projektowanie aplikacji o rozbudowanych możliwościach zajmuje mniej czasu.

W trakcie realizacji projektu napotkano szereg ograniczeń architektonicznych.
\begin{itemize}
	\item Liczba elementów obliczeniowych układu ZYBO nie pozwala na realizację rozbudowanych rozwiązań algorytmicznych przy użyciu zaproponowanych technik. Ze względu na duże zapotrzebowanie na elementy logiki przez moduły AXI VDMA, buforowanie pełnych ramek obrazu jest kosztowe. W przypadku bardziej rozbudowanych algorytmów, konieczne może być wykorzystanie układu o większych możliwościach lub zastosowanie technik optymalizacji zużycia zasobów.
	
	\item Procesor ARM dostępny w układzie Zynq nie pozwala na realizację algorytmów wizyjnych o dużej złożoności obliczeniowej w  czasie rzeczywistym. Próba wykorzystania rozwiązań biblioteki OpenCV do indeksacji obiektów pierwszoplanowych nie spełniała ograniczeń czasowych dla sygnału o częstotliwości $60Hz$.
	
	W przypadku bardziej złożonych algorytmów, konieczne może być wykorzystanie układu o większej wydajności. Innym rozwiązaniem może być użycie protokołów sieciowych do transmisji danych do elementu obliczeniowego oferującego wydajność wystarczającą do realizacji zadań obliczeniowych. Dla zbioru algorytmów, których realizacja strumieniowa jest znana, możliwe jest również przeniesienie zadań obliczeniowych do elementów logiki reprogramowalnej. W przypadku, gdy żadne z zaproponowanych rozwiązań nie jest możliwe, konieczne jest ograniczenie częstotliwości działania algorytmu do poziomu, dla którego układ obliczeniowy będzie spełniać ograniczenia czasowe.
	
	\item Proces budowy systemu operacyjnego na bazie projektu sprzętowego jest złożony i wymaga dużych nakładów czasowych. Z tego powodu, na etapie projektowania połączeń logiki reprogramowalnej, wykorzystanie aplikacji typu bare-metal pozwala skrócić okres prototypowania.
	
	W konsekwencji, konieczne może być zaprojektowanie dwóch aplikacji związanych z projektem: aplikacji bare-metal, wykorzystywanej na etapie prototypu, oraz programu działającego pod obsługą systemu operacyjnego, projektowanego po ukończeniu implementacji sprzętowej.
	
	Z tego powodu, aplikacje systemu operacyjnego nie pozwalają w pełni zastąpić programów bare-metal i powinny być traktowane jako metoda rozbudowy możliwości projektu.
\end{itemize}

\chapter{Proces konfiguracji modułów logiki programowalnej i systemu operacyjnego PetaLinux}
\label{cha:vivado-conf}

%TODO Kiepski tytuł dla rodziąłu - mało mówi o jego treści.
%TODO Dla całego rodziału - za krótkie akapity
% OK

Użycie funkcjonalności opisywanych w niniejszej pracy wymaga przeprowadzenia konfiguracji wykorzystywanych modułów logicznych oraz systemu operacyjnego. 
W poniższych podrozdziałach zebrano informacje związane z poruszanymi zagadnieniami. Przedstawiono proces konfiguracji bazowego projektu Vivado i wykorzystania go na etapie budowy systemu PetaLinux, a także omówiono proces użycia modułów AXI i wybranych funkcjonalności systemu PetaLinux -- obliczeń równoległych, biblioteki OpenCV, mechanizmu przerwań systemowych. Przedstawiono również proces konfiguracji modułu generacji tła zaproponowanego w rozdziale \ref{cha:project}.%TODO bardziej konkretnnie
% OK

Omawiane zagadnienia wymagają użycia oprogramowania Vivado \cite{vivado-home} i zintegrowanego z nim środowiska programistycznego Xilinx SDK \cite{xsdk-home}. Ponadto, konfiguracja systemu operacyjnego odbywa się przy użyciu narzędzi pakietu PetaLinux Tools \cite{petalinux-tools}.

\section{Podstawowa konfiguracja projektu}

Wykorzystana w projekcie karta -- Digilent ZYBO -- nie jest bezpośrednio wspierana przez środowisko Vivado. %TODO karta
% OK
Wynika z tego konieczność sprecyzowania parametrów układu na etapie tworzenia projektu. 
W trakcie modyfikowania projektu, w przypadku dodania modułów wykorzystujących interfejs wejścia/wyjścia, konieczna jest również konfiguracja parametrów interfejsu. 
Aby uprościć proces projektowania, zalecane jest skonfigurowanie obsługi układu przed utworzeniem projektu. 
Proces ten opisano w dokumentacji producenta \cite{zybo-in-vivado}.

\subsection{Vivado}
\label{sec:vivado-conf}
Utworzyć należy projektu typu \emph{,,RTL Project''}, z odznaczoną opcją \emph{,,Do not specify sources at this time''}.
W kroku \emph{,,Add Constraints''} dodać należy plik konfiguracyjny dla wybranego układu. 
W przypadku ZYBO, plik ten jest dostępny na stronie producenta. 
W kolejnym kroku możliwe jest skonfigurowanie parametrów układu. 
Wykorzystać do tego należy zakładkę \emph{,,Boards''} i wybrać wykorzystywany model.

Po utworzeniu projektu, skonfigurować należy właściwą przestrzeń roboczą dla projektu, wykorzystując do tego opcję \emph{,,IP Integrator --> Create Block Design''}.
Do nowo utworzonej przestrzeni dodać należy moduł IP reprezentujący procesor ZYNQ -- \emph{,,ZYNQ7 Processing System''}.
W kolejnych krokach należy dokonać konfiguracji modułu procesora, klikając dwukrotnie na moduł.

\begin{itemize}
	\item Kanały interfejsu AXI mogą być konfigurowane przez zakładkę \emph{,,PS-PL Configuration''}. Możliwa jest aktywacja kanałów ogólnego przeznaczenia (\emph{GP}) oraz wysokiej wydajności (\emph{HP}).
	
	\item Interfejsy komunikacji konfigurowane mogą być z poziomu zakładki \emph{,,MIO Configuration''}. Zalecane jest aktywowanie interfejsów \emph{ENET 0}, \emph{SD 0} i \emph{UART 1} ze względu na ich wykorzystanie na dalszym etapie pracy.
	
	Przykład konfiguracji interfejsów wejścia/wyjścia przedstawiono na rysunku \ref{fig:vivado-mio-configuration}.
	\begin{figure}[ht]
		\centering
		\includegraphics[height=8cm]{img/vivado/mio-configuration.png}
		\caption{Okno konfiguracji interfejsów wejścia i wyjścia.}
		\label{fig:vivado-mio-configuration}
	\end{figure}
	
	\item Parametry sygnałów zegarowych dostępnych z poziomu układów logiki reprogramowalnej modyfikować można w zakładce \emph{,,Clock Configuration/PL Fabric Clocks''}. W projekcie projekcie wymagającym obsługi AXI VDMA wykorzystano trzy sygnały zegarowe:
	\begin{itemize}[label=\textbullet]
		\item bazowy, o częstotliwości $100$MHz,
		\item wykorzystywany do komunikacji interfejsem AXI, o częstotliwości $140$MHz,
		\item zegar obsługi sekwencji wizyjnej, o częstotliwości $200$MHz, umożliwiający współpracę ze strumieniem wideo o częstotliwości $60$Hz i rozdzielczości co $1920 \times 1080$ pikseli.
	\end{itemize}
	
	\item Częstotliwość pracy procesora oraz pamięci zmienić można w zakładce \emph{,,Clock Configuration/Processor/Memory Clocks''}. Zdefiniować należy częstotliwość pracy CPU równą $650$MHz oraz DRR równą $525$MHz.

	%TODO Jakie mają być te częstotliwości ?
	% dodałem
\end{itemize}

Po ukończeniu etapu konfiguracji procesora i powrocie do głównego okna programu, należy użyć opcji \emph{,,Run Block Automation''}. 
Utworzone zostaną połączenia interfejsów pamięci \texttt{DDR} oraz \texttt{FIXED\_IO}.
W przypadku zdefiniowania interfejsów AXI, połączyć należy właściwe sygnały zegarowe. 
Przykład wynikowej konfiguracji projektu przedstawiono na rysunku \ref{fig:vivado-config-result}.

	\begin{figure}[ht]
		\centering
		\includegraphics[]{img/vivado/vivado-config-result.png}
		\caption{Okno projektu.}
		\label{fig:vivado-config-result}
	\end{figure}
	
Przedstawiona konfiguracja stanowi podstawę każdego projektu wykorzystującego moduł procesora Zynq.
Po zakończeniu konfiguracji, wygenerować należy warstwę HDL, korzystając z opcji \emph{,,Create HDL Wrapper''} dostępnej po kliknięciu prawym przyciskiem myszy na utworzony wcześniej plik źródłowy.
Skonfigurowany w ten sposób projekt może być budowany i uruchamiany na platformie Zybo.

\subsection{SDK}

W celu utworzenia projektu aplikacji w SDK, konieczne jest wyeksportowanie plików opisujących projekt z poziomu Vivado, wykorzystując do tego opcję \emph{,,File/Export/Export Hardware''} z zaznaczoną opcją \emph{,,Include bitstream''}.
W efekcie, dostępny powinien być projekt \texttt{\textit{nazwa\_projektu}\_hw\_platform\_0}, zawierająca plik \texttt{nazwa\_projektu.hdf}, zawierający konfigurację sprzętową, stanowiący podstawę każdego budowanego programu \textit{bare-metal}. %TODO to jest aplikacja, czy konfiguracja sprzętowa ?
%OK, poprawiłem

W przypadku budowania aplikacji na platformę PetaLinux, na etapie tworzenia projektu, zmodyfikować należy pole \emph{,,OS Platform''} na wartość \emph{,,linux''}, \emph{,,Processor Type''} na \emph{,,ps7\_cortexa9''} oraz wybrać właściwy język programowania.
Tak zdefiniowana aplikacja nie może korzystać z bibliotek udostępnianych przez firmę \emph{Xilinx} )(na przykład zawierających procedury obsługi modułów VDMA czy kontrolera przerwań), ale mogą korzystać z pełnej biblioteki standardowej jeżyka \emph{C} oraz rozszerzeń \emph{POSIX}, w tym operacji wejścia/wyjścia, komunikacji sieciowej czy bibliotek matematycznych.
%TODO zdanie niejasne
% przeredagowałem

W celu uruchomienia aplikacji systemowej na platformie ZYBO, przeprowadzić należy proces budowania i skopiować wynikowy plik z katalogu \texttt{Debug} lub \texttt{Release} do systemu plików systemu PetaLinux. 
Wykorzystać można do tego narzędzie SSH:

\begin{lstlisting}[breaklines=true]
scp Debug/hello-world.elf root@adres-ip:~/
\end{lstlisting}

Aplikację uruchomić można przy użyciu konsoli użytkownika, również stosując narzędzie SSH.

\subsection{PetaLinux}
\label{sec:petalinux-config}

%TODO - a tu nie trzeba zrobić czegoś wcześniej ? Pobrać zainstalowć itp ?
% nie opisywałem procesu instalacji Vivado, uznałem więc, że nie ma też powodu by opisywać instalację petalinux. Dodam na początku rozdziału informację o wykorzystywanych narzędziach

Utworzenie struktury katalogów projektu wykonywane jest przy użyciu poniższego polecenia.

\begin{lstlisting}[breaklines=true]
petalinux-create -t project --template zynq --name (*@\textit{nazwa-projektu}@*)
cd (*@\textit{nazwa-projektu}@*)
\end{lstlisting}

Powstała struktura zintegrowana jest z systemem kontroli wersji \emph{git}, co pozwala zachować uporządkowanie danych wewnątrz projektu oraz wersjonowanie. 
Kolejnym krokiem jest zaimportowanie projektu \emph{Vivado}.

\begin{lstlisting}[breaklines=true]
petalinux-config --get-hw-description=(*@\textit{/sciezka/do/projektu/projekt.sdk/}@*)
\end{lstlisting}

Jeśli polecenie wywołane zostało po raz pierwszy dla danego projektu, uruchomione zostanie narzędzie konfiguracyjne, domyślne ustawienia są jednak poprawne.
Konfiguracja projektu odbywa się przy użyciu polecenia \texttt{petalinux-config}.
Skonfigurować należy metodę uruchamiania systemu -- w omawianym przypadku, uruchomienie następuje na bazie plików znajdujących się na karcie SD.
\begin{lstlisting}[breaklines=true]
petalinux-config
Image Packaging Configuration -> Root filesystem type -> SD card
\end{lstlisting}

Należy również zmodyfikować argumenty przekazywane systemowi na etapie uruchamiania, umożliwiając wykorzystanie sterowników do modułów logiki reprogramowalnej.

\begin{lstlisting}[breaklines=true]
petalinux-config
Kernel Bootargs -> dezaktywować opcję Generate boot args automatically i zdefiniować własną wartość
console=ttyPS0,115200 earlyprintk uio_pdrv_genirq.of_id=generic-uio root=/dev/mmcblk0p2 rw rootwait 
\end{lstlisting}
Następnie, przeprowadzić należy proces budowania systemu oraz wygenerować pliki wynikowe.

\begin{lstlisting}[breaklines=true]
petalinux-build
petalinux-package --boot --fsbl images/linux/zynq_fsbl.elf --fpga images/linux/system_wrapper.bit --u-boot --force
petalinux-package --image -c rootfs --format initramfs
\end{lstlisting}

Uruchomienie systemu wymaga przygotowania karty SD -- musi ona posiadać dwie partycje, pierwszą, z etykietą \emph{boot} i systemem plików \emph{fat32}, drugą -- odpowiednio \emph{sys} i \emph{ext4}. 
Pierwsza z nich, zawierająca pliki wymagane na etapie inicjalizacji systemu, musi być poprzedzona 4 MB niezaalokowanej przestrzeni i mieć rozmiar co najmniej 40 MB. 
Druga partycja zawiera pliki systemowe, jej rozmiar powinien wynosić co najmniej kilkaset megabajtów. 
Proces formatowania przeprowadzić można przy użyciu narzędzia \emph{gparted}. Na rysunku \ref{fig:gparted-screen} przedstawiono efekt konfiguracji karty SD.
%TODO może screen z gparted jak to wygląda.
% OK

\begin{figure}[H]
	\centering
	\includegraphics[width=12cm]{img/gparted-screen.png}
	\caption{Partycjonowane karty SD przy użyciu programu gparted.}
	\label{fig:gparted-screen}
\end{figure}

Pliki wynikowe należy przenieść na kartę SD, korzystając z poniższych poleceń.

\begin{lstlisting}[breaklines=true]
rm -rf /(*@\textit{punkt-montowania}@*)/sys/*
cp images/linux/BOOT.BIN /(*@\textit{punkt-montowania}@*)/boot/
cp images/linux/image.ub /(*@\textit{punkt-montowania}@*)/boot/
cp images/linux/rootfs.cpio /(*@\textit{punkt-montowania}@*)/sys/
cd /(*@\textit{punkt-montowania}@*)/
pax -rvf rootfs.cpio
sync
cd -
\end{lstlisting}

Ze względu na mechanizm buforowania przez kontroler operacji zapisu danych, pamiętać należy o wywołaniu polecenia \texttt{sync}, zapewniającego zachowanie integralności danych.

Karta SD pozwala na uruchomienie systemu operacyjnego na układzie i przechowywanie danych użytkownika pomiędzy startami układu. 
Dalsza praca z systemem odbywać się może przez protokoły komunikacji \emph{SSH} lub \emph{UART}.

\section{Konfiguracja modułu wykorzystującego interfejs AXI}
\label{sec:vivado-axi-dma}

%TODO 1-2 zdania wstępu
% OK
Interfejs AXI stanowi podstawową metodę komunikacji pomiędzy modułami logiki programowalnej oraz elementami FPGA i CPU. Dzięki wykorzystaniu tego protokołu do obsługi projektowanych narzędzi, możliwe jest użycie zbioru elementów dostępnych w bibliotece Vivado i uproszczenie procesu projektowania procedur komunikacyjnych. Poniżej przedstawiono kroki konfiguracyjne wewnątrz narzędzi Vivado i PetaLinux.

\subsection{Vivado}
Oprogramowanie Vivado umożliwia zbudowanie modułu wykorzystującego protokół AXI przez użycie opcji \emph{,,Create and package new IP...''}, zawartej w menu \emph{Tools}.
Na ekranie wyboru zadania wybrać należy opcję \emph{,,Create a new AXI4 peripheral''}.
Po zdefiniowaniu podstawowych danych związanych z modułem, takich jak jego nazwa i nazwisko autora, w kolejnym kroku możliwe będzie zdefiniowanie interfejsu modułu. 
Na tym etapie konfiguracji dodać należy wszystkie połączenia wykorzystujące interfejs AXI. %TODO ????
% OK?
W przypadku modułu konfiguracyjnego o podstawowej strukturze, interfejs zawierać powinien jedno połączenie wykorzystujące protokół AXI w wersji \emph{Lite}, działający w trybie \emph{slave}, z oczekiwaną liczbą rejestrów. 
Każdy rejestr powinien być związany z jedną wartością, której konfiguracja ma być możliwa. 
Przykład konfiguracji przedstawiono na rysunku \ref{fig:axi-dma-interfaces-conf}.
%TODO jest Pan pewnien, że taki moduł to DMA ?
% xilinx nazywa takie moduły axi dma peripherial, to trochę mylące względem właściwego modułu AXI DMA. Sprecyzuję to jakoś...

\begin{figure}[H]
	\centering
	\includegraphics[width=8cm]{img/vivado/axi-dma-interfaces-conf.png}
	\caption{Konfiguracja interfejsów modułu AXI DMA.}
	\label{fig:axi-dma-interfaces-conf}
\end{figure}

W omawianym przykładzie zdefiniowano interfejs AXI o nazwie \emph{ctl}, związany z ośmioma rejestrami o długości trzydziestu dwóch bitów w pamięci.
Zdefiniowanie interfejsów kończy proces podstawowej konfiguracji modułu. W kolejnym kroku należy wybrać opcję \emph{,,Edit IP''} w celu dostosowania kodu źródłowego modułu.

Po wygenerowaniu, z modułem powinien być związany jeden plik źródłowy, zwierający instrukcje odpowiadające za obsługę komunikacji przy użyciu interfejsu AXI.
Do pliku dodać należy elementy odpowiedzialne za zdefiniowanie wyjść modułu oraz przypisanie im właściwych wartości.
W celu zadeklarowania wyjść modułu, odpowiadające im wpisy należy umieścić po komentarzu \emph{,,// Users to add ports here''}. Przykład przedstawiono na listingu \ref{listing:axi-dma-outputs}. %TODO powt. zdefinowane
% OK

\begin{lstlisting}[breaklines, label=listing:axi-dma-outputs, caption=Definicja interfejsów wyjściowych modułu.]
// Users to add ports here
output wire parameter_a,
output wire [7:0] parameter_b,
output wire [15:0] parameter_c,
output wire [31:0] parameter_d,
// User ports ends
\end{lstlisting}

Zdefiniowano cztery sygnały wyjściowe, o różnej liczbie bitów.
Następnie, należy dokonać modyfikacji kodu odpowiedzialnego za powiązanie wartości parametrów z rejestrami modułu. 
Rejestry AXI zdefiniowane są poniżej linii \emph{,,//-- Number of Slave Registers N''}, gdzie \emph{N} to liczba dostępnych rejestrów. Rejestry te mają nazwy \texttt{slv\_reg\emph{n}}, gdzie \emph{n} to indeks rejestru -- nie jest zalecana modyfikacja tych nazw.
Modyfikacji kodu należy dokonać poniżej linii \emph{,,// Add user logic here''}. 
Przykład przedstawiono na listingu \ref{listing:axi-dma-associate}.

\begin{lstlisting}[breaklines, label=listing:axi-dma-associate, caption=Powiązanie wyjść z rejestrami modułu.]
// Add user logic here
assign parameter_a = slv_reg0[0];
assign parameter_b = slv_reg1[7:0];
assign parameter_c = slv_reg2[15:0];
assign parameter_d = slv_reg3[31:0];
// User logic ends
\end{lstlisting}

Wartości parametrów powiązano bezpośrednio z danymi znajdującymi się w rejestrach. %TODO powt. przed
% OK?
W rozbudowanych aplikacjach może być konieczne dodanie instrukcji modyfikujących wartości rejestrów przed przesłaniem ich na wyjście modułu.
Po ukończeniu modyfikacji modułu, konieczne jest zapisanie zmian i wygenerowanie plików wynikowych. 
W tym celu należy wykorzystać okno \emph{,,Package IP''}, sekcję \emph{,,Review and Package''}. 
Widok narzędzia przedstawiono na rysunku \ref{fig:axi-dma-review-package}.

\begin{figure}[ht]
	\centering
	\includegraphics[width=12cm]{img/vivado/axi-dma-review-package.png}
	\caption{Okno finalizacji modyfikacji modułu.}
	\label{fig:axi-dma-review-package}
\end{figure}

Należy wybrać opcję \emph{,,Merge changes''}, umożliwiającą zintegrowanie wprowadzonych zmian z projektem bazowym. 
Następnie, można zakończyć edycję projektu przez wybór opcji \emph{,,Re-Package IP''}. 
Moduł będzie dostępny z poziomu interfejsu wyszukiwania modułów IP.
Dodanie modułu do projektu wymaga zdefiniowania adresu pamięci z nim związanego. Wykorzystać do tego należy okno \emph{,,Address Editor''}, dostępną w z poziomu głównego okna projektu. W omawianym przykładzie, z modułem powiązano przestrzeń rozpoczynającą się od adresu \texttt{0x43000000} i długości \texttt{64K}.
\subsection{SDK}
\label{sec:vivado-axi-dma-sdk}

Konfiguracja wartości parametrów modułu opiera się na zapisie pod właściwe adresy pamięci. %TODO nie lepiej pod właściwy adres w pamięci ?
% OK
W przypadku pracy w trybie \textit{bare-metal}, wykorzystać można instrukcję \emph{Xil\_Out32} z biblioteki \emph{xil\_io.h}. 
W przypadku pracy z systemem PetaLinux, wykorzystać należy biblioteki systemowe. 
Implementację\textit{ bare-metal} przedstawiono na listingu \ref{lis:axi-dma-bare-metal}, natomiast systemową na listingach \ref{lis:axi-dma-petalinux-main}, \ref{lis:axi-dma-petalinux-axi-h} i \ref{lis:axi-dma-petalinux-axi-c}.

\begin{lstlisting}[breaklines, language=C, label=lis:axi-dma-bare-metal, caption=Obsługa modułu w trybie bare-metal.]
#include "xparameters.h"
#include "platform.h"
#include "xil_io.h"

#define PARAMETER_A_REGISTER 0
#define PARAMETER_B_REGISTER 4
#define PARAMETER_C_REGISTER 8
#define PARAMETER_D_REGISTER 12

#define BASEADDR XPAR_ALGORITHM_PARAMETERS_0_CTL_BASEADDR

int main()
{
	init_platform();
	
	Xil_Out32(BASEADDR + PARAMETER_A_REGISTER, 1);
	Xil_Out32(BASEADDR + PARAMETER_B_REGISTER, 25);
	Xil_Out32(BASEADDR + PARAMETER_C_REGISTER, 1 << 10);
	Xil_Out32(BASEADDR + PARAMETER_D_REGISTER, 1 << 30);
	
	while(1);
}
\end{lstlisting}


\begin{lstlisting}[breaklines, language=C, label=lis:axi-dma-petalinux-main, caption=Obsługa modułu w trybie systemowym - \texttt{main.c}.]
#include <fcntl.h>
#include <stdio.h>
#include <stdlib.h>
#include <sys/mman.h>

#include "axi.h"

#define PARAMETER_A_REGISTER 0
#define PARAMETER_B_REGISTER 4
#define PARAMETER_C_REGISTER 8
#define PARAMETER_D_REGISTER 12

#define BASEADDR 0x43000000

typedef int memory_handle_t;

void setup_virtual_memory(struct axi_interface *interface, size_t length, memory_handle_t memory_handle, off_t base_addr) {
	interface->base_addr = base_addr;
	interface->virt_addr = (virt_address) mmap(NULL, length, PROT_READ | PROT_WRITE, MAP_SHARED, memory_handle, base_addr);
	if (interface->virt_addr == MAP_FAILED) {
		perror("Failed to map virtual memory.");
		exit(1);
	}
}

int main() {
	memory_handle_t memory_handle = open("/dev/mem", O_RDWR | O_SYNC);
	
	struct axi_interface* parameters = (struct axi_interface*) malloc(sizeof(struct axi_interface));
	if (parameters == NULL) {
		perror("Memory allocation failed.");
		exit(1);
	}
	setup_virtual_memory(parameters, 65535, memory_handle, BASEADDR);
	
	axi_write(parameters->virt_addr, PARAMETER_A_REGISTER, 1);
	axi_write(parameters->virt_addr, PARAMETER_B_REGISTER, 25);
	axi_write(parameters->virt_addr, PARAMETER_C_REGISTER, 1 << 10);
	axi_write(parameters->virt_addr, PARAMETER_D_REGISTER, 1 << 30);
	
	unsigned int parameter_a = axi_read(parameters->virt_addr, PARAMETER_A_REGISTER);
	
	while(1);
}
\end{lstlisting}

\begin{lstlisting}[breaklines, language=C, label=lis:axi-dma-petalinux-axi-h, caption=Obsługa modułu w trybie systemowym - \texttt{axi.h}.]
typedef unsigned int* virt_address;

struct axi_interface {
	unsigned int base_addr;
	virt_address virt_addr;
};

void axi_write(virt_address virt_addr, int location, unsigned int value);
unsigned int axi_read(virt_address virt_addr, int location);
\end{lstlisting}

\begin{lstlisting}[breaklines, language=C, label=lis:axi-dma-petalinux-axi-c, caption=Obsługa modułu w trybie systemowym - \texttt{axi.c}.]
#include "axi.h"

void axi_write(virt_address virt_addr, int location, unsigned int value) {
	virt_addr[location >> 2] = value;
}

unsigned int axi_read(virt_address virt_addr, int location) {
	return virt_addr[location >> 2];
}
\end{lstlisting}

%TODO proszę jednak zamieścić jakiś komentarz do tych kodów. Szczególnie Petalinux. Oraz opis testów.
% OK
System Linux udostępnia zbiór procedur związanych z obsługą pamięci operacyjnej, w tym pozwalające na wirtualizację fizycznych adresów, co jest konieczne do obsługi urządzeń peryferyjnych z poziomu systemu operacyjnego.
Adres fizyczny urządzenia zdefiniowano przez nazwę \texttt{BASEADDR}. Określono również przesunięcia adresów kolejnych rejestrów modułu, na przykład \texttt{PARAMETER\_A\_REGISTER}.

Procedura \texttt{setup\_virtual\_memory} przyjmuje jako argumenty wskaźnik do struktury interfejsu AXI, zawierającej informacje o adresach fizycznym i wirtualnym pamięci, rozmiarze przestrzeni adresowej w bajtach, a także uchwyt do kontrolera pamięci systemowej oraz adres fizyczny modułu. W wyniku konfiguracji, obszar pamięci fizycznej mapowany jest w przestrzeni adresowej systemu, co pozwala na odczyt i zapis wartości.
Aby uprościć mechanizm odczytu i zapisu wartości do pamięci, udostępniono procedury \texttt{axi\_write} i \texttt{axi\_read}, których argumentami wywołań jest bazowy adres wirtualny modułu oraz przesunięcie wybranego rejestru.

Poprawność działania komunikacji przetestowano na przykładzie modułu liczącego, kontrolowanego z poziomu systemu operacyjnego. Moduł udostępniał rejestr kontrolny, którego najmłodszy bit odpowiadał za aktywację pracy licznika, oraz rejestr przechowujący zliczaną wartość. Zapisując i odczytując wartości rejestrów zweryfikowano, że komunikacja z modułem ma prawidłowy przebieg.

\subsection{PetaLinux}
\label{sec:vivado-axi-dma-petalinux}

W celu wykorzystania techniki DMA w aplikacji działającej w systemie PetaLinux, konieczna jest aktywacja właściwych parametrów konfiguracji na etapie budowania systemu. 
W tym celu wykonać należy polecenie:

\begin{lstlisting}[breaklines]
petalinux-config -c kernel
\end{lstlisting}

i aktywować funkcjonalność DMA:

\begin{lstlisting}[breaklines]
Device Drivers -> DMA Engine Support
Device Drivers -> DMA Engine Support -> Xilinx AXI DMAS Engine
\end{lstlisting}

Włączenie sterowników DMA oraz zmodyfikowanie argumentów uruchomienia systemu operacyjnego, opisane w rozdziale \ref{sec:petalinux-config} pozwala na wykorzystanie interfejsu AXI i techniki DMA do komunikacji z modułami logiki programowalnej.

%TODO i to powoduje, że DMA działa ? I ten moduł z rejestrami AXI. Też to proszę jasno opisać.
% OK
\section{Konfiguracja modułu AXI VDMA}
\label{sec:vivado-axi-vdma}

Proces konfiguracji modułu VDMA składa się z kroków podobnych do opisanych w sekcji \ref{sec:vivado-axi-dma}, poświęconej modułom AXI DMA. Poniżej przedstawiono dodatkowe kroki, związane bezpośrednio z konfiguracją modułu VDMA. %TODO sekcji !
% OK

\subsection{Vivado}
Do projektu dołączyć należy moduł \textit{AXI Video Direct Memory Access}. 
Okno konfiguracji związane z nim pozwala na wybór obsługiwanych kanałów:
\begin{itemize}
	\item write (\emph{S2MM}) -- kanał zapisu, pozwalający na transmisję danych z formatu strumieniowego do pamięci operacyjnej,
	\item read (\emph{MM2S}) -- kanał odczytu, umożliwiający konwersję danych przechowywanych w pamięci do strumienia.
\end{itemize}

Ustawienia pozwalają na wybór szerokości strumienia informacji dla jednego piksela, maksymalną liczbę buforów w pamięci oraz długość linii buforujących, związanych z oboma kanałami.
Wartości wielkości strumienia danych oraz liczby buforów związane są ściśle z projektowanym algorytmem, natomiast długość linii buforujących może wpłynąć na stabilność działania systemu. 
Zwiększenie tej wartości może poprawić działanie algorytmu w przypadku, gdy operacje związane z pamięcią operacyjną wykonywane są z opóźnieniem.

Zakładka ustawień zaawansowanych pozwala na zdefiniowanie parametrów związanych ze sterowaniem kanałami transmisji.
Wartość parametru \textit{,,Fsync Options''} w aplikacjach nie wymagających zewnętrznej synchronizacji powinna być zdefiniowana jako \texttt{tuser} dla kanału zapisu oraz \texttt{none} dla kanału odczytu, dzięki czemu sygnał synchronizacji modułu będzie związany z wejściowym strumieniem AXI. %TODO zewnętrznej sytuacji ?
% poprawiłem
Część aplikacji może wymagać synchronizacji strumienia odczytu z drugim strumieniem danych, na przykład z inną ramką sygnału wizyjnego. 
W takiej sytuacji wykorzystać należy opcję synchronizacji \texttt{fsync}, a wejście układu \texttt{mm2s\_fsync} połączyć z właściwym sygnałem synchronizacji.

W ramach pracy wykorzystywano również synchronizację pomiędzy kanałami przy użyciu parametru \texttt{GenLock}, o wartości \texttt{master} dla kanału zapisu i \texttt{slave} dla odczytu. 
Pozwalało to zachować przesunięcie o stałej, definiowanej z poziomu aplikacji, wartości pomiędzy buforami wykorzystywanymi przez oba kanały.

Ze względu na dużą wartość przepływu danych przez oba kanały, do komunikacji z procesorem wykorzystać należy połączenia o wysokiej wydajności. Kanały te można aktywować korzystając z opcji konfiguracyjnych modułu \emph{ZYNQ7 Processing System}: \emph{,,PL-PS Configuration/HP Slave AXI Interface''}, i aktywując jeden lub wiele kanałów. %TODO Zynq Processing System...
% OK
Z modułem AXI VDMA powiązać należy sygnał zegarowy o częstotliwości nie mniejszej od wartości tak zwanego zegara piksela, związanego ze strumieniem wizyjnym na wejściu. %TODO (tzw. zegarem piksela)
% OK
Sygnał ten powinien być generowany przez układ ZYNQ, a nie powiązany bezpośrednio z zegarem strumienia obrazu.

\subsection{SDK}
Konfiguracja modułu VDMA wymaga zastosowania technik opisanych w sekcji \ref{sec:vivado-axi-dma-sdk}. %TODO sekcji

Proces uruchamiania transmisji dla modułu wymaga wykonania kroków zdefiniowanych przez producenta i opisanych w sekcji \textit{,,Programming Sequence''} dokumentacji \cite{axi-vdma-guide}. Kod programu odpowiedzialnego za konfigurację modułu AXI VDMA przedstawiono na listingu \ref{lis:axi-vdma-sdk}.
\begin{lstlisting}[breaklines,language=C, label=lis:axi-vdma-sdk, caption=Obsługa modułu AXI VDMA w aplikacji \textit{bare-metal}.]
#include "xparameters.h"
#include "platform.h"
#include "xil_printf.h"
#include "xil_io.h"
#include "sleep.h"
#include "xaxivdma.h"

#define S2MM_VDMACR 0x30
#define S2MM_VDMASR 0x34
#define S2MM_VSIZE 0xA0
#define S2MM_HSIZE 0xA4
#define S2MM_FRMDLY_STRIDE 0xA8
#define S2MM_START_ADDRESS 0xAC
#define PART_PTR_REG 0x28
#define MM2S_VDMACR 0x00
#define MM2S_VDMASR 0x04
#define MM2S_VSIZE 0x50
#define MM2S_HSIZE 0x54
#define MM2S_FRMDLY_STRIDE 0x58
#define MM2S_START_ADDRESS 0x5C
#define HSIZE_FULL 1980
#define VSIZE_FULL 750

#define HSIZE_ACTIVE 1280
#define VSIZE_ACTIVE 720

#define PIXEL_SIZE 3
static volatile u8 framebuffer1[VSIZE_ACTIVE*HSIZE_ACTIVE*PIXEL_SIZE] = {0};
static volatile u8 framebuffer2[VSIZE_ACTIVE*HSIZE_ACTIVE*PIXEL_SIZE] = {0};
static volatile u8 framebuffer3[VSIZE_ACTIVE*HSIZE_ACTIVE*PIXEL_SIZE] = {0};

XAxiVdma AxiVdmaFrameBuffering;

void debug_vdma(UINTPTR addr);

void init_vdma_buffer(UINTPTR addr, u32 fb1, u32 fb2, u32 fb3)
{
	// 1
	Xil_Out32(addr + MM2S_VDMACR,
	(255 << 16) | 0x8 | 0x80 | 0x2 | 0x1);
	Xil_Out32(addr + S2MM_VDMACR,
	(255 << 16) | 0x8 | 0x80 | 0x2 | 0x1);
	
	// 2
	Xil_Out32(addr + MM2S_START_ADDRESS, fb1);
	Xil_Out32(addr + MM2S_START_ADDRESS + 4, fb2);
	Xil_Out32(addr + MM2S_START_ADDRESS + 8, fb3);
	
	Xil_Out32(addr + S2MM_START_ADDRESS, fb1);
	Xil_Out32(addr + S2MM_START_ADDRESS + 4, fb2);
	Xil_Out32(addr + S2MM_START_ADDRESS + 8, fb3);
	
	// 3
	Xil_Out32(addr + MM2S_FRMDLY_STRIDE,HSIZE_ACTIVE*PIXEL_SIZE);
	Xil_Out32(addr + S2MM_FRMDLY_STRIDE,HSIZE_ACTIVE*PIXEL_SIZE | (2 << 24));
	
	// 4
	Xil_Out32(addr + MM2S_HSIZE,HSIZE_ACTIVE*PIXEL_SIZE);
	Xil_Out32(addr + S2MM_HSIZE,HSIZE_ACTIVE*PIXEL_SIZE);
	
	// 5
	Xil_Out32(addr + S2MM_VSIZE,VSIZE_ACTIVE);
	Xil_Out32(addr + MM2S_VSIZE,VSIZE_ACTIVE);
}

int main()
{
	init_platform();
		
	xil_printf("Config - vdma\r\n");
	// vdma frame buffering
	init_vdma_buffer(XPAR_FRAME_BUFFER_VDMA_PREVIOUS_FRAME_BASEADDR,
	(u32)&framebuffer1, (u32)&framebuffer2, (u32)&framebuffer3);

	xil_printf("Config - done\r\n");
	while(1) {
		xil_printf("XPAR_FRAME_BUFFER_VDMA_PREVIOUS_FRAME\r\n");
		debug_vdma(XPAR_FRAME_BUFFER_VDMA_PREVIOUS_FRAME_BASEADDR);
		sleep(1);
	}
	cleanup_platform();
	return 0;
}
\end{lstlisting}

%TODO może jednak przykładowy listing z komentarzem co i jak...
% OK

Funkcja \texttt{init\_vdma\_buffer} jest odpowiedzialna za przeprowadzenie konfiguracji modułu VDMA. Jej argumenty stanowią adres elementu VDMA oraz adresy trzech buforów obrazu. Zagwarantować należy, by przestrzeń zarezerwowana na każdy bufor była wystarczająca do przechowania pełnej ramki obrazu. W przypadku wykorzystania modułu VDMA do transmisji kontekstu związanego z każdym pikselem, zmodyfikować należy wartość \texttt{PIXEL\_SIZE} do liczby bajtów zajmowanej przez jeden piksel.

Procedura \texttt{init\_vdma\_buffer} wykonuje szereg operacji związanych z uruchomieniem transmisji sygnału dla obu kanałów niezależnie. Przedstawione poniżej działania związane są z indeksami znajdującymi się wewnątrz komentarzy w listingu.
\begin{enumerate}
	\item Konfiguracja przerwań i uruchomienie kanału VDMA.
	W prezentowanym przykładzie, oba kanały skonfigurowano do działania w trybie cyklicznym, korzystającym naprzemiennie ze wszystkich buforów obrazu, z aktywowanym trybem synchronizacji \emph{Genlock} o wewnętrznym źródle sygnału.
	
	\item Przypisanie adresów buforów obrazu. Mogą być one wspólne lub unikalne dla każdego kanału.
	
	\item Zdefiniowanie parametrów obrazu wejściowego z sygnałami wygaszania i wzajemnego opóźnienia kanałów.
	W omawianym przykładzie, kanał odczytu zachowuje opóźnienie dwóch klatek obrazu względem kanału zapisu.
	
	\item Przypisanie rozmiaru jednej linii obrazu, z uwzględnieniem wielkości piksela w bajtach, nie uwzględniając cykli wygaszania.
	
	\item Zdefiniowanie liczby linii obrazu, nie uwzględniając cykli wygaszania.
\end{enumerate}

Wpisanie wartości do rejestrów \texttt{S2MM\_VSIZE} i \texttt{MM2S\_VSIZE} powoduje rozpoczęcie transmisji sygnału.

\subsection{Petalinux}
Konfiguracja projektu zgodna jest z opisem dla modułów DMA, przedstawionym w sekcji \ref{sec:vivado-axi-dma-petalinux}. %TODO sekcji.

W przypadku projektu aplikacji działającej pod kontrolą systemu operacyjnego, pamiętać należy, że konfiguracja buforów obrazu wymaga użycia adresów fizycznych, które mogą różnić się od adresów wirtualnych komórek pamięci.

Zdefiniować należy adresy buforów, odległe od siebie co najmniej o rozmiar jednej ramki sygnału wizyjnego. 
Zagwarantować należy nienaruszalność pamięci z perspektywy systemu operacyjnego. 
Efekt ten najprościej jest osiągnąć przez ograniczenie rozmiaru pamięci dostępnej dla systemu operacyjnego i zdefiniowanie adresów buforów poza tym zakresem. 
Wykorzystać do tego można argument \textit{mem} przekazywany na etapie uruchamiania systemu, na przykład \texttt{mem=224M}. 
Proces dodawania argumentów uruchamiania systemu opisano w sekcji \ref{sec:petalinux-config}. %TODO sekacji

Adres fizyczny pamięci nie może być odczytany bezpośrednio, w tym celu musi zostać powiązany z adresem wirtualnym. 
Odpowiedzialną za to procedurę \texttt{setup\_virtual\_memory} przestawiono na listingu \ref{lis:axi-dma-petalinux-main}, przy czym parametr \texttt{base\_addr} to adres fizyczny pierwszej komórki bufora. 

\section{Obliczenia równoległe}
\label{sec:multithreading-config}
Użycie rozwiązań omawianych w rozdziale \ref{sec:openmp} wymaga aktywacji właściwych funkcji kompilacji.
W przypadku zastosowania wątków natywnych lub biblioteki dostępnej w standardzie C++ wymagane jest aktywowanie przełączników: %TODO powt. zast
% OK

\begin{lstlisting}[language=bash]
gcc main.c -o main.out (*@\textbf{-pthread}@*)
g++ main.cpp -o main.out (*@\textbf{-std=c++11}@*) (*@\textbf{-pthread}@*)
\end{lstlisting}

Dla biblioteki TBB wymagane jest przeprowadzenie linkowania względem jej kodu źródłowego:

\begin{lstlisting}[language=bash]
g++ main.cpp -o main.out (*@\textbf{-ltbb}@*)
\end{lstlisting}

Natomiast dla interfejsu OpenMP, konieczne jest użycie przełącznika:

\begin{lstlisting}[language=bash]
g++ main.cpp -o main.out (*@\textbf{-fopenmp}@*)
\end{lstlisting}

Wykorzystanie omawianych rozwiązań wiąże się z użyciem dedykowanych procedur lub dyrektyw kompilatora. 
Etap projektowania aplikacji działającej pod kontrolą systemu operacyjnego PetaLinux nie różni się od budowania oprogramowania na inne platformy. Należy jednak pamiętać, że układ Zynq wyposażony jest w procesor ARM o dwóch rdzeniach, więc potencjalne korzyści zastosowania aplikacji wielowątkowej nie przekraczają dwukrotnego zwiększenia szybkości działania aplikacji.
Sposób zastosowania bibliotek znaleźć można w dokumentacji każdego z narzędzi i literaturze cytowanej w związanym z tym zagadnieniem rozdziale.
%TODO Testował to Pan. Jakiś komentarz ?
% hm... testowałem na tyle, by sprawdzić czy działa. Realnie wykorzystałem wątki do obsługi www w aplikacji, choć trudno mi tutaj dodać jakieś wnioski.

\section{Biblioteka OpenCV}
\label{sec:opencv-config}
\subsection{OpenCV 2}
Biblioteka OpenCV w wersji 2.4 nie jest oficjalnie dostępna w pakiecie PetaLinux, może jednak zostać dołączona do systemu operacyjnego dzięki mechanizmowi aplikacji użytkownika.
Przeprowadzić należy proces kompilacji kodu źródłowego biblioteki wraz z zależnościami, ponieważ prekompilowane pliki na platrofmę ARM nie są publicznie dostępne. Poniżej przedstawiono proces instalacji zależności.%TODO styl. coś jest nie tak...
% prada, poprawiłem

\begin{lstlisting}[breaklines=true, language=Bash, caption=Definicja zmiennych środowiskowych.]
export ARMPREFIX=(*@\textit{ścieżka/instalacji}@*)
export CCPREFIX=arm-linux-gnueabihf-
\end{lstlisting}

Zmienna \texttt{CCPREFIX} wskazuje na prefiks kompilatora zawartego w pakiecie PetaLinux, a zmienna \texttt{ARMPREFIX} wskazuje na ścieżkę, gdzie zainstalowane zostanę pliki wynikowe.

\begin{lstlisting}[breaklines=true, caption=Kompilacja biblioteki \textit{xVideo}.]
wget http://downloads.xvid.org/downloads/xvidcore-1.3.3.tar.gz
tar -zxvf xvidcore-1.3.3.tar.gz
cd xvidcore/build/generic/
./configure --prefix=${ARMPREFIX} --host=arm-linux-gnueabihf --disable-assembly
make
make install
\end{lstlisting}

\begin{lstlisting}[breaklines=true, caption=Kompilacja biblioteki \textit{x264}.]
git clone git://git.videolan.org/x264
cd x264
./configure --enable-shared --host=arm-linux-gnueabihf --disable-asm --prefix=${ARMPREFIX} --cross-prefix=${CCPREFIX}
make
make install
\end{lstlisting}

\begin{lstlisting}[breaklines=true, caption=Kompilacja biblioteki \textit{ffmpeg}.]
git clone git://source.ffmpeg.org/ffmpeg.git
cd ffmpeg
git checkout release/2.6
./configure --enable-cross-compile --cross-prefix=${CCPREFIX} --target-os=linux \
	--arch=arm --enable-shared --disable-static --enable-gpl --enable-nonfree \
	--enable-ffmpeg --disable-ffplay --enable-ffserver --enable-swscale \
	--enable-pthreads --disable-yasm --disable-stripping --enable-libx264 \
	--disable-libxvid --prefix=${ARMPREFIX} --extra-cflags="-I"${ARMPREFIX}"/include" \
	--extra-ldflags="-L"${ARMPREFIX}"/lib"
make
make install
\end{lstlisting}

Kompilowane biblioteki zapewniają dostęp do procedur obsługi strumieni wideo oraz obrazów w najczęściej wykorzystywanych formatach.
Po zainstalowaniu zależności, przystąpić można do pobrania i instalacji biblioteki OpenCV.

\begin{lstlisting}[breaklines=true, caption=Pobieranie biblioteki OpenCV w wersji 2.4.10.]
git clone https://github.com/Itseez/opencv.git
cd opencv
git checkout 2.4.10
\end{lstlisting}

\begin{lstlisting}[breaklines=true, caption=Kompilacja biblioteki \textit{OpenCV}.]
mkdir build && cd build
cmake -DBUILD_DOCS=OFF -DBUILD_TESTS=OFF -DWITH_1394=OFF -DWITH_CUDA=OFF \
	-DWITH_CUFFT=OFF -DWITH_EIGEN=OFF -DWITH_GSTREAMER=OFF -DWITH_GTK=OFF \
	-DWITH_JASPER=OFF -DWITH_JPEG=OFF -DWITH_LIBV4L=OFF -DWITH_OPENEXR=OFF \
	-DWITH_PNG=OFF -DWITH_PVAPI=OFF -DWITH_TIFF=OFF -DWITH_V4L=OFF \
	-DENABLE_PRECOMPILED_HEADERS=OFF -DWITH_FFMPEG=ON \
	-DCMAKE_SYSTEM_NAME=Linux -DCMAKE_SYSTEM_PROCESSOR=arm \
	-DCMAKE_C_COMPILER=arm-linux-gnueabihf-gcc \
	-DCMAKE_CXX_COMPILER=arm-linux-gnueabihf-g++ \
	-DCMAKE_INSTALL_PREFIX=$ARMPREFIX \
	-DCMAKE_FIND_ROOT_PATH=(*@\textit{katalog/zawierający/narzędzia/kompilacji}@*) ../
make
make install
\end{lstlisting}

Aby zmniejszyć rozmiar biblioteki, a także skrócić proces instalacji, część modułów została dezaktywowana. 
Wartość zmiennej \texttt{CMAKE\_FIND\_ROOT\_PATH} to ścieżka zawierająca strukturę katalogów wykorzystywanego kompilatora. 
W przypadku pakietu PetaLinux w wersji 2016.3, właściwa ścieżka względem punktu instalacji pakietu to \path{Xilinx/Petalinux/tools/linux-i386/gcc-arm-linux-gnueabi/arm-linux-gnueabihf}.

Po zakończeniu procesu, pliki wynikowe znaleźć można w katalogu \texttt{\$ARMPREFIX/lib}.

Pliki te mogą być dołączone do budowanego systemu operacyjnego jako dodatkowe zależności. %TODO powt. pliki.
% OK
W tym celu wykorzystać należy polecenie:

\begin{lstlisting}[breaklines=true]
petalinux-create -t libs --template install --name opencv2
\end{lstlisting}

Utworzona zostanie struktura katalogów \path{components/libs/opencv2}, do której skopiować należy pliki wynikowe kompilacji biblioteki i jej zależności. 
Następnie, zmodyfikować należy plik \texttt{Makefile}, zgodnie z zawartymi w nim instrukcjami. 
W przypadku biblioteki OpenCV, wykorzystać można tekst generowany w wyniku wywołania polecenia:

\begin{lstlisting}[breaklines=true]
for f in $(find . -type f -name "*.so*" -printf '%P\n'); \
	do echo -e '\t$(TARGETINST) -d' $f /lib/$f; done
\end{lstlisting}

Aktywacja biblioteki wewnątrz projektu wymaga wywołania polecenia przedstawionego na listingu \ref{lis:petalinux-activate-lib} i wyboru biblioteki w zakładce \textit{,,Libs''}. 

\begin{lstlisting}[caption=Dołączenie biblioteki do projektu PetaLinux., label=lis:petalinux-activate-lib]
petalinux-config -c rootfs
\end{lstlisting}

%TODO ew. nieco lepszy komentarz dla poszczególnych etapów oraz napisać, że to się udało zrobić i działało.
% OK

Bibliotekę skompilowano i potwierdzono poprawność działania dla plików testowych dostępnych na stronie internetowej twórców. Porównano wyniki działania aplikacji modyfikującej obraz na wejściu i zapisującego wynik do pliki z programem uruchomionym na procesorze architektury \emph{x86} i nie stwierdzono różnic.

\subsection{OpenCV 3}
Biblioteka OpenCV w wersji 3.1 dołączona jest do pakietu PetaLinux. 
W celu jej aktywacji, wykorzystać należy polecenie przedstawione na listingu \ref{lis:petalinux-activate-lib} i wybrać biblioteki w zakładce \emph{,,Filesystem Packages/libs/opencv''}.

Działanie biblioteki przetestowano na przykładzie programu dokonującego segmentacji obiektów pierwszoplanowych i ich indeksacji, przedstawionego na listingu \ref{lis:opencv-con-com}.

\begin{lstlisting}[language=C, breaklines=true, label=lis:opencv-con-com, caption=Aplikacja indeksująca obiekty pierwszoplanowe.]
#include <opencv2/opencv.hpp>
#include <iostream>

int main(int argc, char* argv[])
{
	cv::Mat input_image = cv::imread(argv[1], cv::IMREAD_GRAYSCALE);
	
	cv::Mat binary;
	cv::threshold(input_image, binary, 200, 255, 0);
	cv::imwrite("binary.png", binary);
	
	cv::Mat labels, stats, centroids;
	
	int num_labels = cv::connectedComponentsWithStats(binary, labels, stats, centroids);
	
	cv::imwrite("components.png", labels);
	
	for (int l = 1; l < num_labels; l++)
		std::cout << "#" << l << "(x,y) = (" << centroids.at<long double>(l, 0) << ", " << centroids.at<long double>(l, 1) << ")" << std::endl;
	
	return 0;
}
\end{lstlisting}
\subsection{SDK}
Wykorzystanie bibliotek w projekcie SDK wymaga wskazania katalogu ze źródłami oraz bibliotekami w ustawieniach projektu.
W przypadku użycia biblioteki w wersji 3.1, wystarczające jest utworzenie aplikacji w języku C++ i typu \textit{OpenCV Example Application}.
Dla wersji 2.4, konieczne jest ręczne zmodyfikowanie parametrów kompilacji projektu, w sposób analogiczny do konfiguracji aplikacji wykorzystującej OpenCV i działającej na platformie x86, wskazując jednak na skompilowane wcześniej pliki dla platformy ARM.
%TODO jeśli Pan to robił to proszę opisać szczegóły tego procesu.
% OK

Dokonać należy modyfikacji opcji projektu w ścieżce \emph{,,Tool Settings -> ARM v7 Linux g++ compiler -> Directories''} i dodać katalog \texttt{include} znajdujący się w strukturze plików: \texttt{\textit{ścieżka/instalacji}/include}.
Zmodyfikować należy również opcje konfiguracji programu linkującego \emph{,,Tool Settings -> ARM v7 Linux g++ linker -> Libraries''}. Zdefiniować należy ścieżkę poszukiwania bibliotek \texttt{\textit{ścieżka/instalacji}/lib}, a także dodać wszystkie wykorzystywane moduły do listy używanych bibliotek, na przykład \texttt{opencv\_core}, \texttt{opencv\_imgproc}, \texttt{opencv\_video}.

\section{Wykorzystanie mechanizmu przerwań systemowych}
\label{sec:interrupts-config}

Użycie mechanizmu przerwań systemowych wymaga zbudowania połączeń wewnątrz logiki programowalnej oraz konfiguracji agentów przerwań na poziomie aplikacji użytkownika. 
Poniżej opisano kroki wymagane do użycia omawianego mechanizmu w aplikacjach \textit{bare-metal} oraz działających w~systemie PetaLinux.

\subsection{Vivado}
Moduły wspierające mechanizm przerwań wyposażone są w dedykowane połączenia wyjściowe, wykorzystywane do transmisji sygnału przerwania. 
W przypadku modułu AXI Timer właściwe połączenie ma sygnaturę \emph{interrupt}, natomiast w przypadku modułu AXI VDMA, sygnały przerwań dla kanałów odczytu i zapisu mają odpowiednio nazwy \emph{mm2s\_introut} oraz \emph{s2mm\_introut}.

Obsługa przerwań wymaga konfiguracji modułu procesora. 
Aktywować należy ścieżkę \emph{,,Fabric Interrupts --> PL-PS Interrupt Ports --> IRQ\_F2P''} wewnątrz zakładki \emph{Interrupts}. 
W rezultacie, dostępne będzie wejście procesora \emph{IRQ\_F2P} o szerokości do szesnastu linii. 
We wspomnianej zakładce ustawień aktywować można również inne połączenia przerwań, w tym szybkie przerwania w kierunku procesora oraz połączenia prowadzone od procesora do układów logiki, pozwalające na transmisję zdarzeń z interfejsów procesora, takich jak DMA, UART czy Ethernet.

Kanał \emph{IRQ\_F2P} pozwala na połączenie nie więcej niż szesnastu linii przerwań. 
W przypadku wykorzystania mechanizmu na platformie PetaLinux, pierwszym ośmiu liniom, zaczynając od najmłodszego bitu, przypisane będą identyfikatory przerwań w zakresie $[61-68]$, natomiast pozostałym ośmiu -- $[84-91]$.

W przypadku konieczności zaprojektowania interfejsu wykorzystującego więcej niż szesnaście linii przerwań, konieczne jest zastosowanie układu dedykowanego obsłudze zdarzeń -- \emph{,,AXI Interrupt Controller''}. 
Pozwala on na połączenie nie więcej niż trzydziestu dwóch linii przerwań do jednej linii na wejściu procesora i udostępnia interfejs umożliwiający identyfikację układu odpowiedzialnego za wysłanie sygnału przerwania. 
Zapewnia również mechanizmy priorytetyzacji i zagnieżdżania przerwań.

W sytuacji, gdy interfejs nie zawiera więcej niż szesnastu przerwań, wystarczające jest użycie modułu konkatenacji sygnałów zdarzeń do jednego wektora, którego wyjście połączone jest z wejściem \emph{IRQ\_F2P} procesora.

\subsection{Aplikacja \textit{bare-metal}}

Wykorzystanie przerwań wymaga napisania procedury odpowiedzialnej za obsługę zdarzeń oraz zarejestrowanie jej jako agenta danego przerwania.
Ponadto zwykle wymagane jest przeprowadzenie konfiguracji modułu w taki sposób, aby aktywować funkcję zgłaszania przerwań. %TODO emisja... (chyba lepiej generację)
%poprawiłem

Wymagane funkcje znaleźć można w plikach nagłówkowych \texttt{xparameters.h}, \texttt{xscugic.h}, \texttt{xil\_exception.h}, oraz \texttt{xaxivdma.h} dla modułu AXI VDMA i \texttt{xtmrctr.h} dla AXI Timer.

Procedurę konfiguracji obsługi przerwań podzielić można na kilka etapów:
\begin{enumerate}
	\item Zdefiniowanie agentów zdarzeń.
	
Konieczne jest zdefiniowane funkcji, które będą wywołane w przypadku wystąpienia przerwania. 
W najprostszym rozwiązaniu, ich celem jest akceptacja zdarzenia i przeprowadzenie konfiguracji modułu w taki sposób, aby umożliwić jego dalsze działanie -- w przypadku modułu zegarowego jest to wykonanie restartu zegara. 
Moduł AXI VDMA nie wymaga żadnych kroków na etapie wywołania przerwania.

Ponadto, procedura jest odpowiedzialna za wykonanie obliczeń związanych z wystąpieniem przerwania.

Na listingu \ref{lis:interrupt-handlers} przedstawiono funkcje agentów przerwań dla modułu zegara oraz obu kanałów AXI VDMA.

\begin{lstlisting}[breaklines=true, language=C, caption=Procedury obsługi przerwań., label=lis:interrupt-handlers]
void Timer_InterruptHandler(void *data, u8 id) {
	// dodatkowe obliczenia
	
	// zerowanie przerwania
	XTmrCtr_Stop(data, id);
	XTmrCtr_Reset(data, id);
	XTmrCtr_Start(data, id);
}

void AxiRead_InterruptHandler(void *data, u32) {
	// dodatkowe obliczenia
}

void AxiWrite_InterruptHandler(void *data, u32) {
	// dodatkowe obliczenia
}
\end{lstlisting}

	\item Konfiguracja modułów.
	
Oba omawiane moduły wymagają przeprowadzenia dodatkowych kroków konfiguracji. 
W przypadku modułu zegarowego, konieczne jest aktywacja obsługi przerwań w rejestrze kontrolnym -- \texttt{TCSR\textit{n}}, natomiast w przypadku modułu VDMA, parametryzacja odbywa się przez rejestry \texttt{MM2S\_VDMACR} dla kanału zapisu oraz \texttt{S2MM\_VDMACR} dla kanału odczytu.

Ponadto, konieczna jest rejestracja agentów przerwań dla obu modułów. 
Proces ten przedstawiono na listingu \ref{lis:interrupt-handlers-2}, zmienne \texttt{TimerInstancePtr} i \texttt{AxiVdmaInstancePtr} są wskaźnikami do wykorzystywanych struktur typu \texttt{XTmrCtr} i \texttt{XAxiVdma}.

\begin{lstlisting}[language=C, caption=Rejestracja agentów przerwań., label=lis:interrupt-handlers-2]
XAxiVdma_SetCallBack(AxiVdmaInstancePtr, XAXIVDMA_HANDLER_GENERAL,
	&AxiWrite_InterruptHandler, AxiVdmaInstancePtr, XAXIVDMA_WRITE);

XAxiVdma_SetCallBack(AxiVdmaInstancePtr, XAXIVDMA_HANDLER_GENERAL,
	&AxiRead_InterruptHandler, AxiVdmaInstancePtr, XAXIVDMA_READ);

XTmrCtr_SetHandler(TimerInstancePtr, Timer_InterruptHandler, TimerInstancePtr);
\end{lstlisting}

	\item Konfiguracja kontrolera przerwań.
	
W ostatnim kroku następuje konfiguracja kontrolera zdarzeń. Procedurę przedstawiono na listingu \ref{lis:interrupt-controller}.

\begin{lstlisting}[language=C, caption=Konfiguracja kontrolera przerwań., label=lis:interrupt-controller]
XScuGic InterruptController;
XScuGic_Config *GicConfig;
int ScuGicInterrupt_Init(u16 DeviceId, XTmrCtr *TimerInstancePtr,
	XAxiVdma * AxiVdmaIntancePtr) {
	int Status;
	GicConfig = XScuGic_LookupConfig(DeviceId);
	if (NULL == GicConfig)
		return XST_FAILURE;
	
	// a
	Status = XScuGic_CfgInitialize(&InterruptController, GicConfig,
		GicConfig->CpuBaseAddress);
	if (Status != XST_SUCCESS)
		return XST_FAILURE;
	
	Xil_ExceptionRegisterHandler(XIL_EXCEPTION_ID_INT,
		(Xil_ExceptionHandler) XScuGic_InterruptHandler,
		&InterruptController);
	Xil_ExceptionEnable();
	
	// b
	Status = XScuGic_Connect(&InterruptController,
		XPAR_FABRIC_AXI_TIMER_0_INTERRUPT_INTR,
		(Xil_ExceptionHandler) XTmrCtr_InterruptHandler,
		TimerInstancePtr);
	if (Status != XST_SUCCESS)
		return XST_FAILURE;
	
	Status = XScuGic_Connect(&InterruptController,
		XPAR_FABRIC_AXI_VDMA_RESULT_S2MM_INTROUT_INTR,
		(Xil_ExceptionHandler) (XAxiVdma_WriteIntrHandler),
		AxiVdmaIntancePtr);
	if (Status != XST_SUCCESS)
		return XST_FAILURE;
	
	// c
	XScuGic_Enable(&InterruptController,
		XPAR_FABRIC_AXI_TIMER_0_INTERRUPT_INTR);
	
	XScuGic_Enable(&InterruptController,
		XPAR_FABRIC_AXI_VDMA_RESULT_S2MM_INTROUT_INTR);
	XScuGic_Enable(&InterruptController,
		XPAR_FABRIC_AXI_VDMA_RESULT_MM2S_INTROUT_INTR);
	return XST_SUCCESS;
}
\end{lstlisting}

Wewnątrz procedury ma miejsce kilka etapów konfiguracji:
\begin{enumerate}
	\item Uruchomienie kontrolera przerwań i rejestracja agenta zdarzeń, odpowiedzialnego za wstępną obsługę wszystkich zgłaszanych wyjątków.
	
	\item Rejestracja wszystkich modułów logiki programowalnej, które połączone są z wejściem \texttt{IRQ\_F2P} i których przerwania powinny być obsługiwane przez aplikację. Definiowane są również procedury odpowiedzialne za obsługę każdego zdarzenia.
	
	\item Aktywacja kanałów obsługi przerwań. Wykonanie tego kroku rozpoczyna proces oczekiwania kontrolera na zdefiniowane przerwanie.
\end{enumerate}
\end{enumerate}

%TODO nieco obszerniejszy komentarz do kodu.
% OK

\subsection{PetaLinux}

Obsługa przerwań w systemie PetaLinux wymaga wykorzystania dedykowanych sterowników sprzętu i przeprowadzenia przy ich użyciu procesu konfiguracji.
Pakiet PetaLinux udostępnia sterowniki do modułów AXI, które ich wymagają i w ramach niniejszej pracy ograniczono się do ich wykorzystania. 
W przypadku konieczności obsługi przerwania z niestandardowego modułu, konieczne może być dostarczenie dedykowanego mu sterownika, co wymaga specjalistycznej wiedzy z dziedziny działania systemów operacyjnych i komunikacji z urządzeniami peryferyjnymi. %TODO możę nie tyle szerokiej co specjalistycznej
% OK

Aby uzyskać dostęp do modułów zaimplementowanych w logice programowalnej, konieczna jest aktywacja tak zwanych modułów systemowych. %TODO raczej do modułów zaimplementowanych w logice rekong...
% OK
Na etapie konfiguracji systemu operacyjnego aktywować należy poniższe opcje:

\begin{lstlisting}[breaklines=true, caption=Konfiguracja modułów systemowych.]
petalinux-config -c kernel

Device Drivers -> Userspace I/O drivers
Device Drivers -> Userspace I/O drivers -> Userspace I/O platform driver with generic IRQ handling
Device Drivers -> Userspace I/O drivers -> Userspace I/O platform driver with generic iqr and dynamic memory
\end{lstlisting}
%TODO ale na jakim etapie jest to robione ?
%sprecyzowałem

Konieczna jest również znajomość identyfikatorów linii przerwań. 
Można je odczytać z poziomu SDK, po utworzeniu projektu \emph{Board Support Package} dla wykorzystywanej konfiguracji sprzętowej. 
Identyfikatory linii przerwań zdefiniowane są w pliku \texttt{xparameters.h}, na przykład:

\begin{lstlisting}[language=C]
/* Definitions for Fabric interrupts connected to ps7_scugic_0 */
#define XPAR_FABRIC_AXI_VDMA_RESULT_MM2S_INTROUT_INTR 61
#define XPAR_FABRIC_AXI_VDMA_RESULT_S2MM_INTROUT_INTR 62
#define XPAR_FABRIC_AXI_TIMER_0_INTERRUPT_INTR 63
\end{lstlisting}

Wartości te mogą być również znalezione w strukturze \textit{device tree}, generowanej przez pakiet PetaLinux na etapie parametryzacji, w której zdefiniowane są informacje o konfiguracji sprzętowej, wymagane do poprawnego działania systemu.
Wymagane informacje znajdują się w pliku \path{subsystems/linux/configs/device-tree/pl.dtsi}. 
Na listingu poniżej przedstawiono fragment konfiguracji związany z modułem AXI Timer.

\begin{lstlisting}
axi_timer_0: timer@42800000 {
	# ...
	compatible = "xlnx,xps-timer-1.00.a";
	interrupt-parent = <&intc>;
	interrupts = <0 31 4>;
	reg = <0x42800000 0x10000>;
	# ...
};
\end{lstlisting}
Kolejne wpisy w konfiguracji definiują informacje o sterowniku, który powinien być odpowiedzialny za obsługę modułu z poziomu procesora, module odpowiedzialnym za kontrolę przerwań oraz definicję zdarzeń. 
Ostatni wpis zawiera informację o adresie urządzenia w pamięci oraz rozmiarze tego zasobu.
Definicja przerwania zawiera trzy elementy, z których kluczowa jest wartość \texttt{31}. 
Ze względu na specyfikę formatu danych, w celu uzyskania właściwego identyfikatora przerwania, konieczne jest zwiększenie jej o \texttt{32}. Uzyskany wynik -- \texttt{63} -- jest zgodny z definicją wewnątrz pliku \texttt{xparameters.h}.
W razie konieczności zaprojektowania dedykowanego sterownika sprzętu, wymagana jest wiedza na temat struktury \textit{device tree} oraz zasad budowy oprogramowania tego typu. 
Informacje na ten temat znaleźć można we właściwych źródłach \cite{Corbet2005,device-tree-tutorial}.

Pakiet PetaLinux pozwala na dodanie do konfiguracji własnych modułów systemowych. 
W celu utworzenia struktury plików dla nowego modułu, wykorzystać można polecenie:

\begin{lstlisting}[breaklines=true]
petalinux-create -t modules -n (*@\textit{nazwa\_modułu}@*) --enable
\end{lstlisting}

W wyniku działania polecenia utworzona zostanie struktura, którą następnie należy zmodyfikować dodając funkcjonalności sterownika.
Skompilowany na etapie budowania projektu moduł znajduje się w ścieżce \path{/lib/modules/identyfikator-kernela/extra} i może być uruchomiony poleceniem

\begin{lstlisting}
insmod (*@\textit{nazwa\_modułu}@*).ko
\end{lstlisting}

Logowane przez moduł wiadomości mogą być odczytane przy użyciu polecenia \texttt{dmesg}.
W celu weryfikacji poprawności konfiguracji przerwań systemowych, wykorzystać można interfejs \texttt{/proc/interrupts}.
Wszystkie przerwania mogą być wypisane przy użyciu polecenia

\begin{lstlisting}
cat /proc/interrupts
\end{lstlisting}

W przypadku modułu AXI Timer, spodziewany jest wpis o treści:
\begin{lstlisting}
 63:          1          0  axi-timer  40
\end{lstlisting}
Potwierdza on obecność linii przerwania o identyfikatorze \texttt{63}, związanej ze sterownikiem \texttt{axi-timer}, która została wywołana jeden raz w przypadku pierwszego rdzenia procesora.

%TODO Tak samo opisać eksperymenty, jakiś screen. VDMA też działao ? 
%TODO Mam też wrażenie, że nie ma wszystkich informacji jak to przerwanie uruchomić, chyba, że to inaczej działą niż w bare i nie trzeba powiązać funkcji. Ogólnie czy zrobił Pan to samo co w bare-metal ?
% nie testowałem VDMA, nie doszedłem do etapu, w którym potrzebowałbym przerwania z vdma. Tzn na bare-metal działało, jest nawet na listingu aplikacji, więc "powinno" działać i w peta.
% w ogóle, te przerwania działają bardzo dziwnie. Miałem z nimi duże problemy i właściwie nic, poza jednym przypadkiem, nie udało mi się osiągnąć. Stąd tak mocno teoretyczny opis...
% Podejrzewam, że "coś" się zepsuło w jednej z ostatnich edycji peta, bo gdy miałem problemy z konfiguracją, to co prawda znajdowałem rozwiązania w dokumentacji xilinxa lub na forach, autorzy twierdzili, że działają, ale w moim przypadku już tak nie było... 
% I raczej nie jestem z tym sam, bo łatwo znaleźć tematy w tym stylu: https://forums.xilinx.com/t5/Embedded-Linux/UIO-Interrupts-on-Zynq/td-p/765123
% dość swieże i z problemami, które i ja miałem. Ten zalinkowany to jest etap, do którego doszedłem, tzn przerwanie teoretycznie działa, ale nigdy nie jest pobudzane.
% Ogółem, znalazłem jeden program, na którym te przerwania rzeczywiście działały, ale zacząłem wątpić w ich przydatność. Bo w literaturze, którą znalazłem autorzy albo obsługiwali przerwania bezpośrednio z poziomu modułu systemu, drivera - co może być uciążliwe w naszych zastosowaniach, bądź też wykrywali je na zasadzie cyklicznego odpytywania właściwego rejestru modułu - nie było więc feedbacku od modułu do aplikacji, tak jak w bare-metal. A to samo mogę osiągnąć bez wykorzystywania modułów systemowych.

% podsumowując:
% 1) "nie działa" - podchodziłem do tego tematu chyba trzykrotnie i nie udało mi się osiągnąć zbyt wiele
% 2) nie rozumiem praktyczności wykorzystania przerwań w tej formie, którą widziałem. Być może brakuje mi jakiegoś kluczowego fragmentu wiedzy, ale na tyle, na ile wiem teraz, przerwania nie byłyby dla mnie pomocne.
% Dlatego opisałem teorię, by zebrać to, co wiem na przyszłość, może komuś się przyda. Ale praktycznych rozwiązań nie udało mi się zaprojektować...
\section{Konfiguracja algorytmu generacji tła}
\label{sec:background-buffer-conf}

\chapter{Podsumowanie}
\label{chap:podsumowanie}
\lipsum[1-3]

\printbibliography

\appendix
\chapter*{Dodatki}
\addcontentsline{toc}{chapter}{Dodatki}
\renewcommand{\thesection}{\Alph{section}}

\section{Spis zawartości płyty CD}

Dołączona do pracy płyta CD zawiera pliki źródłowe omawianych projektów:
\begin{itemize}
	\item \texttt{ip-repo} --moduły logiki programowalnej, wykorzystane do realizacji projektów wizyjnych:
	\begin{itemize}
		\item \texttt{algorithm\_parameters} -- moduł konfiguracji parametrów algorytmu,
		\item \texttt{background\_model} -- moduł algorytmu generacji tła,
		\item \texttt{frame\_difference} -- moduł odejmowania dwóch ramek obrazu,
		\item \texttt{frame\_synchronizer} -- moduł synchronizujący dwa strumienie \emph{AXI-Stream},
		\item \texttt{rgb2ycbcr} -- moduł odpowiedzialny za konwersję obrazu z przestrzeni bart \emph{RGB} do \emph{YCbCr},
	\end{itemize}
	
	\item \texttt{proj-background-model-petalinux} -- projekt PetaLinux pozwalający na uruchomienie projektów odejmowania ramek i generacji tła,
	\item \texttt{proj-background-model-sdk} -- aplikacje \emph{bare-metal} oraz systemowa związane z algorytmem generacji tła,
	
	\item \texttt{proj-background-model-vivado} -- projekt sprzętowy związany z algorytmem generacji tła,
		
	\item \texttt{proj-frame-difference-sdk} -- aplikacje \emph{bare-metal} oraz systemowa związane z algorytmem odejmowania ramek,
	
	\item \texttt{proj-frame-difference-vivado} -- projekt sprzętowy związany z algorytmem odejmowania ramek,
	
	\item \texttt{proj-rtos-petalinux} -- projekt PetaLinux pozwalający na uruchomienie systemu operacyjnego czasu rzeczywistego,
	
	\item \texttt{proj-frame-difference-vivado} -- projekt sprzętowy pozwalający na uruchomienie systemu operacyjnego czasu rzeczywistego,
	
	\item \texttt{praca-dyplomowa-w-gumula.pdf} -- plik zawierający treść niniejszej pracy.
\end{itemize}

\section{Aplikacja w architekturze NEON}
\label{cha:neon-source}

%TODO by się przydał "lekki" komentarz
% OK

Na listingach poniżej zaprezentowano implementacje procedury wyznaczającej wartość iloczynu skalarnego dwóch wektorów o zadanej długości. Listing \ref{lis:dot-product-base} zawiera implementację bazową, wykorzystującą podstawowe operacje dostępne w języku C.
Listing \ref{lis:dot-product-neon} wykorzystuje funkcjonalności modułu NEON w celu potencjalnego zmaksymalizowania wydajności operacji. Obie implementacje porównano z procedurą napisaną w asemblerze, wykorzystującą instrukcje koprocesora VFP, przedstawioną na listingu \ref{lis:dot-product-asm}.

Sposób wykorzystania procedur w programie przedstawiono na listingu \ref{lis:dot-product-main}. Kompilacja programu wymaga aktywacji przełączników odpowiedzialnych za obsługę instrukcji NEON. Wykorzystano również techniki optymalizacji udostępniane w kompilatorze \textit{gcc}. Polecenie kompilacji przedstawiono poniżej:

\begin{lstlisting}[breaklines=true]
arm-linux-gnueabihf-gcc -Wall -O3 -mcpu=cortex-a9 -mfpu=neon -ftree-vectorize -mvectorize-with-neon-quad -mfloat-abi=hard -ffast-math -funsafe-math-optimizations -g -c -o "main.o" "main.c"
\end{lstlisting}

\begin{lstlisting}[breaklines=true, language=C, caption=Implementacja bazowa., label=lis:dot-product-base]
float dot_product(float *first, float *second, unsigned int len) {
	float sum = 0.0;
	for (unsigned int i = 0; i < len; i++)
		sum += first[i] * second[i];
	return sum;
}
\end{lstlisting}

\begin{lstlisting}[breaklines=true, language=C, caption=Implementacja w architekturze NEON. (Źródło: \cite{xilinx-neon}), label=lis:dot-product-neon]
float dot_product_neon(float * restrict first, float * restrict second, unsigned int len) {
	float32x4_t vec1_q, vec2_q;
	float32x4_t sum_q = {0.0, 0.0, 0.0, 0.0};
	float32x2_t tmp[2];
	float result;
	for( int i=0; i<( len & ~3); i+=4 )
	{
		vec1_q=vld1q_f32(&first[i]);
		vec2_q=vld1q_f32(&second[i]);
		sum_q = vmlaq_f32(sum_q, vec1_q, vec2_q );
	}
	tmp[0] = vget_high_f32(sum_q);
	tmp[1] = vget_low_f32 (sum_q);
	tmp[0] = vpadd_f32(tmp[0], tmp[1]);
	tmp[0] = vpadd_f32(tmp[0], tmp[0]);
	result = vget_lane_f32(tmp[0], 0);
return result;
}
\end{lstlisting}

\begin{lstlisting}[breaklines=true, language=C, caption=Implementacja w asemblerze. (Źródło: \cite{dot-product-asm}), label=lis:dot-product-asm]
float dot_product_asm(float * restrict first, float * restrict second, unsigned int len) {
	float net1D=0.0f;
	asm volatile (
		"vmov.f32 q8, #0.0"
		"1:"
		"subs %3, %3, #4"
		"vld1.f32 {d0,d1}, [%1]!"
		"vld1.f32 {d4,d5}, [%2]!"
		"vmla.f32 q8, q0, q2"
		"bgt 1b"
		"vpadd.f32 d0, d16, d17"
		"vadd.f32 %0, s0, s1"
		: "=w"(net1D)
		: "r"(first), "r"(second), "r"(len)
		: "q0", "q2", "q8");
	return net1D;
}
\end{lstlisting}

\begin{lstlisting}[breaklines=true, language=C, caption=Główna procedura programu., label=lis:dot-product-main]
int main() {
	float first[] = {1.1, 2.2, 3.3};
	float second[] = {-1.1, -2.2, -3.3};
	
	float result_base = dot_product(first, second, 3);
	float result_neon = dot_product_neon(first, second, 3);
	float result_asm = dot_product_asm(first, second, 3);
	
	return 0;
}
\end{lstlisting}

\section{Konwersja danych pomiędzy VDMA i OpenCV}
\label{sec:vdma-to-opencv}

Na listingu \ref{lis:vdma-to-opencv-code} zaprezentowano metodę konwersji sygnału wizyjnego pomiędzy elementami obliczeniowymi wykonanymi w architekturach FPGA i ARM.

\begin{lstlisting}[language=C++, label=lis:vdma-to-opencv-code, caption=Konwersja sygnału wizyjnego pomiędzy AXI VDMA i \texttt{cv::Mat}.]
#include "opencv2/core/core.hpp"

cv::Mat const from_vdma(unsigned char *ptr, std::size_t width, std::size_t height,
	std::size_t bytes_per_pixel)
{
	return cv::Mat(height, width, CV_8UC(bytes_per_pixel), ptr);
}

void to_vdma(cv::Mat const &image, std::size_t bytes_per_pixel, unsigned char *ptr)
{
	assert(image.isContinuous());
	if (ptr != image.ptr())
		std::memcpy(ptr, image.ptr(),
			image.rows * image.cols * bytes_per_pixel);
}

int main(int, char**)
{
	const std::size_t width = 1280, height = 720, bytes_per_pixel = 4;
	unsigned char framebuffer_ptr[width * height * bytes_per_pixel];
	
	for(int i =0; i < 10000;i++)
	{
		cv::Mat image = from_vdma(framebuffer_ptr,
			width, height, bytes_per_pixel);
		
		// opcjonalna kopia
		image = image.clone();
		
		algorithm(image);
		
		to_vdma(image, bytes_per_pixel, framebuffer_ptr);
		
		await_next_frame();
	}
	return 0;
}

\end{lstlisting}


\end{document}
